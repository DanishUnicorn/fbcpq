\chapter*{Course Description}
\setlength{\headheight}{12.71342pt}
\addtolength{\topmargin}{-0.71342pt}

\section*{Education}
MSc Programme in Agriculture

MSc Programme in Environment and Development

MSc Programme on Global Environment and Development

\section*{Content}
The course focuses on developing capacities for sustainable production of tropical crops. The students will be exposed to major crop science elements that are instrumental for a sustainable crop production. Focus is on optimizing the use of agrobiodiversity and management practices considering the socio-economic characteristics and climate change challenges.

\vspace{1em}
Main disciplines are:

\vspace*{0.5em}
i. Agronomy with reference to tropical conditions.

Tropical crop physiology; crop genetic resources, agrobiodiversity and breeding; crop management; crop protection; soil fertility. Cultivation of crops under challenging conditions of climate change (e.g drought, salinity).

\vspace*{0.5em}
ii. Tropical Crops

An overview of major tropical crops groups in relation to their uses (roots and tubers; legumes; minor cereals; spices; stimulants; underutilized species), their intrinsic properties and their cultivation with special emphasis on small-holder conditions and resilience for climate change.

\vspace*{0.5em}
iii. Cropping systems

Crop production optimization strategies for sustainable production (intercropping, use of legumes for mitigation/adaptation). Innovations to optimize sustainable production systems (crop: phenotyping, breeding, protection). The use of agrobiodiversity for diversification, sustainable intensification and value chain enhancement.


\section*{Learning Outcome}
Provide students, having a BSc-level background in agricultural, social sciences or sciences involved with development of the tropical region, with a comprehensive understanding of the properties of selected tropical environments, crop species and their management facing climate change. Focus is on climate related production constraints; that is abiotic and biotic stresses, and human endeavor to optimize crop production in small-scale farming, within the context of poverty alleviation and sustainable crop production.

When students have completed the course, they should have attained:

\subsection*{Knowledge}

\begin{itemize}
    \item Manage key elements to characterize production systems in the tropics
    \item Demonstrate knowledge of the principles of tropical crop production
    \item Understand the characteristics of major tropical crops
    \item Demonstrate overview of tropical cropping systems in relation to agro-ecological and socio-economic conditions
    \item Demonstrate knowledge on different strategies to optimize production systems in the tropics
    \item Manage basic tools for participatory work and research
\end{itemize}

\subsection*{Skills}

\begin{itemize}
    \item Characterize production systems of tropical areas of the globe
    \item Design cropping calendars for selected major crops species
    \item Analyze and synthesize diverse types of information and data on tropical crop production
    \item Apply a relevant analytical software for statistics
    \item Apply relevant participatory rural appraisal methods
    \item Develop tropical crop production plans in relation to given agro-ecological and socioeconomic conditions
    \item Design and analyze the implementation of projects in a tropical crop production environment
\end{itemize}

\subsection*{Competences}

\begin{itemize}
    \item Data management, analysis, and critical approach
    \item Assess and formulate agronomic components of development support programmes
    \item Advice extension and research institutions in tropical countries
    \item Perform and interpret quantitative and qualitative statistical information to analyze scenarios of crop production and innovation
    \item Propose innovative optimization strategies for sustainable crop production in the tropics
\end{itemize}


\section*{Litterature}
Papers and videos uploaded on Absalon

Tropical Crop Production I - Selected papers

Tropical Crop Production II – Manual for practical and theoretical exercises

\section*{Recommended Academic Qualifications}
Basic courses in biology, statistics, social sciences and sciences related to sustainable development

Academic qualifications equivalent to a BSc degree is recommended.

Academic qualifications equivalent to a BSc degree is recommended .


\section*{Teaching and Learning Methods}
The course applies blended learning with lectures supported by videos, digital tools, theoretical and practical exercises.

\section*{Workload}

\begin{table}[h]
    \centering
    \caption{A table with an overview over the workload for the course.}
    \label{tab:workload}
    \rowcolors{2}{white}{gray!7}
    \begin{tabular}{ l | c}
        \textbf{Category} & \textbf{Hours} \\ 
        \hline
        Lectures & 30 \\ 

        Preparation & 68 \\

        Theory exercises & 55 \\ 

        Practical exercises & 24 \\

        Excursions & 7 \\

        Project work & 8 \\

        Guidance & 10 \\ 
        Exam & 4 \\
        \hline
        Total & 206 \\ 
    \end{tabular}
\end{table}

\section*{Exam}

\begin{table}[h]
    \centering
    \caption{A table with an overview over the workload for the course.}
    \label{tab:workload}
    \rowcolors{2}{white}{gray!7}
    \begin{tabular}{ l | p{10cm} }
        Credit & 7.5 ECTS \\ 
        
        Type of assessment & Oral examination, 30 min \\ 

        Type of assessment details & During the course the student participate in group work in which they write a group report (approximate 10 pages). The students are individually examined in the content of the group report and are further examined in the rest of course curriculum. Examination in the report weight 35 \% and examination in curriculum weight 65 \%. No preparation time before the oral examination. \\

        Examination prerequisites & Submitted and approval of the reports for theoretical and practical exercises \\ 

        Aid & All aids allowed \\

        Marking scale & 7-point grading scale \\

        Censorship form &   \begin{itemize}
                                \item No external censorship
                                \item Several internal examiners
                            \end{itemize} \\

        Re-exam &   \begin{itemize}
                        \item As the ordinary exam.

                        \item If the student did not participate in a approved group report, an assignment is given three weeks before the exam. The student has to hand in an individual report based on the assignment (approximate 5 pages). At the oral examination the students will then be examined in the report and in the rest of the curriculum. Examination in the rapport weight 35 \% and examination in curriculum weight 65 \%.
                    \end{itemize} \\ 
    \end{tabular}
\end{table}


\newpage