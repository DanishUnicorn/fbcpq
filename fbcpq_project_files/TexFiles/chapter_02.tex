\setcounter{chapter}{0}
\setcounter{section}{0}
\chapter{Lecture Notes}
\setlength{\headheight}{12.71342pt}
\addtolength{\topmargin}{-0.71342pt}

\section{Lecture 01 - 02/09-2025}
\textbf{Fruit and Berry Crop Physiology and Quality or Fruit and Berry Crop Physiology, Quality and Use}

\textbf{Torben T-A, KU-PLEN}

\vspace{1em}
The course aims to provide broad knowledge on the physiology and growing of fruit and berry crops (including pit-, stone-, cane-, and bush fruits, other berries, nuts, and fruit vegetables), focusing at the whole plant and organ level. Key objectives are the development of specialized knowledge on a course topic, the ability to analyze a crop to identify important aspects in growing and quality, and understanding how fruit quality is influenced, specifically balancing productivity vs. quality. The curriculum seeks to build links between the production and quality of raw materials and the food science aspects of their use, including sensory aspects. The underlying idea is to address both the Plant science and the Food Science aspects of fruits and berries, emphasizing the importance of these crops for human health and addressing lacking knowledge regarding the potentials in different genotypes (raw materials).
The course focuses primarily on temperate fruits, nuts, and berries grown mainly in open field or tunnel systems. While fruit vegetables such as tomato may be included in the individual report, the emphasis includes common commercial systems, organic growing, and relevance for small-scale/home gardening. These crops are characterized as Diverse (many species and cultivars), high value crops, demanding intensive hand labor and resources, and often requiring manipulation at the single plant level. Although often fresh consumed/used, there is a large diversity in processed uses, such as juice, jam, fruit wine, vinegar, cakes, desserts, and ice cream. They are important to human health and provide pleasure through nice aroma and taste.
The course content is structured around eight main topic areas: 

\begin{enumerate}
    \item Genetic basis and Cultivar variation
    \item Yield and quality components and determinants, including Organ development and interactions
    \item Allocation of dry matter and nutrients, focusing on 'Sources' and 'sinks' in fruiting plants
    \item Effects of pre-harvest factors on fruit quality, such as the importance of control of plant vigour and shape
    \item Content and development of secondary compounds in fruits
    \item Maturation, ripening, and assessment of optimal harvest time and quality aspects
    \item Post-harvest usability (fresh consumption, cooking, industry processing, especially juice)
    \item Sensory aspects of quality evaluation
\end{enumerate}

Pedagogic methods utilize a mix of Lectures, Exercises ('Hands on'), Written assignments, Student lectures, and Excursions. Specialization is achieved through the individual report and the student lecture, which provide case specific supplements to the general lectures. The individual report is an opportunity to develop specific competence on a free topic relevant to the course (e.g., Bud dormancy, Aroma development, Canopy management). The report structure mandates an Intro (1-2 pages), Detailed presentation and discussion based on international literature (4-6 pages), and Summary/conclusion (½ - 1 page), not exceeding 8-9 pages in total (max 10 pages including figures). Students must use their own writing and avoid 'copy-paste' and plagiarism.
The assessment is an Oral exam lasting 25 minutes in total. The exam includes a short talk about the individual report, followed by two curriculum questions: one focusing on crop physiology and one on quality aspects. The final grade is based on 50\% the individual report and 50\% the curriculum questions.

\section{Lecture 02 - 05/09-2025}
\textbf{Yield and quality components}
\textbf{Torben T-A, KU-PLEN}

\vspace{1em}
The overarching aims in fruit growing are to maximise fruit yield and maximise fruit quality. The tools utilized to achieve these aims include Genetic choice (cultivars) and Growing technique. Fruit science focuses heavily on the physiological background informing the growing technique. Maximized Yield is typically expressed as tons/Ha (or hekto L/ha of juice or wine) and is fundamentally a product of the components Fruit number X Fruit size. Determinants influence the size or the level of development of single components within the genetic potential. These physiological elements are connected to growth and development and can be influenced by growing techniques, climate, and growing conditions.
Maximized fruit quality is less clearly defined than yield and may vary with fruit species and final use. Important quality parameters include Colour and other aspects of appearance (especially for Fresh consumption), Taste components (sugar, acid, aroma), and 'Health components' (vitamins, phenols ......) which are a focus of research. Fruit size serves as both a yield and a quality component. The presence of unwanted substances such as pesticides is also an important aspect of fruit quality. The optimal compromise is crucial because physiological and technical factors may have opposite effects on yield and quality components. The environmental impact of the growing technique used, such as carbon footprint and pollution, also requires knowledge.
The physiological elements that serve as yield and quality components and determinants include: 

\begin{enumerate} 
    \item Planting system and Growing system (fx Rubus, strawberry), determining Number of plants/ha. 
    \item Plant size and structure, including Elongation growth, shoot type development, Bud development, and Flower bud initiation. 
    \item Bud number/plant, bud type, and the ratio of leaf bud/flower bud. 
    \item Flower development, specifically Number of flowers/cluster and Flower quality (e.g., Number of seed primordia, Position in cluster). 
    \item Pollination, fertilization, and initial set, defining Initial fruit number/flower. 
    \item Fruit drop (June drop), which determines the Final fruit number/initial number of fruits. 
    \item Fruit growth and fruit development, the Leaf/fruit ratio, and specific factors like Number of seeds/fruit (Pollination) and Amount of flesh/seed (in Strawberries), leading to Fruit size and quality. 
    \item Yield/ha (Fruit number x fruit size) and Fruit quality (content). 
\end{enumerate}

Specific crop examples illustrate the variability: Sour cherry yields may vary drastically, from 3 to 18 tons/ha, due to factors like Bud death (sometimes reaching 90\%) and Fruit set percentage. Unstable yields in sour cherry 'Stevnsbær' pose a significant problem for the Market (Industry). In strawberries, vegetative growth (runner production) is influenced by long days and high temperature, while flower cluster formation is promoted by short day and low temperature. Pollination is very important in strawberries as the final fruit size depends on the Achenes/berry. For grapes, yield components include Number berries/cluster x weight/berry and Number of clusters x weight/cluster, influenced by factors like pruning, thinning, and the growing system. Analysis of 51 grape cultivars harvested in 2018 showed a high correlation ($R^2=0.9516$) between Yield kg/plant and total sugar (Fruktose + Glukose g/plante).


\section{Lecture 03 - 08/09-2025}
\textbf{Bud and shoot development}
\textbf{Torben T-A, KU-PLEN}

\vspace{1em}
This lecture explores Bud and shoot development, covering Meristems, Bud types and shoots, and seasonal Growth patterns. Embryogenesis initiates plant development by establishing the primary meristems and determines the Apical $\leftrightarrow$ basal axial development, including the Shoot apex, Hypocotyl (The stem), Root, Root apical meristem, and Root cap. It also establishes Radial patterning, such as Epidermal cells, Cortical tissue, and the Vascular cylinder. The initial stages of development include Cotyledons (The first leaves). Development transitions from meristem to shoot, detailing the structure of the shoot apex in 3D and the Transition from vegetative to floral meristem. Variation exists among species in how strong the xylem or wood develop, with Trees having strong development and Bushes and canes exhibiting weak development. The purposes of leaves include Production of assimilates (the source of carbon), Water transpiration resulting in uptake and transport of water and nutrients, Production of hormones, and Control of water status/cooling.
Buds are defined as locations of growth with a potential for development. The Types (fates) of buds include Vegetative, Generative, Sleeping (not dead), and Dead. Buds determine the plant's dimensions by forming shoots of different lengths, such as Short shoots (spurs), emphasizing the Importance of the shoot's position. In strawberries, the vegetative bud results in a side crown or a 'runner'. Pome fruits, exemplified by apple and pear, possess mixed buds, while Stonefruits utilize 'naked' buds. The topography of flowerbuds differs between pome- and stonefruit, which raises questions about which structure bears flowers in the terminal position on short shoots and the Importance of this difference.
Seasonal growth patterns involve a Flush of growth in spring and early summer, followed by Growth termination or indefinite growth, depending on the species. Terminal bud formation also occurs, alongside Adventive buds formed along with shoot development. The timing of terminal bud formation depends on Shoot type (short spur is early, long terminal shoot is late) and Vigour level (strong is late, weak is early). These factors interact: low vigour results in fewer and more short shoots, whereas high vigour leads to more and longer shoots.
Factors influencing seasonal growth include: 

\begin{enumerate} 
    \item Shoot type and vigour. 
    \item Correlative inhibition within the plant, defined as 'communication among buds'. 
    \item Apical dominance, which influences adventive buds and inhibition. This involves Polar gravitropic transport and the activation of adventive buds upon Removal of the terminal bud. The underlying Auxin - Cytokinin balance varies among species (peach < apple < sweet cherry) and strongly influences the Branching pattern. 
    \item Competition for assimilates, specifically the relationship between fruits $\leftrightarrow$ vegetative growth. 
    \item Number of growing meristems in top/root ratio. 
    \item Tree age, where Older trees produce many shoots but exhibit weak growth, and Young trees produce few and strong shoots. 
    \item Management and climate factors, such as Pruning, water and nutrient availability. 
\end{enumerate} 

Development after terminal bud formation Happens suddenly, and differentiation continues inside the bud(s). A compact shoot is formed, and development continues until bud break next spring, though the intensity of this differentiation varies (dormancy).

\section{Lecture 04 - 09/09-2025}
