\setcounter{chapter}{0}
\setcounter{section}{0}
\chapter{Lecture Notes}
\setlength{\headheight}{12.71342pt}
\addtolength{\topmargin}{-0.71342pt}

\section{Introduction}
The following lecture notes have been compiled by combining personal notes taken during class, personal highlights from the lecture slides, and targeted use of generative AI. The AI was provided with my notes and the highlighted slides, together with carefully designed prompts, to ensure that it focused on the most relevant and interesting aspects of each lecture. This approach aims to produce a coherent and focused summary that reflects both the course content and my individual learning perspective.


\section{Lecture 01: Introduction - 01/09-2025}
\textbf{Fruit and Berry Crop Physiology and Quality or Fruit and Berry Crop Physiology, Quality and Use}

\textbf{Professor: Torben Toldam-Andersen, KU-PLEN}

\vspace{1em}
The course aims to provide broad knowledge on the physiology and growing of fruit and berry crops (including pit-, stone-, cane-, and bush fruits, other berries, nuts, and fruit vegetables), focusing at the whole plant and organ level. Key objectives are the development of specialized knowledge on a course topic, the ability to analyze a crop to identify important aspects in growing and quality, and understanding how fruit quality is influenced, specifically balancing productivity vs. quality. The curriculum seeks to build links between the production and quality of raw materials and the food science aspects of their use, including sensory aspects. The underlying idea is to address both the Plant science and the Food Science aspects of fruits and berries, emphasizing the importance of these crops for human health and addressing lacking knowledge regarding the potentials in different genotypes (raw materials).

\vspace{0.5em}
The course focuses primarily on temperate fruits, nuts, and berries grown mainly in open field or tunnel systems. While fruit vegetables such as tomato may be included in the individual report, the emphasis includes common commercial systems, organic growing, and relevance for small-scale/home gardening. These crops are characterized as Diverse (many species and cultivars), high value crops, demanding intensive hand labor and resources, and often requiring manipulation at the single plant level. Although often fresh consumed/used, there is a large diversity in processed uses, such as juice, jam, fruit wine, vinegar, cakes, desserts, and ice cream. They are important to human health and provide pleasure through nice aroma and taste.

\vspace{0.5em}
The course content is structured around eight main topic areas: 

\begin{enumerate}
    \item Genetic basis and Cultivar variation
    \item Yield and quality components and determinants, including Organ development and interactions
    \item Allocation of dry matter and nutrients, focusing on 'Sources' and 'sinks' in fruiting plants
    \item Effects of pre-harvest factors on fruit quality, such as the importance of control of plant vigour and shape
    \item Content and development of secondary compounds in fruits
    \item Maturation, ripening, and assessment of optimal harvest time and quality aspects
    \item Post-harvest usability (fresh consumption, cooking, industry processing, especially juice)
    \item Sensory aspects of quality evaluation
\end{enumerate}

\vspace{0.5em}
Pedagogic methods utilize a mix of Lectures, Exercises ('Hands on'), Written assignments, Student lectures, and Excursions. Specialization is achieved through the individual report and the student lecture, which provide case specific supplements to the general lectures. The individual report is an opportunity to develop specific competence on a free topic relevant to the course (e.g., Bud dormancy, Aroma development, Canopy management). The report structure mandates an Intro (1-2 pages), Detailed presentation and discussion based on international literature (4-6 pages), and Summary/conclusion (½ - 1 page), not exceeding 8-9 pages in total (max 10 pages including figures). Students must use their own writing and avoid 'copy-paste' and plagiarism.

\vspace{0.5em}
The assessment is an Oral exam lasting 25 minutes in total. The exam includes a short talk about the individual report, followed by two curriculum questions: one focusing on crop physiology and one on quality aspects. The final grade is based on 50\% the individual report and 50\% the curriculum questions.

\section{Lecture 02: Yield and quality components - 01/09-2025}
\textbf{Professor: Torben Toldam-Andersen, KU-PLEN}

\vspace{1em}
The overarching aims in fruit growing are to maximise fruit yield and maximise fruit quality. The tools utilized to achieve these aims include Genetic choice (cultivars) and Growing technique. Fruit science focuses heavily on the physiological background informing the growing technique. Maximized Yield is typically expressed as tons/Ha (or hekto L/ha of juice or wine) and is fundamentally a product of the components Fruit number X Fruit size. Determinants influence the size or the level of development of single components within the genetic potential. These physiological elements are connected to growth and development and can be influenced by growing techniques, climate, and growing conditions.

\vspace{0.5em}
Maximized fruit quality is less clearly defined than yield and may vary with fruit species and final use. Important quality parameters include Colour and other aspects of appearance (especially for Fresh consumption), Taste components (sugar, acid, aroma), and 'Health components' (vitamins, phenols ......) which are a focus of research. Fruit size serves as both a yield and a quality component. The presence of unwanted substances such as pesticides is also an important aspect of fruit quality. The optimal compromise is crucial because physiological and technical factors may have opposite effects on yield and quality components. The environmental impact of the growing technique used, such as carbon footprint and pollution, also requires knowledge.

\vspace{0.5em}
The physiological elements that serve as yield and quality components and determinants include: 

\begin{enumerate} 
    \item Planting system and Growing system (fx Rubus, strawberry), determining Number of plants/ha. 
    \item Plant size and structure, including Elongation growth, shoot type development, Bud development, and Flower bud initiation. 
    \item Bud number/plant, bud type, and the ratio of leaf bud/flower bud. 
    \item Flower development, specifically Number of flowers/cluster and Flower quality (e.g., Number of seed primordia, Position in cluster). 
    \item Pollination, fertilization, and initial set, defining Initial fruit number/flower. 
    \item Fruit drop (June drop), which determines the Final fruit number/initial number of fruits. 
    \item Fruit growth and fruit development, the Leaf/fruit ratio, and specific factors like Number of seeds/fruit (Pollination) and Amount of flesh/seed (in Strawberries), leading to Fruit size and quality. 
    \item Yield/ha (Fruit number x fruit size) and Fruit quality (content). 
\end{enumerate}

\vspace{0.5em}
Specific crop examples illustrate the variability: Sour cherry yields may vary drastically, from 3 to 18 tons/ha, due to factors like Bud death (sometimes reaching 90\%) and Fruit set percentage. Unstable yields in sour cherry 'Stevnsbær' pose a significant problem for the Market (Industry). In strawberries, vegetative growth (runner production) is influenced by long days and high temperature, while flower cluster formation is promoted by short day and low temperature. Pollination is very important in strawberries as the final fruit size depends on the Achenes/berry. For grapes, yield components include Number berries/cluster x weight/berry and Number of clusters x weight/cluster, influenced by factors like pruning, thinning, and the growing system. Analysis of 51 grape cultivars harvested in 2018 showed a high correlation ($R^2=0.9516$) between Yield kg/plant and total sugar (Fruktose + Glukose g/plante).


\section{Lecture 03: Bud and shoot development - 01/09-2025}
\textbf{ Professor: Torben Toldam-Andersen, KU-PLEN}

\vspace{1em}
This lecture explores Bud and shoot development, covering Meristems, Bud types and shoots, and seasonal Growth patterns. Embryogenesis initiates plant development by establishing the primary meristems and determines the Apical $\leftrightarrow$ basal axial development, including the Shoot apex, Hypocotyl (The stem), Root, Root apical meristem, and Root cap. It also establishes Radial patterning, such as Epidermal cells, Cortical tissue, and the Vascular cylinder. The initial stages of development include Cotyledons (The first leaves). Development transitions from meristem to shoot, detailing the structure of the shoot apex in 3D and the Transition from vegetative to floral meristem. Variation exists among species in how strong the xylem or wood develop, with Trees having strong development and Bushes and canes exhibiting weak development. The purposes of leaves include Production of assimilates (the source of carbon), Water transpiration resulting in uptake and transport of water and nutrients, Production of hormones, and Control of water status/cooling.

\vspace{0.5em}
Buds are defined as locations of growth with a potential for development. The Types (fates) of buds include Vegetative, Generative, Sleeping (not dead), and Dead. Buds determine the plant's dimensions by forming shoots of different lengths, such as Short shoots (spurs), emphasizing the Importance of the shoot's position. In strawberries, the vegetative bud results in a side crown or a 'runner'. Pome fruits, exemplified by apple and pear, possess mixed buds, while Stonefruits utilize 'naked' buds. The topography of flowerbuds differs between pome- and stonefruit, which raises questions about which structure bears flowers in the terminal position on short shoots and the Importance of this difference.

\vspace{0.5em}
Seasonal growth patterns involve a Flush of growth in spring and early summer, followed by Growth termination or indefinite growth, depending on the species. Terminal bud formation also occurs, alongside Adventive buds formed along with shoot development. The timing of terminal bud formation depends on Shoot type (short spur is early, long terminal shoot is late) and Vigour level (strong is late, weak is early). These factors interact: low vigour results in fewer and more short shoots, whereas high vigour leads to more and longer shoots.

\vspace{0.5em}
Factors influencing seasonal growth include: 

\begin{enumerate} 
    \item Shoot type and vigour. 
    \item Correlative inhibition within the plant, defined as 'communication among buds'. 
    \item Apical dominance, which influences adventive buds and inhibition. This involves Polar gravitropic transport and the activation of adventive buds upon Removal of the terminal bud. The underlying Auxin - Cytokinin balance varies among species (peach < apple < sweet cherry) and strongly influences the Branching pattern. 
    \item Competition for assimilates, specifically the relationship between fruits $\leftrightarrow$ vegetative growth. 
    \item Number of growing meristems in top/root ratio. 
    \item Tree age, where Older trees produce many shoots but exhibit weak growth, and Young trees produce few and strong shoots. 
    \item Management and climate factors, such as Pruning, water and nutrient availability. 
\end{enumerate} 

\vspace{0.5em}
Development after terminal bud formation Happens suddenly, and differentiation continues inside the bud(s). A compact shoot is formed, and development continues until bud break next spring, though the intensity of this differentiation varies (dormancy).

\section{Lecture 04: Flowers and flowerbud development - 05/09-2025}
\textbf{Torben Toldam-Andersen, KU-PLEN}

\vspace{1em}
The lecture addresses Flower and flowerbud development, encompassing Phases and theories, Structure - node development, Factors, and specific examples including Apples, Cherry, Currant, Raspberry, and Strawberry. 

\vspace{0.5em}
Flowerbud formation occurs across three distinct phases: 

\begin{enumerate} 
    \item Induction (May - June) 
    \item Initiation (July - September) 
    \item Differentiation (July - May) 
\end{enumerate} 

\vspace{0.5em}
Historical theories regarding flowerbud formation include the 1822 concept that the flower bud is formed through April - June (induction) and then 'filled out' through July - October (initiation - differentiation). Later, the 1920's highlighted the C/N-ratio as decisive, while the 1970's focused on the Cytokinin/Gibberellin-ratio and Source-Sink relationships, stressing availability and activity. Node development speed influences the bud's fate: slow development results in remaining Vegetative, while relatively fast development leads to Floral buds. The time required for node formation is termed the plastochron.

\vspace{0.5em}
The structure of the flower bud is akin to a compressed shoot. In apple, a flower bud contains 3 bracts, 6 real leaves, 3 transitional leaves, and 9 budscales, totaling 21 nodes, with a minimum critical node number approximately 18-20. The flower organs are formed sequentially from the outside inward: sepals are formed first, followed by petals, which are formed second. The 'King flower' is formed first. Pomefruit (apple, pear) flower buds are “mixed” buds, often terminal, and are considered “strong” buds (high critical node number). Apple buds are less developed in the dormant stage, with a larger proportion of differentiation occurring in the spring. Conversely, Stonefruit (cherry, plum) flower buds are “naked” buds, always leaf buds in the terminal position, and are “weak” buds (low node number). Cherries reach an advanced stage of development before entering dormancy, which may render them more sensitive to winter damage. Sour cherry (Prunus cerasus) leaf buds contain 7 nodes, while flower buds contain 13 nodes.

\vspace{0.5em}
Flower bud formation in tree fruit is affected by growing techniques related to Vigour or Source-sink dynamics: 

\begin{enumerate} 
    \item Variety 
    \item Tree age 
    \item Rootstock 
    \item Dormant pruning 
    \item Root pruning 
    \item Light condition 
    \item Fruit load 
    \item Thinning 
    \item Nutrition 
    \item Irrigation 
\end{enumerate} 

\vspace{0.5em}
For shoot growth, a minimum length (e.g., >15 or preferably 20 nodes per shoot) is necessary before initiation. Initiation usually spreads upwards the shoot, except for the terminal bud, and lower buds do not form flowers. Very strong (long) shoot growth, potentially induced by Gibberellins, may counteract formation. Short shoots (spurs) easily form flowers and show greater bud density, a higher percentage of live buds, a higher percentage of flowerbuds, and more flowers/bud than long shoots. Climate factors play a role, as flower initiation normally happens if day length is <12-14 hours, and decreasing temperature has a positive effect.

\vspace{0.5em}
In raspberries (Rubus), initiation requires reaching a minimum physiological stage, and is promoted by short day and low temperatures, typically starting from the top of the cane earliest in shoots where growth has terminated. Normally 2/3 of buds become flowers, but low positions on the cane are usually without flowers. For yield, the Number of laterals/cane and flowers/lateral are the most important components. Strawberries exhibit types ranging from obligate short day (June bearers), where day length is critical (<11-14 hours), to day neutral types that initiate at both short and long days. Vegetative growth (runners) is promoted by long day and high temperature. Warm periods in late autumn have positive effects on flower development, leading to more flowers/bud. This differentiation affects flower quality; for instance, strawberry 1. order flowers have 400 pistels, while 4. order flowers have only 80 pistels. In currants, the terminal flowers in the string are often poorly developed, resulting in few seeds, small berries, and a high risk of drop.

\section{Lecture 05: Bud dormancy - 05/09-2025}
\textbf{Torben Toldam-Andersen, KU-PLEN}

\vspace{1em}
This lecture details Bud dormancy, covering its development, breaking, and specific studies on Sour Cherry 'Stevnsbær' buds in winter. Key concepts include Growth phases during the year, Dormancy terminology, and Metabolic activities during dormancy. The lecture introduces the concept of chilling units (CU) and Growing Degree Hours (GDH). Chilling requirement estimates for various fruit and nut species are noted. Dormancy models, including the calculation of Chill Units (Saure, 1985), and the Growing Degree Hour model are essential tools. A GDH curve was specifically developed for 'Montmorency' cherry (Anderson and Richardson 1986). Studies examining 'Redhaven' peach flowers and lateral/terminal buds explored the effect of temperature and length of chilling period on percent budbreak and GDH demand for budbreak (Scalabrelli and Couvillon 1986). Chilling periods of 600, 1340, and 2040 Chilling Hours were analyzed.

\vspace{0.5em}
The yield components of sour cherry are multifaceted and include: 

\begin{enumerate} 
    \item Tree size 
    \item Shoot number and type 
    \item Bud type 
    \item Number of flower buds 
    \item Flowers/bud 
    \item Bud death 
    \item Fruit set\% 
    \item Number of fresh flowers in spring 
    \item Initial total number of fruits 
    \item Fruit drop 
    \item Final number of fruits 
    \item YIELD 
\end{enumerate} 

\vspace{0.5em}
Floral bud mortality is a critical factor. Dormancy levels in sour cherry buds can be determined by forcing shoots. Studies investigated the effects of irrigation and nutrients. Forcing of buds showed an increase in water content of forced buds over the period of October 4 to November 29 when under +drip conditions (both tip and base) compared to -drip. Shoot studies indicated a relationship between High water content and high bud death, and Low water content and low bud death. Observed damage included a missing pistil or a necrotic pistil.

\vspace{0.5em}
Research examined the Effects of urea (0.75\% and 4\% urea) on sour cherry. In May 1996, the 4\% urea treatment significantly reduced Living buds (79.3\% compared to 91.3\% in Control, LSD 4.0) and decreased Flowers/tree, Fruit set\% (7.9\% compared to 12.8\%), and Berries/tree. Shoot length and Buds / meter shoot were not significantly affected. Furthermore, in November 1998, buds on annual shoots treated with 4\% Urea showed reduced ABA content (1675 ng/drymatter versus 3948 in Control) and increased Water content (118.0\% of drymatter versus 108.1\% in Control), although Bud size was not significantly different. Freeze tests were conducted to determine the hardiness level in 'Stevnsbær' at -6\textdegree C (laboratory) and -7\textdegree C (field). Bud death was analyzed in relation to rootstock (Weiroot, Colt, P.avium Grundstamme) and frost exposure during winter. Counting flowers is noted as a considerable task.


\section{Lecture 06: Flowering, pollination and fruit set - 08/09-2025}
\textbf{Torben Toldam-Andersen, KU-PLEN}

\vspace{1em}
The physiological processes of flowering, pollination, and fruit set are critical determinants of Yield, alongside factors such as Flower number and quality, Effective Pollination Period (EPP), Pollen quantity and quality, Pollen transfer, Initial fruit set, and June drop, leading to the Final fruit set. Flower quality is defined as "The capacity to develop into a fruit if pollinated with the right pollen at the right time". Indicators of flower quality include Ovule longevity, Flower weight, and Number of flowers per cluster (position in cluster). Stone fruit quality depends on carbohydrate reserve availability, while pome fruit quality depends on burse leaf area.

\vspace{0.5em}
Fertility varies, categorizing species as Self fertile (Self compatible), such as apricots, peaches, strawberries, and grapes, or Self sterile (Self-incompatible), such as hazel, pear, and apple. Sterility is not absolute; in normal self-sterile conditions, own pollen mostly fails, but cross-pollination often results in faster pollen tube growth. Parthenocarpy is fruit development without fertilisation, without seeds. Some plum cultivars, such as Opal and Victoria, are self-fertile. Self-incompatibility is controlled by S alleles at the S-locus. Self-fertile sweet cherry cultivars (e.g., Stella, Lapins) are universal pollinators, often possessing a mutated S4 allele. Parthenocarpy or Apomiksis (fruit or seed development without fertilisation) may be mistaken for self-compatibility.

\vspace{0.5em}
Important flowering related terms include: 

\begin{enumerate} 
    \item \textbf{Monoecious:} Having pistillate (female) and staminate (male) flowers on the same plant. 
    \item \textbf{Dichogamy:} Opening of male and female flowers at different times on a monoecious plant, which ensures cross-pollination. 
    \item \textbf{Protandrous:} Male flowers open before female flowers. 
    \item \textbf{Protogynous:} Female flowers open before male flowers. 
\end{enumerate} 

\vspace{0.5em}
Pollen tube growth requires its own reserves at the stigma surface, followed by interaction with style tissue (proteins, carbohydrates). The Effective Pollination Period (EPP) defines the time when fertilization and fruit development occur. EPP varies with flower quality and is reduced by factors such as low nutrient status and high cropping level.

\vspace{0.5em}
The required quality of pollination depends on the species: 

\begin{enumerate} 
    \item For \textbf{large fruited species} with intensive flowering, pollination does not need to be optimal, but should be optimal if flowering is weak ($<40-50\%\ $ of max). Heavy crop load leads to small fruits. 
    \item For \textbf{small fruited species}, yield is mainly determined by fruit number, and pollination should always be optimal. 
    \item For \textbf{species with several pistels - embryos} (where fruit size is seed-number related), pollination should be optimal. 
\end{enumerate} 

\vspace{0.5em}
Pollinator demands include Quantity (e.g., Malus types like M. Floribunda), Quality (compatibility, ability to germinate at low temperatures), and Timing (overlapping flowering periods). Triploid cultivars (e.g., Gråsten apple) do not produce fertile pollen.

\vspace{0.5em}
Pollen transfer occurs via Wind (for self-fertile species like sour cherry), Insects (for cross-pollinators), or artificially. Critical distance for transfer is 10-15 m. Honeybees (47\%) and Soil living bees (25\%) are common pollinators. Critical flying limits are 13-14\textdegree C for Bees and 8-12\textdegree C for Bumblebees. Bumblebees visit 2-3 x as many flowers as bees. Experimental trials on 'Stevnsbær' sour cherry showed that the addition of 'artificial wind' (6 km/t) increased fruit set (\%) and yield (kg/tree) compared to controls, and that early pollination (Day 1) resulted in higher fruit set than later pollination (Day 2). Pollinator branches and hand pollination also improved fruit set, demonstrating that pollination is often not good enough.


\section{Lecture 07: Fruit Thinning - 08/09-2025}
\textbf{Professor: Torben Toldam-Andersen, KU-PLEN}

\vspace{1em}
Fruit Thinning is a necessary practice carried out in Large fruited species, specifically Apple, Pear, Plum, Peach, and Grapes (where grape clusters may be seen as similar to a big fruit). The primary objectives of thinning are to maximise fruit size distribution, prevent the breaking of branches, improve flower development, and stabilize cropping (prevent alternance). Thinning also aims to improve quality, resulting in a higher sugar\%, Dry matter+acid, and potentially improved firmness, although larger fruits are naturally softer. A negative aspect is that Increased size may lead to an increase in 'Bitterpit'.

\vspace{0.5em}
The Timing of thinning is crucial. Early thinning is important as it affects both return bloom and fruit size. Late thinning has no effect on return bloom, and the effect on size decreases with delay in time. Very late thinning (starting at ripening) primarily affects quality by increasing sugar accumulation during ripening. Thinning timing effects on size were observed in apple, and in 'Clara Frijs' pear on three different rootstocks.

\vspace{0.5em}
Thinning can be performed using various methods: 

\begin{enumerate} 
    \item Hand thinning. 
    \item Hormones: BA (benzyladenin/Cytokinine), Ethephon, NAA (1-napthylacetic acid, 'Pomoxon'). (Note: NAA is Not registered in DK). 
    \item Fertilizers: ATS (ammoniumthiosulfate), lime sulphor. 
    \item Organic compounds: Oils, soaps, salts. 
    \item Other: shade, mechanical. 
\end{enumerate} 

\vspace{0.5em}
Fertilizers such as ATS (ammoniumthiosulfate) function by burning the stigma and are applied before pollination, often requiring several applications to be effective. ATS may burn leaves, but is low cost, effective, and environmental friendly. To reduce leaf damage, fast drying must be ensured by using a low amount of water and spraying on dry leaves, and the weather must stay dry after spraying. Urea is similar to ATS but less effective, and Lime Sulphur is used in plums but is less effective in apple. The general mode of action for many organic compounds tested is the burning of the stigma, often associated with leaf damage when effective.

\vspace{0.5em}
Mechanical thinning, developed in Germany, requires optimized tree training, but risks removing the highest quality buds. Constraints include thinning delicate and fragile fruitlets at an early stage without bruising the remaining fruit or removing too many leaves and spurs. Strong and sturdy training structures are necessary, especially when using M9-style rootstocks whose trees are firmly attached to the top wire. Shade experimentally induces thinning through increased internal competition and can be applied for a short period. Leaf damage caused by fertilizers is avoided in shade methods because the resulting permanent reduction of carbon supply is not wanted.

\vspace{0.5em}
Factors affecting the efficiency of thinners include Temperature and humidity (increased effect with high temp and high humidity), Volume of spray, Additives (surfactants, which can cause leaf damage), and the benefit of Repeated treatments. Cultivar differences exist, as Strong flowers are difficult to thin. Thinning is enhanced if trees have a weak root system, drought stress, Nitrogen deficient status, few flowers in the cluster, large yield the year before (indicating weak flowers), young trees in strong growth (high vigor), or bad pollination (few bees). Thinning is decreased if trees are in good balance (growth and nutrition), fruits sit on horizontal or weeping branches, many flowers are in clusters with well developed leaves (strong flowers), trees are open, or pollination is optimal.

\vspace{0.5em}
Thinning serves as a growth control tool. If growth is weak, thinning should be early, especially on weak shoots. If growth is strong, thinning should be late, keeping most fruit on strong shoots. The top of the canopy can be used to judge vigour, particularly in species like Solaris which has a strong canopy. Management techniques such as Topping send carbon down to clusters, enhancing growth in shoots which are not topped and stimulating lateral development. Removing inner leaves ensures good ventilation and facilitates later 'green harvest' or thinning. Furthermore, dense clusters may be 'tipped' before they close up, and big clusters (Fx Bolero) may be reduced in size. A high crop load can be sustained on a strong root stock.


\section{Lecture 08+09: Source-sink and carbon allocation - 08/09-2025}
\textbf{Professor: Torben Toldam-Andersen, KU-PLEN}

\vspace{1em}
This lecture addresses Source-sink and carbon allocation, focusing on definitions, effects, internal competition, seasonal changes in allocation, and the relationship between source-sink dynamics and fruit size. The basis of growth is founded on the fact that fruits generally contain about 80-90\% water, and of the dry matter, at least 90\% originates from assimilates formed in the leaves. Assimilates must be translocated from leaves to fruit. The end product exported is mostly sucrose, but in Pome and Stone fruit, sorbitol is the end product.

\vspace{0.5em}
The source strength is defined as the ability of assimilating tissue to produce and export assimilates, quantified by Source size (leaf area/leaf number) multiplied by source activity. Sink strength is the ability of importing organs to attract and metabolize assimilates, quantified by Sink size (fruit number, number of growing shoot tips) multiplied by sink activity.

\vspace{0.5em}
Cropping significantly impacts plant physiology, as demonstrated by apple studies. High cropping stimulates Photosynthetic intensity (140-210 relative to 100 without/few fruits), Translocation intensity (160-175), and Stomata opening degree (165). However, high cropping shifts Dry matter production allocation: while total dry matter production increases slightly (112 vs 100), the dry matter allocated to roots drops sharply (<1 vs 23), and 81 relative units are allocated to fruits. The sugar + sorbitol concentration decreases during high cropping (80 relative units). Increased $CO_2$ level leads to increased Sorbitol\% (13.4 vs 12.2), Sucrose\% (3.6 vs 2.7), and Transported 14\textdegree C (\% of uptake: 59.6 vs 53.6).

\vspace{0.5em}
Internal competition involves 'Supply - Demand'. Apple has a strong sink strength but a relatively weak source, whereas Black currant has a weak sink strength but a strong source. Internal competition, characterized by the Fruit/leaf ratio, is strongly important in large fruited species such as apple, pear, plum, and large sweet cherry, necessitating thinning. However, the effect is weakly important in small fruited species like currants, blueberry, raspberry, and sour cherry. Sour cherry 'Stevnsbær' showed no difference in total dry matter (\%), sugar (\%), acid, or colour when comparing low (7 kg/tree) versus normal (25 kg/tree) cropping levels.

\vspace{0.5em}
Source and sink strength of organs vary dynamically with development and time. Source strength increases with shoot growth, light, and temperature. Types of competition include internal competition between shoots and fruits, and between fruits and flowers.

\vspace{0.5em}
There are two primary ways to increase fruit size: 

\begin{enumerate} 
    \item By changing the fruit/Leaf ratio, thereby increasing Assimilate availability, which results in a Positive correlation between size and dry matter concentration (internal quality). 
    \item Through an increase in sink strength (or sink 'vigour'), such as by K fertilisation, which results in a Negative correlation between size and dry matter (internal Quality). 
\end{enumerate} 

\vspace{0.5em}
At the tree level, no correlation exists between size and quality.


\section{Lecture 10: Table Grapes - growing techniques and cultivars - 12/09-2025}
\textbf{Professor: Torben Toldam-Andersen, KU-PLEN}

\vspace{1em}
This lecture focuses on aspects connected to the tunnel growing of table grapes and berries, specifically within an unheated tunnelsystem. The slides cover inherent benefits and challenges of this system. The experimental tunnel was established in 2007, and is relatively small (5m wide, 2,75m tall). It uses an internal trellis system allowing training in 3 levels (low, medium, and high stem plants) for approximately 120 plants. The training system is simple and similar for all: 2 Codon on low, medium or high trunk height, typically using short 2 bud stabs later developed into double stabs with (3)-4 shoots. Ventilation is implemented in both sides and utilizes a fan.

\vspace{0.5em}
Management priorities are defined across seasonal stages: 

\begin{enumerate} 
    \item \textbf{Budbreak and the early growth:} Focus on the importance of humidity, Shoot thinning (when and how), selecting the best shoot on a spur to carry the fruit, and selecting the best shoot for next year. 
    \item \textbf{Flowering and fruit set:} Identifying Important factors and management, and Cluster thinning?. 
    \item \textbf{The summer period:} Canopy management (How are the shoots treated?), and managing Diseases (which are most important? prevention or reduction methods?). 
    \item \textbf{Late summer and autumn:} Fruit development (including cluster thinning and shaping), determining what to do, when, and why, and establishing When to harvest?. 
\end{enumerate}

\vspace{0.5em}
The descriptor system used for testing table grape cultivars in Denmark (VitiNord Nov 2015) included 38 cultivars. The descriptor employs a scale from 1-9 for scoring parameters, where 9 is generally the best, most wanted, or most attractive. However, the scale may be modified so 1 may not always be the 'less wanted', or 5 may be the best from a growing perspective. The system uses 25 Culture technical parameters (e.g., C1 Vigor, C3 Formation of laterals, C6 Fruit-fullness from the second eye, C17 Resistance to peronospera), 36 Technological parameters (e.g., T5 Cluster weight, T9 Berry shape, T23 Sugar content when ripe, T30-32 Aroma), and 25 Morphological parameters.

\vspace{0.5em}
Specific parameters reflect desired growing characteristics; for example, T7 (Berry weight at BBCH 89) scores 9 for Very large (>7 g), while C3 (Tendency to formation of laterals) scores 9 for Very weak occurrence (1 lateral/main shoot) because low occurrence saves Work with canopy management. C10 (Density degree of the cluster) rates both very dense (1) and very loose (1,1 or 3,3) low, while Appropriate (7) and Very appropriate (9) densities are rated high. The Relative productivity index (C25) is calculated as (cluster number/no of main shoots) x weight of 1 cluster in gram. Cultivars are ranked based on an average equivalent score derived from summing and normalizing the Culture technical and Technological parameter scores.

\vspace{0.5em}
Highly ranked cultivars include Augustostuzi Muskotaly, Conegliano 218, New York Muscat, and Palatina. These cultivars generally achieve a high sugar content (score 7) of about 19\% brix, and acid levels (score 5-7) between 6 to 6,8 g/L. The cultivar Augusta exhibits medium vigor, Above medium fruit fullness from eye no 2, and Good resistance against Oidium, Perenospera, and Botrytis. Augusta clusters are large (780 g), and the berries are very large, exhibiting a very fine crunchy structure and very high eating quality. Augusta begins harvest in week 37 in the plastic tunnel and provides a qualitative and quantitative improvement in relation to its parents, 'Italia' and 'Königin der Weingarten'.


\section{Lecture 11+12: Fruit development of large fruited spe- cies - 15/09-2025}
\textbf{Professor: Torben Toldam-Andersen, KU-PLEN}

\vspace{1em}
The development of large-fruited species involves examining fruit structure, growth curves, volume changes in apple, gravity, intercellular air space, and changes in fruit shape during development. Fruit shape, as noted by Westwood (1978), is influenced by climate, with cool climates often resulting in more pronounced shapes (greater length/diameter ratio) compared to warm climates. Pear shape (e.g., 'Conference') is known to be affected by seeds, as demonstrated by parthenocarp fruit. Fruit size is determined very early, influenced by cell size and thinning practices. The relationship between fruit size and yield dictates that a significant leaf area is required per fruit; for example, 30 leaves at 20 $cm^2$ per leaf equals 600 $cm^2$ or approximately an A4 sheet/fruit.

\vspace{0.5em}
Quality effects observed in plums (Vangdal 1982) show strong interactions between fruit load, size, and sugar content. For instance, reducing the number of fruits per spur significantly increases sugar content; fruits/spur at 0.14 resulted in 11.1\% total sugar and 7.6\% sucrose, compared to 0.98 fruits/spur yielding 5.6\% total sugar and 2.4\% sucrose. Similarly, larger fruit diameter (>40 mm) correlates with higher sugars (13.1\% total sugar) than smaller fruit (30-35 mm, 11.1\% total sugar). Increased skin color (80\%) also shows correlation with increased sugar levels (13.2\% total sugar).

\vspace{0.5em}
Yield relations in fruit growing can be modeled as a Michelis-Menten kinetic system. If Y is yield t/ha and x is the number of fruits/ha, then $Y=(k \cdot x)/(K+x)$. Fruit size is defined as $Y/x=k/(K+x)$. The physiological parameters k and K are crucial: k represents source activity, determined mainly by total assimilate production (potential max. yield), and 1/K represents sink activity. The term k/K is the potential maximal fruit size. Comparison across different tree statuses shows that older, well-established trees typically exhibit lower sink activity (1/K lower) but possess a larger source capacity (k) than young trees. Dense trees demonstrate lower source activity.

\vspace{0.5em}
Internal quality, maturity, and harvest time are studied alongside external qualities like size, focusing on internal chemical changes during development. Apple C-14 labelling studies conducted end June and end July tracked chemical changes. In apple fruit development, both sorbitol (Sor) and sucrose (Suc) are unloaded into the cell wall space. Sor is taken up via sorbitol transporter (SOT). Suc is transported either directly by sucrose transporter (SUT) or converted to fructose (Fru) and glucose (Glc) by cell wall invertase (CWINV), followed by transport via hexose transporter (HT). In the cytosol, Sor converts to Fru via sorbitol dehydrogenase (SDH). Suc converts to Fru and Glc via neutral invertase (NINV) or to Fru and UDP-glucose via sucrose synthase (SUSY). Glc and Fru are phosphorylated by hexokinase (HK) and fructokinase (FK). F6P enters glycolysis/TCA cycle, and G1P is utilized for starch synthesis. Sucrose can be re-synthesized using UDPG and F6P via sucrose phosphate synthase (SPS) and sucrose-phosphatase (SPP). Finally, most non-metabolized Fru, Glc, and Suc are stored in the vacuole using special tonoplast transporters, where Suc can also be broken down by vacuolar acid invertase (vAINV).


\section{Lecture 13+14: Fruit development of small fruited spe- cies / berries - 15/09-2025} 
\textbf{Torben Toldam-Andersen, KU-PLEN}

\vspace{1em}
This lecture details the development of small fruited species, focusing on fruit types, growth phases and curves, size factors, chemical changes, and mechanisms for sugar and acid accumulation. Aggregate fruits are derived from many ovaries. Growth curves show that berries and stone fruits have a distinct Stage II characterized by slow size enlargement. A berry, such as a grape, gooseberry, or a currant, is a multiseeded fruit derived from a single ovary. The cell division period, specifically the first weeks, is critical in fruit development for both small and large fruited species. Fruits primarily consist of water (85-90\%) and 10-15\% dry matter, with about 90\% of the dry matter originating from assimilates produced in the leaves.

\vspace{0.5em}
Internal competition can be very strong during the cell division phase due to vigorous growth in all organs. Fruit drop in black currant plants was observed in shaded plants, but effects were minimal or absent in sun-exposed plants. For strawberries, size is a linear function of achene number/berry at a given achene density. Berry weight is calculated as (total achene number/berry-C) x F, where C is a correction factor depending on the cultivar, and F is the flesh density in $g/cm^2$. While maximal swelling yields 6 $achenes/cm^2$, it is often lower, resulting in 8-12 $achenes/cm^2$, with water being probably the most important factor. In black currant, berry size is positively correlated to shoot growth intensity. Conversely, berry size is negatively correlated with the concentration of total and soluble solids, although acid content increases. This suggests the main effect on black currant composition is via sink activity. Cultivar variation is often a larger factor influencing development than growing techniques or climate. Fruit refixation of $CO_2$ occurs.

\vspace{0.5em}
The relative sweetness of common compounds (scale where Sucrose =100) is Fructose: 173; Glucose: 74; and Sorbitol: 50. Relative sourness (scale 1-9; 9=most sour) is highest for Malic acid (8.2), followed by Tartaric acid (4.3) and Citric acid (3.6). Chemical analyses of black currant cultivars 'Tenah' and 'Ben Nevis' reveal changes in sugars (Sucrose, Glucose, Fructose) and acids (Citric and Malic acid) over time. A study noted that 1992 was a very warm summer. There is a relationship between water content and citrate content in black currant. Light affects fruit quality; the Sucrose/citric acid ratio in black currant was 0.28 in the Sun top compared to 0.15 in the Shade bottom.

\vspace{0.5em}
Regarding vascular mechanisms, in the early phase (pre-veraison), the xylem is fully functional, transporting water and nutrients into the fruits. Post-veraison (developmental phase 3), the xylem loses its conductivity, leaving the fruit connected to the plant via phloem transport, which delivers sugar; the central connection to the seeds is maintained. Studies showed that when gas exchange via the berry surface is reduced by a coating with Vaseline, the sugar accumulation is impaired. Furthermore, Vaselin coated fruits combined with leaf removal resulted in fruit drop.

\vspace{0.5em}
Fruit quality varies across berry types: 

\begin{enumerate} 
    \item Black currant has the highest recorded ranges for Total dry matter (\%) (17-28\%) and C-vit mg/100 g (95-253). 
    \item Red currant exhibits the highest Acid g/100 g range (2.7-4.1). 
    \item Strawberry has the lowest Total dry matter (\%) (7-11\%) and Acid g/100 g range (0.8-1.2). 
\end{enumerate}


\section{Lecture 15+16: Aromas in fruits - 22/09-2025}
\textbf{Mikael Agerlin Petersen, FOOD-DCB}

\vspace{1em}
This lecture explores Aroma in fruits, specifically addressing advances in volatile research, differences in aroma patterns, and the influence of the fruit/leaf ratio. Flavour is defined as the combination of basic tastes (Sour, Salt, Sweet, Bitter, Umami), odour (orthonasal and retronasal), and trigeminal stimuli. Aroma compounds are organic compounds with a certain volatility that interact with our olfactory receptors (not taste). They are effective in very small amounts; for example, 10 ppb of 2-Methoxy-3-isopropylpyrazin corresponds to mixing one teaspoon into a small swimming pool, resulting in a significant odour.

\vspace{0.5em}
Aroma analyses are typically performed using GC-MS analyses. Gas Chromatographs (GC) cannot inject large amounts of water, lipids, sugars, or amino acids, but accept gas ($\approx 1 mL$) or organic solvents ($\approx 2 \mu L$). Headspace methods are frequently used, such as Dynamic Headspace Sampling (DHS), where headspace is continuously removed and collected/concentrated on a trap, ensuring equilibrium is never reached. The GC separates mixtures of volatiles, while the Mass Spectrometer (MS) utilizes an electron beam to create charged fragments of molecules, which are filtered by a Quadropole.

\vspace{0.5em}
The aroma of a food is normally complex, resulting from a combination of many aroma compounds. Small peaks in a chromatogram (which may show >75 peaks) can also be important. 

\vspace{0.5em}
Fruit volatile compounds are mainly comprised of: 

\begin{enumerate} 
    \item Esters 
    \item Alcohols 
    \item Aldehydes 
    \item Ketones 
    \item Lactones 
    \item Terpenoids 
    \item Some sulfur compounds 
\end{enumerate} 

\vspace{0.5em}
The specific content of these compounds determines, to a very high degree, the distinctive sensory quality of a fruit.

\vspace{0.5em}
Aroma compounds are produced during ripening. Climacteric fruits show a distinct production controlled by ethylene. Studies on transgenic 'Royal Gala' apple (AO3) unable to synthesize ethylene confirmed that exposure to exogenous ethylene induces ripening. The genes involved in aroma biosynthesis are not coordinately regulated by ethylene; typically, only the first and final steps are ethylene regulated. Alcohol Acyl Transferase (AAT), which catalyzes the last step of volatile ester biosynthesis, is regulated by ethylene gene expression. Varieties like 'Golden Delicious' (high ester production) and 'Granny Smith' (low ester production) represent phenotypic extremes in volatile production.

\vspace{0.5em}
The ethylene receptor blocker 1-MCP (1-Methylcyclopropene, "SmartFresh") binds to receptors without triggering a response. 1-MCP treatment immediately after harvest influenced the aroma profile in 'Ildrød Pigeon' apples after cold storage. 1-MCP is approved for minor use in closed storage rooms but is under re-evaluation.
Aromas are generally under genetic control, but production can be influenced by Climate, Soil, and Fertilisation. The effect of the fruit/leaf ratio on aroma quality has been studied using different levels of fruit thinning (no thinning, moderate 10 cm, and heavy 20 cm) in potted apple trees.

\vspace{0.5em}
Specific fruits contain characteristic compounds: Strawberries have one of the most complex aromas ($\approx$ 350 volatile compounds). Furaneol (2,5-dimethyl-3(2H)-furanones) is a key component, and Methyl anthranilate is responsible for the typical spicy-aromatic and flowery note of wood strawberry (Fr. vesca L.). Banana fruity top notes derive from volatile esters like isoamyl acetate and isobutyl acetate. Citrus characteristic volatiles include aliphatic aldehydes (e.g., Octanal) and monoterpenes (e.g., Limonene). Raspberry aroma is particularly influenced by $\alpha$-ionone, $\beta$-ionone, and raspberry ketone.


\section{Lecture 17+18: Organic growing of fruit and
berries 22/09-2025}
\textbf{Maren Korsgaard, Inst. of Plant and Environment, mkor@PLEN.ku.dk}

\vspace{1em}
This lecture provides guidelines and legislative background for the organic production of fruit and berries. The main vision for organic production is built upon four foundational principles: 

\begin{enumerate} 
    \item HEALTH: Sustain and enhance the health of soil, plant, animal, human, and planet as one and indivisible. 
    \item ECOLOGY: Based on living ecological systems and cycles, working with and emulating them. 
    \item FAIRNESS: Build on relationships that ensure fairness regarding the common environment and life opportunities. 
    \item CARE: Managed in a precautionary and responsible manner to protect the health and well-being of current and future generations and the environment. 
\end{enumerate} 

\vspace{0.5em}
Legislation governing organic production is based on EU 834/2007 (2021) and Guidelines of Organic agriculture in Denmark (Landbrugsstyrelsen). The EU Green Deal aims for 25\% of agricultural land to be organic by 2030, rising from 10,5\% in EU27 in 2022.

\vspace{0.5em}
In Denmark, the total organic cultivated area was 295.233 ha in 2024, representing 11,1\% of the Danish agricultural land, showing a small decline (2,7\%) related to 2023. The organic fruit and berry-acreage is increasing. The largest organic fruit areas in Denmark (2024, Ha) include Apple (616 Ha, 42,2\% of total DK fruit-area) and Hazelnuts (504 Ha, 83,4\%). Conventional Danish fruit production is often not profitable, and organic production is also risky business due to low yield per hectare. The organic apple-yield per hectare is approximately 5,5 ton/ha, but at least 15 ton/ha is necessary for a profitable organic apple-production.
Organic production utilizes specific approved inputs. Organic fertilizers are always organic Nitrogen-fertilizers, including Farmyard-manure, Green manure, Rock phosphate, and Potassium sulphate*. Agents allowed for pest and disease regulation include I: Azadirachtin (Neem), I: Plant oils, F: Copper-products, F: Sulphur, and F: Potassium bicarbonate (Armicarb).

\vspace{0.5em}
Studies suggest that organic agriculture is the most environmental friendly agricultural system. Although the climate impact per kg food is the same as conventional production, organic soil contains 10\% more carbon and sequesters more C (+256 kg C/ha/year), and emission of nitrous gasses is 24\% lower. Local production is emphasized, as transport and cooling represent the largest environmental impact of apples. Danish apples in season have a very low climate impact (0,1 kg $CO_2$ -eq./ kg food). Quality of organic products includes absence of pesticide-residues (apart from background pollution). Organic products also contain approximately 12\% more health-promoting substances than conventional products, showing slightly higher content of total antioxidants and dry matter (taste). Organic apples have 4-7\% more sugar.

\vspace{0.5em}
To reduce risk in organic apple growing, planning should focus on dry climate locations, robust varieties, and well-drained soil. Preventing fungus diseases involves choosing weak-growing rootstocks, robust varieties, open pruning, low nitrogen-level, and potentially covering trees with a roof. Apple scab can be reduced by spraying sulphur or sodium bicarbonate. Insect damage can be prevented using beneficials by sowing flowering stripes and putting up nest boxes for insect-eating birds.

\vspace{0.5em}
A trial involving five model organic apple orchards in 2004 found that the gross profit varied from - 83.000 kr/ha to +141.000 kr/ha, averaging 39.000 kr/ha, which was comparable to conventional production ($\approx$ 35.000 kr/ha). Main problems identified were attacks from aphids and scab.


\section{Lecture 19: Light utilisation and canopy management 26/09-2025}
\textbf{Torben Toldam-Andersen, KU-PLEN}

\vspace{1em}
The physiological context of fruit production emphasizes that Biomass Yield is determined by the formula: (available light)x(\% light intercepted)x(Photosynthesis)-Respiration. Of these factors, the percentage of light intercepted is the only one that can be directly manipulated. Efficient utilization of available light is of key importance in light-limited conditions, as Productivity is defined as a linear function of intercepted PAR (Photosynthetically Active Radiation). Photosynthesis and carbon allocation must be understood in this production physiological context, particularly concerning yield and quality components. A strong sink, such as a high number of fruits per $m^2$ leaf area, stimulates a higher source activity, resulting in increasing Net photosynthesis. An optimal leaf area of 600 $cm^2$ /fruit is required to support approximately 1 g $CO_2$ /apple/day.

\vspace{0.5em}
Canopy management techniques are essential for manipulating light interception and distribution. Light distribution is crucial, affecting development from the single leaf up to the canopy level. The light level must not be too low to form flowers, requiring a minimum of 30\% of full light for flower bud development. Light effects influence dry matter allocation, plant shape, fruit size, and colour. Data indicates that at only 10\% relative light level, the percentage of large fruits drops to 22\% and the percentage of fruits more than a quarter red drops to 5\% (Based on Hansen 1995). The artificial application of post-harvest light underlines the very strong effect of light on red over colour development.

\vspace{0.5em}
Light interception is fundamentally connected to the Leaf Area Index (LAI = $m^2$
leaf area/$m^2$ ground area). While light interception can be maximized through increased LAI, the optimal strategy requires a compromise between maximum interception and effective light distribution, ensuring the 30\% minimum light requirement is met. An overall light interception of 50-65\% and an LAI of 1.5(-2) is considered optimal. The theoretical maximum light interception ($F_{max}$ ground area). While light interception can be maximized through increased LAI, the optimal strategy requires a compromise between maximum interception and effective light distribution, ensuring the 30\% minimum light requirement is met. An overall light interception of 50-65\% and an LAI of 1.5(-2) is considered optimal. The theoretical maximum light interception ($F_{max}$) is reduced by light transmitted between trees to the ground ($T_f$) and light transmitted through the canopy ($T_c$). If LAI reaches 3, the canopy is generally deemed too dense.

\vspace{0.5em}
Methods applied to manage light interception include: 
\begin{enumerate} 
    \item Planting system (distances, orientation). 
    \item Canopy shape and size. 
\end{enumerate} 

\vspace{0.5em}
Increased light interception can also be achieved by maximizing Foliage density, using Multi-row systems, and increasing tree height in Single row systems. The leaves must be "arranged" in 3D in an optimal way. Row orientation effects show variation across cultivars and rootstocks.

\vspace{0.5em}
The importance of shoot type is evident as the leaf area on longer shoots is not correlated to yield, but the area on spurs is. Spur leaves are critical because they fast become exporters of assimilates, supplying the developing fruit during the first 5-6 weeks after flowering, which are critical for fruit set (fruit number) and cell division (fruit size). Spurs are highly fruitful due to a high density of flowers.


\section{Lecture 20: Canopy management/General techniques - 26/09-2025}
\textbf{Torben Toldam-Andersen, KU-PLEN}

\vspace{1em}
Canopy management encompasses methods aimed at manipulating growth and development in fruit crops. The primary aims are to control tree shape and esthetics, influence cropping and fruit development, increase fruit size and quality, facilitate the renewing of shoots, and improve overall productivity. Management is also critical for addressing space limits, improving work ease for machines and picking, and controlling tree height and shoot orientation. Furthermore, canopy management helps prevent and control diseases by removing sick shoots/parts, creating open trees or canopies, and optimizing spraying efficiency.

\vspace{0.5em}
Growth regulating methods function by affecting different parts of the tree. 

\begin{enumerate} 
    \item \textbf{Top/Root Mechanisms:} Planting density, Root stock/spur types, Pruning (top), and Root pruning affect both the Top and Root. These methods change the Top/root ratio, leading to 'root hormones' becoming dominant. 
    \item \textbf{Top-Only Mechanisms:} Chemical growth regulators, Arching/ringing/saw cuts (affecting Stem/transport), Bending, and Thinning (fruits, flowers) affect the Top. 
    \item \textbf{Root-Only Mechanisms:} Ground covers and Fertilizers/water affect the Root. 
\end{enumerate} 

\vspace{0.5em}
The removal of tissue (shoot or root tips) is equivalent to the removal of hormone producing tissue. Affected hormones include Cytokinins, Gibberellins, and Auxin, with Auxin sensitivity being crucial in controlling Apical dominance.
Pruning methods include the Shortening of shoot, Thinning/removal, and Shoot Positioning. Pruning intensity and timing significantly affect growth: Winter 'dormant' pruning is growth promoting, while Summer 'green' pruning is growth weakening. Summer pruning reduces root growth, delays bud break, and reduces total shoot growth, resulting in shorter shoots and more spurs. Medium late summer pruning may stimulate flower bud development, but late pruning risks reduced hardiness. The regrowth response is stronger from many small cuts than from a few larger cuts, and shortening cuts are stronger than thinning cuts (for similar mass removed). Vertical branches respond with stronger growth than horizontal branches or those at 30\textdegree.

\vspace{0.5em}
Pruning affects source and sink dynamics. It increases light penetration, which leads to changes in leaf structure and increased Photosynthesis. Strong pruning may result in compensating growth responses, such as increasing leaf size (e.g., from 9.0 $cm^2$ /leaf (No pruning) to 16.0 $cm^2$ /leaf (Strong pruning)). 
 
\begin{enumerate} 
    \item \textbf{Effects on Fruit Set:} Dormant pruning can be positive (due to improved water/nutrient/C availability) or negative (in case of high vigor). Early green pruning is positive due to a decrease in competing sinks. 
    \item \textbf{Effects on Yield:} Pruning generally delays cropping in young trees. In older trees, the decrease in fruit number caused by pruning can be compensated by increased fruit size (e.g., 181 fruits/tree at 123 g/fruit (No pruning) yields 22.3 Kg/tree compared to 146 fruits/tree at 145 g/fruit (Strong pruning) yielding 21.1 Kg/tree). 
    \item \textbf{Effects on Quality:} Fruit size increases due to increased light (source), increased root/top ratio (sink activity), and younger tissue/increased leaf/fruit ratio. Color development is generally improved if moderate growth is induced. 
\end{enumerate} 

\vspace{0.5em}
The total effect of pruning depends on the level and time of pruning, as well as the vigor level and density of the tree canopy.

\vspace{0.5em}
Pruning systems include the Spindel (central leader) and Espalie. The Slender spindel emphasizes ensuring a balance between the top (starting strong, becoming weak later) and the basis (3-5 permanent branches). Training involves maintaining a strong top shoot in young trees, preventing branching on side shoots, and removing or spur pruning strong shoots. Growth types, such as Spur types and Tip cropping types, exhibit different branching and renewal characteristics, necessitating adjustments in management practices, such as bending or tying. The principles of Zahn (1998) state that side shoots more than 50\% of the trunk diameter form a split or bushy crown.


\section{Lecture 21: Juice processing and Quality - 29/09-2025}
\textbf{Torben Toldam-Andersen, KU-PLEN}

\vspace{1em}
The lecture outlines the main concepts related to Fruit Juice and their importance in juice quality, recognizes the main processes involved in fruit juice processing, and identifies the theory linked to the sensory evaluation part of the course. Fruit juice is a product derived from fruits and berries, characterized by no addition of sugar, acid, aroma, or water; however, sweetened juice exists for products like Black currant and Sour cherry. Nectar is defined as juice combined with water and sugar.

\vspace{0.5em}
Juice processing involves a sequence of steps: Harvest, Wash, Grinding, (Enzyme treatment), Pressing, Fining, Filtration, Pasteurizing, Juice storage tank, and Concentration (resulting in Aroma Concentrate). Juice can be either clear or cloudy. Key components determining juice quality include Sugars (Fructose, glucose, (sucrose)), Acids (Malic acid, citric acid), and Phenolic compounds (Quercetin, Caffeic acid, Cinamic acid, Catechin). Important ingredients vary by fruit: apple typically contains 9-14 g/100g sugar, 0,5-1,2 g/100g acid, and 5-25 mg/100g Ascorbic acid. Sour cherry 'Stevnsbær' contains 18-22 g/100g sugar, 2,0-2,5 g/100g acid, and 10-15 mg/100g Ascorbic acid. Black currant contains 13-14 g/100g sugar, 3,0-3,5 g/100g acid, and 65-115 mg/100g Ascorbic acid. Anthocyanin content is highest in Black currant (350-450 mg/100 g).

\vspace{0.5em}
Sensory evaluation, specifically the correlation between the sensory impression of sugar/acid and the measured sugar/acid ratio, is critical for quality assessment, using a sensory scale from -5 (very sour) to +5 (very sweet), where 0 is the optimal sugar/acid ratio. Sensorically important aroma compounds are generally volatile (e.g., Acetateesters, Butanoateesters, Butanol) or less volatile (e.g., Benzaldehyde, Bensylalkohol, Linalool, Geraniol, Eugenol, Terpenes). Cultivars exhibit diverse aroma profiles, as shown by heat map analyses of 116 Volatile Organic Compounds (VOCs) across 8 apple cultivars.

\vspace{0.5em}
The Pressing Yield is defined as kg juice/kg fruit. The ripening stage is important: unripe fruits, with protopectin and insoluble midtlamella intact, offer good possibilities for run off, while very ripe fruits, possessing much soluble pectin and high viscosity, result in low press yields. The vacuole must be emptied for water and solutes during pressing. Pressing can be carried out using machines like a Belt Press or small hydropres for small-scale/hobby production. The resulting juice may be separated into Free-run and Press fractions.

\vspace{0.5em}
Enzyme treatment of the pulp increases juice yield. Pectin degrading enzymes, such as Polygalacturonase and Pectin esterase, are used. Pectin consists mainly of galacturonic acid, with some acid groups estherified with methanol ($COOCH_3$). Enzyme treatment results in Lower viscosity, Better possibilities for run off, an Increase of juice yield (from 70-75\% to 80-85\%), Higher dry matter content, higher sugar, and Higher Methanol content.

\vspace{0.5em}
Pasteurization, achieved by heating the juice to 90\textdegree C in 10-20 seconds, is necessary because, due to the low pH value of 3-4, only yeasts, fungus, and lactic acid bacteria can grow. Preservation methods for fruit juice include Concentration, aimed at reducing the volume from 100 l to 15-20 l. Concentration is achieved by Evaporation, Freeze-/cryo- concentration, or Reverse osmosis (membrane technology).

\vspace{0.5em}
A major challenge during concentration by heat evaporation is the loss of volatile aroma components, as Esters, alcohols, and terpenes are more volatile than water. If only 10\% of the water is evaporated, the concentration of aroma components has decreased to approximately zero. The concentration process, as illustrated, separates a 100 kg fruit juice (10\% DM) into a 90 kg juice fraction (11,1\% DM) and a 10 kg water + aroma fraction. Evaporation yields 20 kg concentrate (50\% DM) and 10 kg water vapour + aroma components, which, via a Retification column and Condensator, yield 1 kg Aroma concentrate. Aroma components in apple juice are more volatile than those in strawberry juice, requiring a greater volume to be used as aroma fraction from strawberry juice.

\vspace{0.5em}
Loss of anthocyanin and aroma during juice processing occurs through: 

\begin{enumerate} 
    \item Loss to the press cake 
    \item Destruction by heating 
    \item Evaporation 
    \item Loss by clarification 
\end{enumerate} 

\vspace{0.5em}
The loss of the ester Ethyl butanoate and the concentration of anthocyanins decrease significantly across processing steps such as heating, enzyme treatment, pressing, clarification, and filtering. Cryo-concentration is also a method used, involving cooling curves of sucrose and glucose.


\section{Lecture 22: Canopy management/ special techniques - 29/09-2025}
\textbf{Torben Toldam-Andersen, KU-PLEN}

\vspace{1em}
This lecture summarizes special techniques in Canopy management, specifically Summer pruning and Bending of shoots, which are utilized to manipulate growth and development. Summer pruning aims to open up too dense trees and is relevant for Espalier training systems. During summer pruning, unnecessary shoots are removed. Remaining shoots should be handled as follows: leaders, weak shoots and spurs are not to be touched, while others are pinched after 5-6 leaves. Several figures illustrate summer pinching techniques: Figure A shows the first summer pinching on shoot A at point b. Figure B illustrates the second summer pinching where shoot A developed shoot c, which is pinched at d. Figure C shows a second summer pinching where two shoots developed (c and d); in this case, the shoot c is removed by a cut at e, and shoot d is pinched at f, indicating that two developing shoots are considered too much. Figure D shows a similar scenario but with even stronger growth; the process involves removing the lower and weakest shoot. Figure E demonstrates a very strong growing shoot that was cut down to b in spring, resulting in all 3 buds starting to grow; the lower shoot e is pinched at g, and shoots c and d are removed with a cut at f. In problematic scenarios aimed at developing a fruiting spur (Figure F), where a spring cut at b led to three shoots (c, d, e), the lower shoot e is pinched at g, and c and d are removed at f, in the hope that the remaining shoot is weak enough to develop a spur. Another problematic scenario (Figure G) involves a shoot where lower side buds failed to develop into flowers, and only the terminal bud grew; here, the shoot is pinched short right above the base rosette of leaves as low as possible to force the lower buds to react. Figure H confirms the success of this stimulation two months later, showing a good spur development and the new shoot f pinched at g
See lecture sildes for illustrations.

\vspace{0.5em}
Bending of shoots has the primary aims of achieving Changed growth allocation and Flower bud development stimulation. Upright shoots and top exhibit the largest vigour. The degree of bending results in Reduction of vigour and Compensating growth. The degree of bending, which changes the allocation of growth and flower development, is demonstrated by quantitative data (Shoot angel, Shoot growth on bended part of branch, Shoot growth above the bended branches, Number of flowers year 2): an angle of 30\textdegree resulted in 522 cm shoot growth on the bended part and 4 flowers, while 120\textdegree resulted in 428 cm growth on the bended part and 95 flowers. Timing of bending is also crucial, as early bending results in many shoots. Studies comparing bending timing showed that March bending resulted in 28.3 shoots/tree, while bending on 1. August resulted in 8.9 shoots/tree, with shoot length varying between 32.9 cm and 37.3 cm. The mechanisms of bending involve Hormones and Apical dominance, specifically relating to Auxine and Ethylene (stress). It is important to be aware of the growth level (vigor); if the tree is weak (e.g., due to a weak root stock), bending may cause it to stop growing. Bending is an important tool in young trees.


\section{Lecture 24: Nutrients, use and effects on development - 06/10-2025}
\textbf{Torben Toldam-Andersen, KU-PLEN}

\vspace{1em}
This lecture details the physiological aspects of Nutrients and fertilisation, covering Root development, Mobility, Storage, Nutrient demand and uptake, and Effects on yield components. Mobility in soil is pH dependent, where Rt=pH+1/2. Due to nutrient uptake (ion exchange) by plants, pH will fall. Approximately 4 tons/ha of Calcium Carbonate is needed every 4 years to increase soil pH with 0,5. Optimal Rt values for nutrients range from 4.0-10.0, for instance N is 5.8-8.0 and Mo is 7.0-10.0. Mobility in plants classifies nutrients as Mobile (N, P, K and (Mg)), Less/partially mobile (S, Mo), or Immobile (Ca, B, Fe, Mn, Cu and Zn). Symptoms of deficiency depend on mobility, appearing on young leaves/top of plant, older leaves/Base of plant, or the whole plant, in addition to differences in coloring at the leaf level.

\vspace{0.5em}
Regarding Storage, nutrients move from leaves to tree storage in Autumn. N and P storage approximates 50\%, K storage is about 10\%, and Ca, Mg storage is only about 1\%. Early leaf drop may reduce storage N. Urea spray can increase leaf N and storage N in apple. Seasonal changes show that Arginine is the storage form, and Asparagine is the transport form of soluble N. In Spring, storage moves to new growth, causing concentration to fall. Low storage results in less shoot growth and lower fruit set. Remobilisation occurs in spring, confirming that new uptake is important, requiring early nutrient supply.

\vspace{0.5em}
Nutrient demand and uptake correlate with seasonal growth. The leaves form the major pool for nutrients, leading to the largest demand early season. Seasonal accumulation in apple fruits shows that K, P (and N) accumulation runs parallel with fruit growth, while Ca accumulation occurs only in early season. Total uptake estimates suggest N about 60 kg/ha, K about 60 kg/ha, and P and Mg about 10 kg. The type of growth influences demand. Vegetative growing plants demand double the amount of most nutrients, except K. Fruits contain very Low Ca, but are high in K.

\vspace{0.5em}
The effects of nutrients on yield components show that nutrient use (demand) is similar across fruit species. Small fruited species with a short period of fruit development show less impact of fruits on nutrient demand. Strawberry growers may vary nutrient solutions based on three growth phases: 

\begin{enumerate} 
    \item until flowering, 
    \item fruit development, 
    \item after harvest. 
\end{enumerate} 

\vspace{0.5em}
Flowering intensity in apple depends on cropping level and N status the year before. Moderate N level (1.9-2.6\% N in leaves) results in a higher percentage of maximal flower number than low N level (1.6\% N in leaves). Fruit set (relative value of flower number) increases with N\% of leaves just after flowering. For Ribes, shoot growth is linked directly to bud number, flowers, fruit number, and yield. 'Stevnsbær' and Elderberry also exhibit positive responses at high N levels, while Raspberry shows less positive responses. Strawberry response depends on the growing system: young plants in a short lifetime system need higher N levels, whereas older plants need less N to prevent them from becoming too dense.


\section{Lecture 25: Nutrients use and effects on quality - 06/10-2025}
\textbf{Torben Toldam-Andersen, KU-PLEN}

\vspace{1em}
This lecture focuses on the effects of nutrients on quality components in fruit and berry crops, covering Nitrogen (N), Phosphor (P), Potassium (K), Calcium (Ca), and Magnesium (Mg), along with micro nutrients, optimal status determination, and application timing.

\vspace{0.5em}
Nitrogen (N) negatively affects apple quality components, resulting in reduced red color (due to shade and a direct effect), more green fruits, and reduced firmness. N also reduces taste, decreases the concentration of acids and sugars, and potentially reduces aroma. Furthermore, N reduces storability. In berries, N results in less sugar, but has no effect on acid. N affects growth and development, including vegetative growth and generative growth (flowers, fruit set). Its effects on fruit development include colour (red and green) and aroma. N effects are often indirect, dominating via the fruit/leaf ratio or carbohydrate availability, although direct effects may also exist.

\vspace{0.5em}
Phosphor (P) is not a critical nutrient in DK but may improve storability.
Potassium (K) is characterized by a high demand in fruits. However, high K and high yields can be negatively correlated to bud development. K can also increase biannual bearing and induce deficiency of Mg and Ca due to the K/Ca+Mg ratio. Thus, it is crucial to be careful not to use too much K.

\vspace{0.5em}
Calcium (Ca) is important for storability. Studies show the percentage of bitter pit attack is a function of the calcium content of apple fruits. The incidence of 'Møsk' (tissue breakdown) is also correlated with Ca content in fruits. Ca is only mobile in the xylem and not in the phloem. Seasonal uptake in fruits of Ca occurs during development. Sprays with CaCl$_2$ during late season may increase the Ca level. Xylem function, as observed in Braeburn' and Granny Smith' fruits assessed at 64 and 67 DAFB, shows differences in vascular function.

\vspace{0.5em}
Magnesium (Mg) is part of Chlorophyll. Deficiency of Mg leads to necroses and leaf drop, which can be treated with leaf sprays of MgSO$_4$ early season. Boron (B), Zinc (Zn), and Iron (Fe) are the most important micro nutrients. There is a danger of excess levels of micro nutrients, and their content may be found in fungicides (e.g., Dithane). Leaf sprays with these minerals are frequently used.

\vspace{0.5em}
Optimal nutrient status determination using symptoms of deficiency is often seen too late. Soil samples are difficult in fruit cultures as they vary with depth, in and between rows, drip irrigation, soil types, etc.. Leaf samples are considered the basis of next years application. They should be taken in mid - late August, specifically from the middle of annual shoots.

\vspace{0.5em}
Regarding application: N should be applied early spring, with high levels requiring 2/3 early spring +1/3 June (mid summer). K should be applied in winter (KCL). Ca is applied via leaf sprays of CaCl$_2$ (0.75\%) late season. Micro nutrients are applied via leaf sprays early season. Nutrients should be applied in the tree row, and Fertigation is used a lot.

\section{Lecture 23: Stresses and effects on quality - 10/10-2025}
\textbf{Torben Toldam-Andersen, KU-PLEN}

\vspace{1em}
This lecture addresses stresses and quality in fruit and berry crops, covering the use of water stress, damage from low and high temperature stress, prevention of damage, and damage caused by wind and biological stresses.

\vspace{0.5em}
Low temperature stresses include freezing damage to berries (braun and dry out) and freezing in open flowers. Damage from frost (freeze) protection with water is based on the principles of high thermal mass of water and latent heat. Latent heat is the energy transferred during a phase change. The phase change from liquid water to ice transfers 80 cal of latent heat, which is crucial for frost protection. In comparison, 1 cal (4.17 Joule) is required to increase the temperature of 1 g of water by 1\textdegree C, and 600 cal of latent heat is transferred during a phase change from vapour to liquid water. Frost damage, post bloom and Winter damage are also covered.

\vspace{0.5em}
The water vapour content of the air is important: a high content results in a slow temperature drop during the night, while a low content results in a quick temperature drop. The location near the coast is important for protection against frost damage. In spring, the minimum temperature near the coast is higher and the maximum is lower, which results in delayed Budbreak and lowered frost risk. Data for 'Red Boskoop' apples (1975-2022) indicates that the beginning of flowering is now >3 weeks earlier than in 1975, corresponding to a temperature rise of >1.5 K since 1975. This makes frost protection by irrigation increasingly important. Frost protection methods include Overhead sprinklers and Water tubes with micro sprinklers at each tree, as well as using Heaters/burners. Radiative frost occurs on clear and calm nights and is very local. Techniques to mitigate frost include using Wind mixers (e.g., in Washington State, US) and sand cover (e.g., Soil protection in China).

\vspace{0.5em}
High temperature and water stresses include fruit cracking due to water problems. This has been observed in Washington, US and Ullensvang, Norway. Sunburn damage results from too much sun. Hail, especially when combined with strong wind, can be very harmful. Hail can also severely damage the shoots. Rain, hail and sun protection methods are available.

\vspace{0.5em}
Wind stress/damage requires the establishment of shelters. Pest and disease stresses lead to various negative outcomes: 

\begin{enumerate} 
    \item Reduce leaf carbon assimilation and export 
    \item Deformities of fruits (spots, cracks, splitting) 
    \item Less attractive 
    \item Lower quality 
    \item Lower yield 
    \item Fruit rot 
    \item Bad taste 
    \item Lower storability 
\end{enumerate} 

\vspace{0.5em}
The shoots can also be severely damaged.


\section{Lecture 28: Potentials in fruit and berries for fruit wine production in Denmark - 10/10-2025}
\textbf{Torben Toldam-Andersen, KU-PLEN}

\vspace{1em}
This lecture explores the Potentials for (Danish) fruit wines through the 'NATVIN' Project, focusing on Apple cultivars, the content of apple juice, the importance of concentration, wine potentials, the process (How to do it), Quality impact, and the Benefits compared to alternatives to cryo concentration. For ordinary juice containing 12\% Brix, the composition is typically 120 g/L sugar, 8 g acid, and 2 g sorbitol, resulting in 110 g Fermentable Sugars (FS). This concentration allows for a yield of 52 g alcohol or 6,5\% vol Alc.. Wine acidity levels dictate the requirement for malolactic fermentation (MLF), defined as the conversion of malic acid to lactic acid by bacteria's (Oenococcus oenie). Moderate acid levels facilitate dry wines without MLF, while high levels necessitate MLF or wines with residual sugars (RS).

\vspace{0.5em}
The implementation of cryo concentration expands wine potentials, increasing 12\% Brix to 20\% brix. This results in sugar content rising from 110 to 180-185 g/L, achieving 10,8\% vol alc.. Acid levels concurrently rise from 5-8 g/L to 10-13 g/L, often requiring some MLF. Sparkling wines require Bottle Fermentation of added sugar to 4-5 bar, adding +1-1,2\% vol alc., thereby reaching 12\% total alcohol. Still wines made from 22\% brix can reach 12\% vol alc., generally requiring MLF. High Concentrate wines achieve 30-35\% Brix, suitable for Sweet dessert wines, 'ice wines', although high acidity is a challenge, potentially exceeding >20 g/L.

\vspace{0.5em}
The cryo concentration process involves freezing the juice, then thawing it slowly to yield concentrated juice. The resulting ice holds 2-5\% Brix. Thawing often utilizes 1000 L tanks, where juice from multiple tanks is collected into one. This method can result in an aroma explosion, with +40 aromas increased with more than 3x from control juice wines to high concentrated wines. Specific volatile increases include Dodecanoic acid, ethyl ester (314x) and Decanoic acid, methyl ester (178x). Only Ethyl lactate was observed to disappear, reduced 24x. Fufural content is also monitored, raising questions regarding Maillard reaktions and/or caramellisation. Alternatives, such as Membrane filtration techniques (Reverse osmosis, Cross flow filtration), are problematic for cloudy juice due to high pectin content, necessitating viscosity reduction, and strong filtered juice may be difficult to ferment.

\vspace{0.5em}
Berries pose unique challenges due to relative low in sugar and high or very high acid content (20-35 g/L), primarily citric and malic acid. Since these acids only form soluble salts, chemical reduction is impossible. Biological reduction via MLF is risky due to potential off flavour (Diacetyl)?, compounded by low or very low pH. Berry juice also exhibits very characteristic aromas, high phenolic content and often high in colour, and lots of pectin. Solutions involve Blends with water or a base of apple or pear. High acid species include: 
\begin{enumerate} 
    \item Sour cherry 'Stevnsbær' 
    \item Black currant 
    \item Red and White Currant 
    \item Gooseberries 
    \item Rhubarb 
    \item Black berries 
\end{enumerate} 

\vspace{0.5em}
Potential techniques and styles include Sparkling, Liquor (juice+alcohol), Fortified wines, Ice wines, and Still wines. The average soluble solids in 114 Gooseberry cv's is 11,8\%.


\section{Lecture 30: Growing of wild berries - 20/10-2025}
\textbf{Martin Jensen, senior scientist, martin.jensen@food.au.dk}

\vspace{1em}
This lecture addresses the domestication of European Blueberry (Vaccinium myrtillus L.) for a new berry crop, focusing on wild fruits and the plant's morphology. The European Blueberry is an erect, woody, rhizomatous shrub that grows to a height of 5-90 cm. It exhibits tiller growth from the rhizome in a 3-angled axis, developing abundant annual shoots in a complex structure. Leaves are glabrous acute, 1-3 cm long, and flowers occur singly or in pairs in axil leaves, reaching 4-7 mm in diameter with a pink corolla. The plant forms patches through rhizome growth, with aerial shoots reaching 20-30 cm. Native wild populations of Vaccinium myrtillus are present in all regions of Denmark.

\vspace{0.5em}
Wild harvested bilberries in Sweden are part of a larger yield in Nordic countries, estimated at 500 million kg per year, though only 5-8\% is exploited. Traditional berry picking is often for personal consumption and local markets, involving a small and fragmented industry that exports to East Asia, with a price typically 20-30 DKK/kg. Yield in nature ranges from 0-450 kg/ha. Hand picking often employs manual rakes, with 3-4000 Asian people flying in annually to pick. For comparison, wild harvesting of lowbush blueberry (Vaccinium Angustifolium) in Canada, conducted as semi-cultivation on 2x50.000 ha of wild plants with a two-year production cycle, yields 99 million kg of berries per year.

\vspace{0.5em}
European blueberry is of interest due to its status as a 'Superfruit', attractive taste, and high price. Potential health effects include lowering cholesterol and blood pressure, improving cognitive performance, and exhibiting antibacterial and vision benefits. It possesses a higher concentration of antioxidants than highbush blueberries. Currently, there is no commercial orchard production, relying only on 'natural collection'. Manual harvesting leads to high costs, limiting production and supply, and is deemed non-sustainable. Market demand is increasing globally, and there are no reported cultivars or breeding/selection efforts done. Orchard production using selected cultivars and mechanical harvesting could create new product options due to lower costs, leveraging the fact that it is a native Danish species with available climate-adapted genetic resources.

\vspace{0.5em}
The domestication of European Blueberry requires overcoming critical barriers: 

\begin{enumerate} 
    \item Significant improved berry yield/plant is needed (selection and breeding), as maximum harvest in nature is too little (around 450 kg to 1 ton/ha). 
    \item Significantly more efficient and non-expensive methods of vegetative propagation are required, as past success has been minimal. 
    \item Development of a sustainable orchard concept/design adapted to mechanical harvesting is necessary. 
    \item Development of efficient mechanical harvesting is critical to reduce the cost of harvest, as Canadian solutions may not be adaptable. 
\end{enumerate} 

\vspace{0.5em}
The domestication project involved collecting clones, establishing a common garden trial of >100 genotypes, characterizing these genotypes for plant growth performance, berry yield, and berry quality. The established collection includes 113 clones from 40 populations across two sites, used for gene conservation, clonal comparison, and as a source of superior genetics.

\vspace{0.5em}
Studies on fertilizer composition established that liquid fertilizers with N values of 200 and 400 ppm N damage and kill plants after 1-3 months. Optimal fertilizer concentration is 50-100 ppm N (EC = 0,3-0,6). 100 ppm N is optimal but only slightly better than 50 ppm in young seedlings, while 50 ppm was found better in cuttings. 25 ppm N suggests P deficiency (red leaf colour). A liquid fertilizer composition was developed for continuous use, using collected rainwater for irrigation, characterized by low levels of Ca, Mg, and Fe.

\vspace{0.5em}
Berry quality analysis of 53 genotypes in 2016 showed mean total soluble solids of 9.1\textdegree Brix, total titratable acid of 0.895 g citric acid eqv./100 g, and a pH of 3.14. Total anthocyanins averaged 330.4 mg CGE*/100 g. Superior clones identified in 2016 achieved high berry production, with the top clone yielding 115 g/plant/year, leading to a potential theoretical yield of 16.56 tons/ha/year (assuming 16 plants/m$^2$). The average yield of all 106 clones was 3.49 tons/ha/year. Observations 11 years after planting showed top genotypes yielding 90-150 g/plant, equaling a potential yield of 8-13 tons of berries/ha (assuming 9 plants/m$^2$). The relationship between mean berry production and the mean weight of the 10 largest berries showed a positive correlation (R$^2$=0,3553). The research has demonstrated efficient and low-tech methods for cutting propagation, a liquid fertilizer composition, and characterized over 100 genotypes, pointing to possible superior genotypes for cultivation and further breeding.


\section{Lecture 31: Bioactive compounds in fruit and berries - 20/10-2025}
\textbf{Martin Jensen, senior scientist, martin.jensen@food.au.dk}

\vspace{1em}
This lecture examines Bioactivity, defined by the effects or possible effects in humans from oral intake of compounds, relating to maintaining health, preventing disease, or resulting in a direct effect in the primary body or indirectly through a prebiotic or antibacterial effect. Regulatory categories include Medicine, Plant medicinal compounds, Functional foods, Supplemental foods, Food, and Novel Foods. Health claims, which must be based on scientific evidence evaluated by EFSA, include 'Function Health Claims' (related to growth, psychological functions, slimming), 'Risk Reduction Claims' (e.g., plant stanol esters reducing blood cholesterol), and 'Health Claims referring to children's development'. Documentation supporting health claims ranges from epidemiological evidence and in vitro tests to human clinical intervention tests, specifically recommending double blinded - placebo controlled experiments for confirmation.

\vspace{0.5em}
Examples of authorized EFSA claims include Pectins (reduction of blood glucose rise and maintenance of normal blood cholesterol) and Melatonin (alleviation of subjective feelings of jet lag and reduction of time taken to fall asleep, e.g., in sour cherry). Non-authorized claims include antioxidants from pomegranate juice and anthocyanins from elderberries.

\vspace{0.5em}
Fruit and berries contain compounds with potential health effects, categorized as: 

\begin{enumerate} 
    \item Anti-bacterial, Anti-viral, and Prebiotic effects. 
    \item Anti-oxidant, Anti inflammation, Anti- diabetic effects. 
    \item Protective against CVD - cardio vascular diseases (e.g., LDL cholesterol lowering, blood pressure lowering). 
    \item Anti-carcinogenic, Anti-mutagenic, Anti-toxic, detoxifying effects. 
\end{enumerate} 

\vspace{0.5em}
Estimated plant phenolics exceed 8000 polyphenolic compounds, including over 4000 flavonoids. Antioxidant capacity (FRAP, mmol/100 g) varies greatly, with Dog rose scoring 39.46 and Apple (Gold del) scoring 0.29. Genetic factors (cultivars), maturity, climate variation, cultivation methods, processing, and storage all cause variation in the concentration of single compounds, such as anthocyanins.

\vspace{0.5em}
Bioavailability studies indicate that anthocyanins and large polymeric tannins generally show low uptake and low concentrations in blood. Anthocyanin metabolites reach maximum concentration after 1-2 hours and are depleted after approximately 10-12 hours. Clinical trials on single fruit species are few, often showing limited effects due to too low concentrations of active compounds in raw fruit. Research on Aronia suggests extracts or concentrates may function as functional food for disorders related to oxidative stress, but more confirmatory clinical trials are needed. A mechanism for cholesterol lowering is the ability of fruit and berries to bind and excrete bile acids in the intestines. Therapeutic effects may be achieved through additive effects in portfolio diets. Specific compounds like Galacto lipid (GOPO) and two triterpene acids from Rose hip fruit are associated with pain reduction in osteoarthritis, as GOPO partly inhibits chemotaxis of human peripheral blood PMN's. Future research requires rigorous experiments, focusing on understanding compound metabolizing in the gastro-intestinal tract and modulation by bacterial flora.


\section{Lecture 32: Water management and water stress - 24/10-2025}
\textbf{Torben Toldam-Andersen, KU-PLEN}

\vspace{1em}
Water is the most abundant constituent of plant tissues and functions as an excellent solvent for salts, other solutes, and gases. The majority of absorbed water is lost to the surrounding air via transpiration. Insufficient water limits horticultural and agricultural production in many areas. Future trends suggest an increase in extreme weather situations, such as intense rain events or droughts. Producers must therefore regulate their irrigation strategy based on crop needs and weather situations. Climate change is projected to increase irrigation needs; for instance, by the 2050s, eastern, southern, and central England are predicted to have greater irrigation needs than those currently experienced anywhere else in England.

\vspace{0.5em}
Water flow through the plant (soil $\rightarrow$ plant $\rightarrow$ atmosphere) is driven by a gradient in water potential ($\psi$). Water potential is defined as the energy state of water in the environment, which governs water availability. The total water potential is calculated as $\psi=\psi_p+\psi_s+\psi_g+\psi_m$, where $\psi_p$ is pressure or hydrostatic potential (turgor), $\psi_s$ is osmotic potential (binding to solutes), $\psi_g$ is gravitational potential, and $\psi_m$ is matric potential (binding to surfaces). In soil, the matric and osmotic potentials are the main components, whereas in plants, hydrostatic/turgor and osmotic potentials dominate. Net photosynthesis is correlated with stomatal conductance in apple leaves under varying water stress levels.

\vspace{0.5em}
When the soil dries, the plant initiates physiological reactions. Moderate soil drying results in a slight decrease in root water potential, an increase in ABA (abscisic acid) in roots, and increased root growth. Subsequently, ABA and pH increase in the xylem, while nitrate decreases. The leaf water potential is kept stable, although stomata gas-exchange and leaf expansion decrease, ABA in the leaf increases slightly, and nitrogen in the leaf decreases slightly. Severe soil drying causes a strong decrease in root water potential, strong increases in root ABA, and strong root growth. The leaf water potential decreases, stomata gas-exchange (including A$_{max}$) decreases strongly, leaf expansion decreases, and nitrogen in the leaf decreases. Water stress affects vegetative development (root and shoot growth), flower bud formation (impacting the next year's crop), yield, and quality.

\vspace{0.5em}
Plant water use and consumption depend on various factors: 

\begin{enumerate} 
    \item Water reserves, which are determined by soil type. 
    \item Transpiration and evaporation, which are determined by temperature, wind, and humidity. 
    \item Leaf area, which depends on plant species, growing system, plant age, and plant shape. 
    \item Root size, which depends on plant species, growing system, plant age, and rootstock. 
\end{enumerate} 

\vspace{0.5em}
Cropping influences physiology in apple, leading to higher values compared to uncropped controls for photosynthetic intensity (100 vs. 140 - 210) and transpiration intensity (100 vs. 150 - 300). The water use coefficient measures L water used/m$^2$ leaf area/mm water evapotranspiration, and water use efficiency measures kg fruit/L water used.

\vspace{0.5em}
Regulated water stress can be utilized to benefit yield and quality in fruit trees and vine production. Sensitivity varies greatly across species and cultivars. Shoot growth stops when new leaf development is reduced by water stress (increasing negative potential), which effectively controls shoot growth and induces terminal bud formation. Growers must control the water status to remain within the 'window' between the effect on shoot growth and the effect on photosynthesis to maintain dry matter production. Mild stress stimulates root growth, while strong stress reduces roots, particularly near the soil surface. Competition for water can be utilized, as green cover evaporates 50-90 mm/month (about 2-3 mm/day) compared to bare soil which evaporates 25-35 mm/month (about 1 mm/day), and covering with straw can save approximately 60 mm in a season.

\vspace{0.5em}
Measurement techniques include Tensiometers, which use a porous ceramic cup to reach equilibrium with the soil solution's water potential, and Time Domain Reflectometry (TDR), which measures volumetric soil water content using electromagnetic waves. Water management strategies are applied to crops such as Apple and pears, Strawberry, Stone fruit, and Bush and cane fruits (e.g., Black currant and raspberry), sometimes in combination with root pruning.

\vspace{0.5em}
Water stress reduces fruit set, berry size, and inflorescence initiation and development. The period most sensitive to stress is the first few weeks after flowering, where berry enlargement can be permanently limited. Just before ripening ('veraison'), there is little effect on fruit expansion. During ripening, mild stress enhances sugar and color content and reduces acid, promoting ripening. Strong stress during ripening can delay and reduce sugar and color. Reduced shoot growth limits competition between vegetative sinks and fruit development.


\section{Lecture 33: Vitamins and colours in fruit and berries - 24/10-2025}
\textbf{Torben Toldam-Andersen, KU-PLEN}

\vspace{1em}
The internal quality of fruits, specifically concerning vitamins and colour, is influenced by climate effects and growing techniques. Fruit colours are derived from three main compound classes: Green colour is attributed to Chlorophyll a and b; Yellow colours originate from Carotenoids, including Carotins (C$_40$H$_56$) like Beta-carotin and Lutein, and Xanthophylls (C$_40$H$_56$O$_2$) such as Violaxanthin and Neoxanthin; and Red colour is due to Anthocyanins. Seasonal changes affect the concentration of the four main carotenoids in ‘Golden del.' apple.

\vspace{0.5em}
Anthocyanins are the dominant red colours found in berries. Based on data from Kaack 1975, the dominant anthocyanins include: 

\begin{enumerate} 
    \item Black currant: Cyanidin-3-glucosid, Delphinidin-3-glucosid 
    \item Elderberry: Cyanidin-3-glucosid, Cyanidin-3-sambubiosid 
    \item Strawberry: Cyanidin-3-glucosid, Pelargonidin-3-glucosid 
    \item Cherry: Cyanidin-3-glucosid, Cyanidin-3-sophorosid 
\end{enumerate} 

\vspace{0.5em}
The synthesis of Anthocyanins begins with green leaves and light producing carbohydrates. Cyanin (anthocyanidin) is produced in the presence of light, and adding galactose (sugar) completes the synthesis of anthocyanin. This process involves the Pentose phosphate pathway, Shikimic acid pathway, Glycolysis, and Malonacid pathway, with glycosidation as the final step.

\vspace{0.5em}
Colour development is regulated by factors including Fruit/leaf ratio (light indirect), Nitrogen, Light (specific via PAL), and Temperature. The ratio of Kg fruit/Kg leaves or Number of fruits/kg leaves influences the yellow and green colour of ‘Golden delicious' apple. Temperature effects are observed during early and late stages of ripening. In unripe fruit, Light combined with warm temperature increases fruit growth and sugar accumulation. In mature fruit, red colour is enhanced by light and warm temperature, as well as by light and cold (night) temperature. Climate affects red colour development, demonstrated by treatments of Control, 50\% shade, and Warm night conditions applied to Jonagold apples from mid-August to October. Post-harvest colour development is also influenced by artificial light exposure on bagged fruit. In Black currant, high Nitrogen (N) levels increased the concentration and content per fruit of chlorophyll and carotens. Leaf removal increased the concentration of chlorophyll and carotens, but decreased the content per fruit due to resulting smaller fruits. Anthocyanin concentration was affected only to a limited degree by N level or leaf removal, although fruits in the top of the canopy exhibited higher anthocyanin content compared to berries at the basis, which suggests the importance of light conditions and assimilate availability.

\vspace{0.5em}
Vitamin C comprises Ascorbic acid plus De-hydro-ascorbic acid. Vitamin C content varies significantly across species in mg/100 g Fresh Weight (FW), ranging from Apple (5-10) and Pear/Grape (5) to Orange (50-60), Black currant (180), and Rose hip (400-2000). Vitamin C is formed before the ripening phase, with concentrations affected by ripening and storage effects, Light (sun/shade, south/north, peel), and other factors like fertilizers/nutrients, often through light or developmental degree. Vitamin C degrades very fast. The distribution of Vitamin C within an apple shows highest concentration in the Epidermis (70 mg/100 g fw), declining significantly in the inner flesh (6 mg/100 g fw) and core (5 mg/100 g fw), leading to the conclusion: do not peel an apple. Light exposure also affects content, with the south side of an apple fruit having 34 mg/100g fw compared to 20 mg/100g fw on the north side. The effect of development on Vitamin C is varied; it remains stable in Strawberry (e.g., 50$\rightarrow$54 mg/100 g fw from unripe to ripe) and Apple (e.g., 10$\rightarrow$13), but decreases in Black currant (219$\rightarrow$140) and Sour Cherry (20$\rightarrow$12). Gooseberry cultivars show an average of 30.9 mg/100 g fresh weight. Storage conditions (Controled Atmosphere, Ventilated, Cold Storage, or room temperature) also impact Vitamin C levels.


\section{Lecture 34: Fruit harvest and postharvest - 27/10-2025}
\textbf{Torben Toldam-Andersen, KU-PLEN}

\vspace{1em}
Fruit development involves specific phases of growth and the role of ethylene, leading up to harvest. Harvest time represents a crucial compromise between optimizing internal quality and maximizing storability. Internal quality is assessed via chemical changes, while external qualities include size and colour. Seasonality significantly impacts fruit characteristics. In the early season (e.g., strawberries), a cool climate results in low sugar and high acid content, and phenology favors the first and largest fruits, with cultivar differences being notable. The main season is often warm and dry, accelerating development, which increases the danger of harvesting too late and resulting in overripe fruit. The late season (e.g., apples) is cool and humid, characterized by lack of light, and involves the last and smallest fruits, alongside increasing risks of pests, diseases, and fruit rot, impacting storage products.

\vspace{0.5em}
Key harvest criteria are essential for balancing the goals of high eating quality and maximum storability. Important parameters include fruit size, colour change, firmness, removal force, sugar/acid ratio, starch content, aroma, and the levels of ethylene, O$_2$, and CO$_2$. The Streif index is calculated as firmness (kg/cm$^2$)/ sugar (\%) * starch index (0-10) and is dependent on both cultivar and climate. Cultivar differences are highly significant, especially regarding firmness in pears, which may change very fast. Prediction models, such as those relying on the temperature sum in 60 days after flowering for apples or the first 40-50 days determining phenology for cherries (Vittrup Christensen, 1973), aid in determining optimal harvest dates. For black currant, as fruits do not ripen uniformly, the harvest time is a compromise reached when some fruits begin to dry out and drop; machine harvest requires care to avoid damaging shoots and leaves. The optimal harvest date for storage can be related to the development of butylacetat (butyl etanoate), which is the most abundant aroma in apple juice.

\vspace{0.5em}
Harvest and post-harvest losses can be very high, reaching 20\% in Gross sale + Detail, and 20\% for the consumer. Fruits are heavy, water-containing, alive, and undergo respiration and senescence, giving them short keep ability. They are fragile and sensitive to bruising and wounding, which causes ethylene release, increased water loss, and infection risks. High losses stem from damaged/reduced appearance, increased respiration and water loss, and infections. Prevention methods include careful harvest, appropriate boxing, and managing whether the stalk is included. Mechanical harvest methods include platforms for apples/pears, machinery for industry crops (e.g., 'Stevnsbær'), cane fruit harvesters, and portal harvesters for grapes in adapted growing systems.

\vspace{0.5em}
Post-harvest physiology is strongly governed by temperature. Higher respiration equates to shorter shelf life. The Q10 ratio, which measures the respiration ratio at two temperatures 10\textdegree C apart, is typically 2-3, but can be 3-4 for fruits, emphasizing the dramatic effect of temperature. Fast cooling is necessary; for instance, strawberry Q10 $\approx$ 4, meaning shelf life drops from 7-8 days at 0-2\textdegree C to about 2 days at 10\textdegree C. A delay in cooling of 6 hours can result in 50\% higher water loss and loss of C-Vitamin and sugar. Fruits are categorized by ripening patterns: \begin{enumerate} \item Climacteric: Respiration and ethylene production increase (e.g., Apple, Pear, Banana). \item Non-climacteric: Respiration gradually falls during development, indicating non-ethylene sensitive ripening (e.g., Cherry, Strawberry, Citrus). \end{enumerate} To minimize drying out, fruits should not lose more than max 2-3\% of water. Water deficit is the primary factor, heavily dependent on temperature. Water loss occurs 30\% via lenticels and 70\% via the cuticle. Minimization strategies include cooling, wrapping/foils, waxing, and the use of humidifiers in stores.

\vspace{0.5em}
SmartFresh, a commercial brand of 1-MCP (1-Methylcyclopropene, C$_4$H$_6$), is structurally related to ethylene and functions as a plant regulator to inhibit ethylene action by binding to its receptors. 1-MCP effects conclusions show that early treatment (at harvest) strongly reduces aroma development, while later treatment (after light exposure) only results in partial effects. 1-MCP maintains higher firmness. Furthermore, treatment applied before light exposure reduces the fruit's ability to develop red over colour. The application time significantly affects the concentration of specific aroma compounds like propanol and ethyl acetate in pigeon apples.


What up ma nigga