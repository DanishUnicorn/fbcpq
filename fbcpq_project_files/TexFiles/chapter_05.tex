\chapter{Exam Questions and Answers}
\setlength{\headheight}{12.71342pt}
\addtolength{\topmargin}{-0.71342pt}

This chapter of the course notes compiles the exam questions for the course held in November 2025, along with their respective answers prepared by me. The purpose of this section is twofold: firstly, to provide a reflective exercise that consolidates understanding of the course material; and secondly, to document my comprehension of the course topics as assessed through the exam questions.

\vspace{1em}
To ensure citation accuracy and academic transparency, NotebookLM has been employed as the primary generative AI platform. Its use has focused on verifying that all citations accurately reference the uploaded course materials and lecture slides provided by the professors. Beyond citation control, this section also represents an ongoing exploration of prompt engineering — refining interaction design to optimise AI output quality, precision, and academic reliability. Through this approach, the work aims to maintain a high academic standard while enhancing clarity, structure, and depth in written responses.

\vspace{1em}
There are a total of 17 questions in the exam, each comprising between three and five sub-questions. The numbering of the sections in this chapter corresponds directly to the numbering of the exam questions, ensuring a clear and consistent structure throughout. Questions 1-9 address aspects related to crop physiology, while questions 10-17 focus on fruit quality, maturity, and usability. Each question is presented below, followed by its respective sub-questions and answers.

\section*{Questions within: Crop Physiology aspects}

\section{Yield and quality determinants and components}
\textbf{Shoot and bud development, growth and flower bud development}

\subsection{Characterise the development and importance of spurs and extension (long) shoots}

A \textit{spur} is a short, slow-growing lateral branch that bears flower buds, while an \textit{extension shoot} is a long, one-year shoot responsible for primary growth and framework development, as seen in Figure~\ref{fig:spur_extension}.

\begin{enumerate}
    \item \textbf{Spurs}
    \subitem \textit{Development:} As seen in the figure, spurs are short side branches that develop from lateral buds on older wood. They grow slowly over several years, forming compact structures with closely spaced buds. 
    \subitem \textit{Importance:} Spurs are perennial and bear flower buds, making them the main sites of flowering and fruit production. Their stability ensures regular cropping and helps maintain yield consistency.
    
    \item \textbf{Extension shoots}
    \subitem \textit{Development:} The long shoot, labelled as the extension shoot in the figure, originates from terminal or vigorous lateral buds and grows actively during one season. It extends the framework of the branch system and promotes primary growth.
    \subitem \textit{Importance:} Extension shoots are essential for canopy renewal and positioning of future spurs. They provide space and light for developing buds and balance vegetative and reproductive growth.
\end{enumerate}

\begin{figure}[h]
    \centering
    \includegraphics[width=0.4\textwidth]{figures/spur_extension.png}
    \caption{Illustration of a fruit tree twig showing the main shoot types. The \textbf{extension shoot} represents one-year primary growth that elongates the branch. \textbf{Spurs} are short, slow-growing side branches that develop on older wood and typically carry \textbf{flower buds}, which later form fruit. \textbf{Vegetative buds} along the extension shoot give rise to new shoots or leaves in the following season.}
    \label{fig:spur_extension}
\end{figure}


\subsection{Describe differences in bud development and structure between stone and pome fruits}

An example of a pome fruit and a stone fruit are illustrated in Figure~\ref{fig:pome_stone_buds}. The key differences in bud development lie in the presence of mixed buds in pome fruits, which contain both floral and vegetative tissues allowing continued growth after flowering, whereas stone fruits have separate simple buds where flowering terminates growth at that point \cite*{rm_02_L03_buds_bud_development,rm_03_L04_flower_bud_formation}.

\begin{enumerate}
    \item \textbf{Pome fruits}
    \subitem Possess \textit{mixed flower buds} that contain leaves, flowers, and a bourse-shoot bud, allowing the shoot to continue growing after flowering \cite*{rm_02_L03_buds_bud_development}. These mixed buds typically develop six to seven flowers \cite*{rm_03_L04_flower_bud_formation}.

    \item \textbf{Stone fruits}
    \subitem Have \textit{simple or naked flower buds} that contain only leaves or only flowers \cite*{rm_02_L03_buds_bud_development}. The development of a stone fruit flower bud results in the loss of the growth point at that position, often leaving bare sections along the shoot in certain cultivars. Stone fruit flower buds generally produce fewer flowers, typically one to five \cite*{rm_02_L03_buds_bud_development}.
\end{enumerate}


\begin{figure}[h!]
\centering
\includegraphics[width=0.75\textwidth]{figures/bud_development_and_structure_stone_vs_pome.png}
\caption{Comparison of bud development and structure in pome and stone fruits. \textbf{Pome fruits} (left) such as apple and pear have \textit{mixed buds} containing leaves, flowers, and a bourse-shoot bud that allows continued growth after flowering. \textbf{Stone fruits} (right) such as cherry and peach possess \textit{simple buds}, where flower and vegetative buds are separate; once flowering occurs, the growth at that position ceases.}
\label{fig:pome_stone_buds}
\end{figure}


\subsection{Describe some important yield components in strawberry and in sour cherry}

Important yield components determine the final fruit yield per hectare (t/ha) as the product of fruit number and fruit size \cite*{rm_01_L02_fruit_yield_quality}. The main yield-determining factors differ between strawberry and sour cherry due to differences in reproductive structures and growth habits:

\begin{enumerate}
    \item \textbf{Strawberry}
        \begin{enumerate}
            \item The \textit{number of plants per hectare} forms the base for yield potential and depends on the growing system and establishment of rooted runners \cite*{rm_01_L02_fruit_yield_quality}.
            \item \textit{Berry weight} is a major component that correlates linearly with the number of achenes (the seeds on the surface of the berry) per berry at a given density \cite*{rm_07.2_L11_13_fruit_growth_quality}.
            \subitem Achenes stimulates flesh growth through hormone production, increasing the berry size \cite*{rm_07.1_L11_13_fruit_growth_quality}.
            \item \textit{Flower quality} determines the number of ovule primordia (the early stage of seed development inside the flower) and hence potential achene number, directly influencing final berry size \cite*{rm_07.1_L11_13_fruit_growth_quality}.
            \item \textit{Berry number per plant} depends on vegetative growth and spacing; higher inflorescence numbers (flower clusters) may increase fruit count but reduce berry size due to competition \cite*{rm_07.1_L11_13_fruit_growth_quality}.
        \end{enumerate}
    
    \item \textbf{Sour cherry}
    \begin{enumerate}
        \item The \textit{number and type of buds} are crucial yield components. Sour cherry has simple flower buds that contain either flowers or leaves, and flowering terminates growth at that site \cite*{rm_02_L03_buds_bud_development}. High yields can therefore reduce bud formation for the following season due to limited shoot growth \cite*{rm_03_L04_flower_bud_formation}.
        \item Each bud produces \textit{one to five flowers per cluster}, depending on cultivar \cite*{rm_07.1_L11_13_fruit_growth_quality}.
        \item \textit{Fruit size} is genetically limited by low sink activity (the ability to attract nutrients from the tree), although minor increases occur when fruit numbers are reduced \cite*{rm_07.2_L11_13_fruit_growth_quality}.
        \item For processing cultivars, yield quality also depends on traits such as \textit{sugar content, acidity, ascorbic acid, anthocyanins, and aroma compounds}, which define fruit value for juice production \cite*{rmb_04_L21_postharvest_biology_technology_fruits_vegetables_flowers}.
    \end{enumerate}
\end{enumerate}


\subsection{Describe some conditions which may affect the development of flower buds negatively}


Flower bud development can be impaired by several interacting factors related to competition for assimilates, environmental stresses, and physiological disturbances \cite*{rm_03_L04_flower_bud_formation}. These conditions can prevent or delay flower initiation and differentiation, reducing flowering potential and yield in subsequent seasons.

\begin{enumerate}
    \item \textbf{Competition and growth imbalance} 
    \subitem A proper source-sink balance is essential for flower bud formation. When assimilates are primarily 
    \subitem used by fruits or vigorous shoots, fewer resources remain available for developing buds. 
    \begin{enumerate}
        \item High fruit load: When too many fruits develop, they compete for carbohydrates and nutrients, leaving insufficient reserves for new flower buds. This often causes alternate (biennial) bearing \cite*{rm_05.1_L07_green_harvest_fruit_thinning,rm_06_L08_growth}.
        
        \item Excessive vegetative growth: Strong shoot growth, stimulated by high nitrogen or water supply, directs assimilates away from buds and reduces flower initiation, especially in young trees \cite*{rm_02_L03_buds_bud_development,rm_03_L04_flower_bud_formation}.
        
        \item Inferior flower quiality due to vigorous growth: In strawberries, high nitrogen or excessive vigor results in fewer pistils (reproductive part of the flower) per flower and smaller trusses (a cluster of flowers), leading to reduced potential berry size \cite*{rm_03_L04_flower_bud_formation}.
    \end{enumerate}


    \item \textbf{Environmental and climatic stresses}  
    \subitem External conditions strongly affect flower initiation and bud survival.
    \begin{enumerate}
        \item Light limitation: Shading reduces photosynthesis and carbohydrate production, leading to poor or absent flower bud formation \cite*{rm_03_L04_flower_bud_formation}.
        
        \item Inadequate chilling: Insufficient exposure to cold prevents normal dormancy release, causing delayed or uneven bud break and poor flowering \cite*{rm_02_L03_buds_bud_development}.
        
        \item Frost damage: Freezing temperatures can destroy developing buds or flowers, especially when warm spells trigger early growth \cite*{rm_04_L05_model_predict_pollen_season,rm_16_L30_european_blueberry_potential_cultivation}.
        
        \item Winter warming effects: Short warm periods in late winter may prematurely activate buds, making them more sensitive to later frost and reducing total flower production \cite*{rm_16_L30_european_blueberry_potential_cultivation}.
        
        \item High temperature during initiation: Warm air temperatures stimulate shoot elongation but suppress flower initiation, leading to reduced bud numbers \cite*{rm_03_L04_flower_bud_formation}.
        
        \item Water stress: Prolonged drought interferes with bud differentiation and may cause unbalanced carbohydrate allocation between fruits and buds \cite*{rm_19_L34.1_streif_index_final_harvest_window_controlled_atmosphere_storage_apples}.
    \end{enumerate}


    \item \textbf{Physiological and developmental impairments}
    \subitem Flower bud mortality and poor differentiation also limit yield potential and long-term productivity.
    \begin{enumerate}
        \item Bud abortion: In stone fruits, many early flower primordia fail to develop or die off, especially on long shoots, resulting in fewer buds \cite*{rm_03_L04_flower_bud_formation}.
        
        \item Inferior node development rate: If shoots do not reach a critical node number before the season ends, buds remain vegetative and fail to form flowers \cite*{rm_03_L04_flower_bud_formation}.
        
        \item Early defoliation: Loss of leaves from disease or drought reduces carbohydrate storage, weakening future flowering and fruit set \cite*{rm_06_L08_growth}.
        
        \item Improper pruning timing: Pinching or pruning too early can trigger premature bud break, preventing buds from reaching the developmental stage needed for flowering \cite*{rmb_03_L20+22_planting_training_systems_pruning_fruiting_control}.
    \end{enumerate}  
\end{enumerate}


\vspace{1em}
\section{Yield and quality determinants and components}
\textbf{Flowers, pollination and fruit set (sterility and fertility)}

\subsection{Describe important factors determining fruit set?}
Fruit set is defined as the process by which a flower remains attached to the plant and begins to develop into a fruit following a series of coordinated physiological events \cite*{rmb_05_L06_physiology_temperate_zone_fruit_trees}. The success of this process depends on three main categories of factors: flower quality, pollination and fertilization effectiveness, and the plant's physiological capacity to support the developing fruit.

\begin{enumerate}
    \item \textbf{Flower quality and physiological prerequisites}
        \begin{enumerate}
            \item Flower quality:  
            Fruit set depends strongly on the physiological quality of both the flower and the very young fruit, which is influenced by the availability of stored carbohydrates \cite*{rm_07.1_L11_13_fruit_growth_quality,rm_07.2_L11_13_fruit_growth_quality}.
            \item Strong flower bud development:  
            The foundation for successful fruit set is laid in the previous growing season through the formation of strong flower buds, which require sufficient supplies of photosynthates and nitrogen \cite*{rmb_05_L06_physiology_temperate_zone_fruit_trees}.
        \end{enumerate}

    \item \textbf{Pollination and fertilization success}
        \begin{enumerate}
            \item Pollen transfer:  
            Fruit set begins with the transfer of viable pollen to a receptive stigma (part of the pistil), either naturally or through artificial pollination \cite*{rmb_05_L06_physiology_temperate_zone_fruit_trees}.
            \item Pollen germination and growth:  
            Following pollination, pollen must germinate and grow through the style to reach the mature embryo sac (the female reproductive structure inside the ovule) at the base of the pistil \cite*{rmb_05_L06_physiology_temperate_zone_fruit_trees}.
            \item Effective Pollination Period (EPP):  
            Fertilization must occur within the EPP, defined as the ovule's lifespan minus the time needed for the pollen tube to reach it. Pollination beyond this period prevents fertilisation \cite*{rmb_05_L06_physiology_temperate_zone_fruit_trees}.
            \item Compatibility and pollinators:  
            Most temperate fruit trees are self-incompatible and rely on cross-pollination for fertile seed production. Cross-pollination, even in self-fruitful types, improves fruit size and yield, depending on the variety's genetics, pollen viability, and the activity of bee pollinators \cite*{rmb_05_L06_physiology_temperate_zone_fruit_trees}.
            \item Temperature effects:  
            Temperature regulates both pollen germination and tube growth, with strong variety-specific differences in temperature tolerance, especially among plums \cite*{rmb_05_L06_physiology_temperate_zone_fruit_trees}.
            \item Seed development:  
            In multi-seeded fruits such as strawberry and Ribes, successful seed (achene) formation is essential for fruit set, as seed development stimulates surrounding tissue growth \cite*{rm_07.2_L11_13_fruit_growth_quality}.
        \end{enumerate}

    \item \textbf{Resource availability and competition}
        \begin{enumerate}
            \item Photosynthate supply:  
            After fertilization, developing fruits act as strong sinks requiring a high supply of photosynthates to sustain growth \cite*{rmb_05_L06_physiology_temperate_zone_fruit_trees}.
            \item Source-sink dynamics:  
            The balance between photosynthetic production (source) and fruit demand (sink) determines whether fruit set is maintained or aborted. Poor flower quality, limited pollination, suboptimal temperature, or low assimilate availability result in early fruit drop \cite*{rm_20_L29_source_sink_status_flowering_everbearing_strawberry,rmb_05_L06_physiology_temperate_zone_fruit_trees}.
            \item Hormonal regulation:  
            Fruit set and retention are also controlled hormonally; auxin applications such as NAA (a synthetic plant hormone) can enhance fruit set or prevent natural fruit drop if applied outside the thinning window \cite*{rm_19_L34.1_streif_index_final_harvest_window_controlled_atmosphere_storage_apples}.
        \end{enumerate}
\end{enumerate}


\subsection{What is the importance of EPP?}
The Effective Pollination Period (EPP) defines the limited window during which successful fertilization can occur and is therefore a critical determinant of fruit set \cite*{rmb_05_L06_physiology_temperate_zone_fruit_trees}. Fertilization must take place within this time frame to ensure ovule viability and the subsequent development of fruit.

\begin{enumerate}
    \item \textbf{Requirement for fertilization}  
    The EPP represents the period in which the ovule remains fertile and capable of being fertilized. It is calculated as the longevity of the ovule minus the time needed for the pollen tube to reach the embryo sac \cite*{rmb_05_L06_physiology_temperate_zone_fruit_trees}.
    
    \item \textbf{Preventing ovule degeneration}  
    Pollination after the EPP has elapsed results in ovule degeneration, meaning fertilization cannot occur even if viable pollen reaches the stigma. The EPP is essential because the mature ovule has an inherently short lifespan \cite*{rmb_05_L06_physiology_temperate_zone_fruit_trees}.
    
    \item \textbf{Indicator of reproductive success}  
    In diploid cultivars, ovule differentiation typically coincides with flower opening, initiating a strict and limited period for fertilization. The EPP thus serves as a measure of the reproductive efficiency of the plant \cite*{rmb_05_L06_physiology_temperate_zone_fruit_trees}.
    
    \item \textbf{Modifying factors}  
    The duration of the EPP is influenced by several factors, including temperature, variety, and nutrient management. Late summer nitrogen applications, for instance, can extend ovule longevity and thus prolong the EPP \cite*{rmb_05_L06_physiology_temperate_zone_fruit_trees}.
\end{enumerate}

\vspace{0.5em}
In essence, the EPP functions as a biological deadline for successful fertilization. If missed, the ovule aborts, preventing the establishment and retention of the young fruit \cite*{rmb_05_L06_physiology_temperate_zone_fruit_trees}.

\subsection{What are important quality parameters for pollen and flowers?}
\subsection{Why and how do we use pollinators?}
\subsection{Are insects (fx bees) needed in pollination of self-pollinating crops?}


\vspace{1em}
\section{Fruit development}
\textbf{Fruit development of small and large fruited species}

\subsection{Describe the general developmental phases in fruit development}
\subsection{Which sugars and acids are important in fruit development and how do they develop during fruit development? Example of species differences.}
\subsection{Which sugars are transported in the plant?}
\subsection{What is the role of starch in the carbon balance of an apple tree and an apple fruit?}


\vspace{1em}
\section{Light use, vigor control and canopy management}
\textbf{Canopy management (pruning, growing systems, light use)}

\subsection{Why do we manipulate the canopy structure in most fruit crops?}
\subsection{Describe the pruning response during the year. Why do we get differences in the growth response to pruning?}
\subsection{How does pruning affect fruit development and quality? (direct and indirect)}
\subsection{Characterise important factors (except from time in the year), which may influence the growth response to pruning?}


\vspace{1em}
\section{Crop load and canopy management}
\textbf{Carbon allocation (source-sink, fruit/leaf)}

\subsection{How does a high fruit load influence photosynthesis and transpiration?}
\subsection{Explain the concept of source strength and sink strength}
\subsection{How do source-sink relationships develop during the season in an apple tree?}
\subsection{Why may some leaves be more important than others for fruit development?}
\subsection{Why do premature fruit drop occur?}


\vspace{1em}
\section{Crop load management, fruit quality and vigor control}
\textbf{Thinning of fruits, how, why, when and effects}

\subsection{Give an example of a crop in which crop load has a strong impact on fruit development - and one where it does not.}
\subsection{Characterize the effects of fruit thinning on growth and development}
\subsection{When is it most optimal to perform fruit thinning? Why?}
\subsection{Explain why the optimal thinning strategy may dependent on the end use of the fruits.}
\subsection{Why do we not want fruits on a young tree the first year(s) after planting?}


\vspace{1em}
\section{Preharvest factor management and quality}
\textbf{Use and management of nutrients}

\subsection{Characterise the differences in nutrient requirements of a vegetative growing and a fruiting plant?}
\subsection{Calcium is important for fruit quality. Why? - And why is the level of calcium low in many fruits, especially big fruits?}
\subsection{When and why are fertilizers often sprayed on the leaves and fruits in the production of apples?}
\subsection{Characterize the importance of potassium for fruit development}


\vspace{1em}
\section{Preharvest factor management and quality}
\textbf{Effects of nutrients on yield and quality}

\subsection{Describe the effects of nitrogen status on plant development}
\subsection{In which ways do nitrogen levels influence the yield components?}
\subsection{Impacts of nitrogen levels on fruit quality?}


\vspace{1em}
\section{Preharvest factor management and quality}
\textbf{Effects of stresses on yield and quality}

\subsection{Describe the effects of stresses of nutrients and water on fruit development and quality.}
\subsection{Why are deficiency symptoms by some nutrients seen in the young leaves and by others in the old?}
\subsection{Describe how water stress can be used as a tool for growth control.}

\vspace{1em}
\section*{Questions within: Fruit quality, maturity and usability aspects}
\section{Fruit development}
\textbf{Influencing factors}

\subsection{Describe some important factors for optimal fruit development in small and large fruited species. Are there differences?}
\subsection{What would you do to optimize fruit development and fruit quality in an apple crop?}
\subsection{What is important for fruit development and quality in raspberry and strawberry?}


\vspace{1em}
\section{Fruit maturity, harvest and quality assessment}
\textbf{Maturity measures, Harvest time and methods}

\subsection{How would you determine the optimal harvest time in apple?}
\subsection{Describe the problems and quality effects you might get, if you harvest either too early or too late.}
\subsection{Hand picking vs mechanical harvest - problems and benefits?}
\subsection{What are the main reasons for post harvest losses and what may be done to minimize it?}


\vspace{1em}
\section{Fruit maturity, cultivar variations and important quality parameters}
\textbf{Aromas in fruits and effects on aroma development}

\subsection{When do aromas develop in fruits?}
\subsection{Characterise some important aroma substances and changes in aroma with maturity}
\subsection{Characterize the importance of harvest time on aroma development}
\subsection{What might affect aroma development pre and post harvest?}


\vspace{1em}
\section{Fruit maturity, cultivar variations and important quality parameters}
\textbf{Colors in fruit and berries and effects on colour development}

\subsection{Characterise some important colour substances in fruits and berries}
\subsection{How does colour change with maturity?}
\subsection{What might affect colour development pre and post harvest?}
\subsection{What is the mechanism behind the occurrence of red clones in fruit cultivars (fx apples, pears and grapes)?}


\vspace{1em}
\section{Cultivar variations and important quality parameters (fresh use and juice)} 
\textbf{Cultivar characterization and uses. Fruit composition and human health}

\subsection{Characterise some of the most important (internal and external) quality characters, which may vary among cultivars in a fruit crop.(Fxstrawberries or apple)}
\subsection{Which compounds are considered especially important in fruit and berries for human health and where are they located? }
\subsection{Which species are believed to be especially healthy to eat? Comment on the consumption of raw or processed fruits and berries.}


\vspace{1em}
\section{Cultivar variations and important quality parameters (fresh use and juice)}
\textbf{Juice processing and juice quality}

\subsection{How does the level of fruit ripening impact on juice processing and juice quality?}
\subsection{Which enzymes may be used in juice processing and why?}
\subsection{Comment on the effects of different juice processing steps on juice quality.}
\subsection{Why are juices pasteurised, and what are important factors for a successful pasteurisation?}


\vspace{1em}
\section{Potentials for producing fruit and berry wines }
\textbf{Challenges and opportunities}

\subsection{Comment on the challenges and potentials in making fruit wine from different fruit and berries}
\subsection{High levels of acidity may be a problem. How may it be handled?}
\subsection{Characteristics of so called ‘cider apple cultivars'}
\subsection{Comment on the importance of ripening levels of fruit and berries for wine making}
\subsection{Characterize the process of cryo-concentration and the impacts on the juice quality and the potential for wine style development}


\vspace{1em}
\section{Domestication of wild berries} 
\textbf{Challenges and opportunities}

\subsection{Why may wild berries be attractive to domesticate?}
\subsection{Comment on some major challenges/barriers.}
\subsection{Describe important yield and quality components in wild/European blueberries.}
\subsection{Blueberries are one of few fruiting plants adapted to low pH soils. Comment on the challenges it causes in growing the plants.}
\subsection{Comment on the importance/impacts of propagation method in European blueberries. European blueberries.}

