\chapter{Exam Questions and Answers}
\setlength{\headheight}{12.71342pt}
\addtolength{\topmargin}{-0.71342pt}

This chapter of the course notes compiles the exam questions for the course held in November 2025, along with their respective answers prepared by me. The purpose of this section is twofold: firstly, to provide a reflective exercise that consolidates understanding of the course material; and secondly, to document my comprehension of the course topics as assessed through the exam questions.

\vspace{1em}
To ensure citation accuracy and academic transparency, NotebookLM has been employed as the primary generative AI platform. Its use has focused on verifying that all citations accurately reference the uploaded course materials and lecture slides provided by the professors. Beyond citation control, this section also represents an ongoing exploration of prompt engineering — refining interaction design to optimise AI output quality, precision, and academic reliability. Through this approach, the work aims to maintain a high academic standard while enhancing clarity, structure, and depth in written responses.

\vspace{1em}
There are a total of 17 questions in the exam, each comprising between three and five sub-questions. The numbering of the sections in this chapter corresponds directly to the numbering of the exam questions, ensuring a clear and consistent structure throughout. Questions 1-9 address aspects related to crop physiology, while questions 10-17 focus on fruit quality, maturity, and usability. Each question is presented below, followed by its respective sub-questions and answers.

\section*{Questions within: Crop Physiology aspects}

\section{Yield and quality determinants and components}
\textbf{Shoot and bud development, growth and flower bud development}

\subsection{Characterise the development and importance of spurs and extension 
(long) shoots}
\subsection{Describe differences in bud development and structure between stone 
and pome fruits}
\subsection{Describe some important yield components in strawberry and in sour 
cherry}
\subsection{Describe some conditions which may affect the development of 
flower buds negatively}


\vspace{1em}
\section{Yield and quality determinants and components}
\textbf{Flowers, pollination and fruit set (sterility and fertility)}

\subsection{Describe important factors determining fruit set? }
\subsection{What is the importance of EPP?}
\subsection{What are important quality parameters for pollen and flowers?}
\subsection{Why and how do we use pollinators?}
\subsection{Are insects (fx bees) needed in pollination of self-pollinating crops?}


\vspace{1em}
\section{Fruit development}
\textbf{Fruit development of small and large fruited species}

\subsection{Describe the general developmental phases in fruit development}
\subsection{Which sugars and acids are important in fruit development and how 
do they develop during fruit development? Example of species 
differences.}
\subsection{Which sugars are transported in the plant?}
\subsection{What is the role of starch in the carbon balance of an apple tree and 
an apple fruit?}


\vspace{1em}
\section{Light use, vigor control and canopy management}
\textbf{Canopy management (pruning, growing systems, light use)}

\subsection{Why do we manipulate the canopy structure in most fruit crops?}
\subsection{Describe the pruning response during the year. Why do we get 
differences in the growth response to pruning?}
\subsection{How does pruning affect fruit development and quality? (direct and 
indirect)}
\subsection{Characterise important factors (except from time in the year), which 
may influence the growth response to pruning?}


\vspace{1em}
\section{Crop load and canopy management}
\textbf{Carbon allocation (source-sink, fruit/leaf)}

\subsection{How does a high fruit load influence photosynthesis and transpiration?}
\subsection{Explain the concept of source strength and sink strength}
\subsection{How do source-sink relationships develop during the season in an apple 
tree?}
\subsection{Why may some leaves be more important than others for fruit 
development?}
\subsection{Why do premature fruit drop occur?}


\vspace{1em}
\section{Crop load management, fruit quality and vigor control}
\textbf{Thinning of fruits, how, why, when and effects}

\subsection{Give an example of a crop in which crop load has a strong impact on fruit development - and one where it does not.}
\subsection{Characterize the effects of fruit thinning on growth and development}
\subsection{When is it most optimal to perform fruit thinning? Why?}
\subsection{Explain why the optimal thinning strategy may dependent on the end 
use of the fruits.}
\subsection{Why do we not want fruits on a young tree the first year(s) after 
planting?}


\vspace{1em}
\section{Preharvest factor management and quality}
\textbf{Use and management of nutrients}

\subsection{Characterise the differences in nutrient requirements of a vegetative 
growing and a fruiting plant?}
\subsection{Calcium is important for fruit quality. Why? - And why is the level 
of calcium low in many fruits, especially big fruits?}
\subsection{When and why are fertilizers often sprayed on the leaves and fruits in 
the production of apples?}
\subsection{Characterize the importance of potassium for fruit development}


\vspace{1em}
\section{Preharvest factor management and quality}
\textbf{Effects of nutrients on yield and quality}

\subsection{Describe the effects of nitrogen status on plant development}
\subsection{In which ways do nitrogen levels influence the yield components?}
\subsection{Impacts of nitrogen levels on fruit quality?}


\vspace{1em}
\section{Preharvest factor management and quality}
\textbf{Effects of stresses on yield and quality}

\subsection{Describe the effects of stresses of nutrients and water on fruit development and quality.}
\subsection{Why are deficiency symptoms by some nutrients seen in the young leaves and by others in the old?}
\subsection{Describe how water stress can be used as a tool for growth control.}

\vspace{1em}
\section*{Questions within: Fruit quality, maturity and usability aspects}
\section{Fruit development}
\textbf{Influencing factors}

\subsection{Describe some important factors for optimal fruit development in small and large fruited species. Are there differences?}
\subsection{What would you do to optimize fruit development and fruit quality in an apple crop?}
\subsection{What is important for fruit development and quality in raspberry and strawberry?}


\vspace{1em}
\section{Fruit maturity, harvest and quality assessment}
\textbf{Maturity measures, Harvest time and methods}

\subsection{How would you determine the optimal harvest time in apple?}
\subsection{Describe the problems and quality effects you might get, if you harvest either too early or too late.}
\subsection{Hand picking vs mechanical harvest - problems and benefits?}
\subsection{What are the main reasons for post harvest losses and what may be done to minimize it?}


\vspace{1em}
\section{Fruit maturity, cultivar variations and important quality parameters}
\textbf{Aromas in fruits and effects on aroma development}

\subsection{When do aromas develop in fruits?}
\subsection{Characterise some important aroma substances and changes in aroma with maturity}
\subsection{Characterize the importance of harvest time on aroma development}
\subsection{What might affect aroma development pre and post harvest?}


\vspace{1em}
\section{Fruit maturity, cultivar variations and important quality parameters}
\textbf{Colors in fruit and berries and effects on colour development}

\subsection{Characterise some important colour substances in fruits and berries}
\subsection{How does colour change with maturity?}
\subsection{What might affect colour development pre and post harvest?}
\subsection{What is the mechanism behind the occurrence of red clones in fruit cultivars (fx apples, pears and grapes)?}


\vspace{1em}
\section{Cultivar variations and important quality parameters (fresh use and juice)} 
\textbf{Cultivar characterization and uses. Fruit composition and human health}

\subsection{Characterise some of the most important (internal and external) quality characters, which may vary among cultivars in a fruit crop. (Fx strawberries or apple)}
\subsection{Which compounds are considered especially important in fruit and berries for human health and where are they located? }
\subsection{Which species are believed to be especially healthy to eat? Comment on the consumption of raw or processed fruits and berries.}


\vspace{1em}
\section{Cultivar variations and important quality parameters (fresh use and juice)}
\textbf{Juice processing and juice quality}

\subsection{How does the level of fruit ripening impact on juice processing and juice quality?}
\subsection{Which enzymes may be used in juice processing and why?}
\subsection{Comment on the effects of different juice processing steps on juice quality.}
\subsection{Why are juices pasteurised, and what are important factors for a successful pasteurisation?}


\vspace{1em}
\section{Potentials for producing fruit and berry wines }
\textbf{Challenges and opportunities}

\subsection{Comment on the challenges and potentials in making fruit wine from different fruit and berries}
\subsection{High levels of acidity may be a problem. How may it be handled?}
\subsection{Characteristics of so called ‘cider apple cultivars’}
\subsection{Comment on the importance of ripening levels of fruit and berries for wine making}
\subsection{Characterize the process of cryo-concentration and the impacts on the juice quality and the potential for wine style development}


\vspace{1em}
\section{Domestication of wild berries} 
\textbf{Challenges and opportunities}

\subsection{Why may wild berries be attractive to domesticate?}
\subsection{Comment on some major challenges/barriers.}
\subsection{Describe important yield and quality components in wild/European blueberries.}
\subsection{Blueberries are one of few fruiting plants adapted to low pH soils. Comment on the challenges it causes in growing the plants.}
\subsection{Comment on the importance/impacts of propagation method in European blueberries. European blueberries.}
