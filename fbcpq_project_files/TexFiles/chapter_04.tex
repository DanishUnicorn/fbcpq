\chapter{Literature résumés}
\setlength{\headheight}{12.71342pt}
\addtolength{\topmargin}{-0.71342pt}

This section of the course notes is designed to streamline access to the key findings from each reading material (RM), providing a concise and accessible overview of essential information. Created through experimentation with various AI platforms, this chapter also serves to enhance prompt engineering skills, exploring diverse methods of note-taking for maximum efficiency and clarity. The procedures for creating these summaries have varied, but all methods share a common approach: each RM has been fully read, with summaries and notes prepared after completing each respective subsection. By using these AI-co-op'ed approaches, these notes aim to be both a reliable reference and a resource for continuous improvement in capturing complex microbiology concepts.

\section{RM for L02 - Fruit Yield and Quality. Components and Determinants}

Fruit cultivation will normally be intended to maximize fruit yield and fruit quality \cite*{rm_01_L02_fruit_yield_quality}. The available tools for achieving this are variety selection (genetic selection) and choice of cultivation techniques, the physiological basis of which is treated in subsequent chapters. Maximizing fruit yield, often expressed as tonnes (or hekto litre in case of wine or juice) per hectares (ha), requires no further definition. Fruit yield is ultimately determined by fruit number per hectares and fruit size, understood as a product of several components. While component weights vary across fruit species, the main components can be summarized as shown in Table 1 and illustrated for cherry in Figure 1 \cite*{rm_01_L02_fruit_yield_quality}. Determinants define the amount or degree of development of each component within the genetic potential. Plant number per ha is primarily determined by the growing technique, mainly the selected planting system. The cultivation system may also determine factors such as the number of shots in raspberries and the number of rooted runners in strawberries. Other determinants are primarily physiological factors associated with growth and development, although they may also be influenced to some extent by growing technical interventions. The first main section of the course includes a review of these physiological determinants, serving as a basis for explaining and understanding the effects of different cultivation technical interventions.

\vspace{0.5em}
Maximizing fruit quality is also a primary target in growing, although fruit quality cannot be defined nearly as unambiguous as fruit yield. Fruit quality includes a variety of components that vary with fruit type and fruit use, involving different weightings in different contexts. Fruit size is an important quality component, especially for table fruit. Colour and other factors related to fruit appearance are also important, particularly those factors that are part of EU standards for the grading of the various fruit species. These components are of specific interest to fruit growers because they contribute to price differences between different grades. Components attached to fruit taste and enjoyment value—e.g., sugar, acid, and aromatics content—are also important and are expected to become even more important in the future. For industry fruit, such as sour cherries, emphasis is placed on juice colour and acidity, and for blackcurrant, on vitamin C content, despite there being no real price differences (quality premium). The concept of fruit quality can also include aspects such as the content of foreign substances like residuals from spraying of chemicals, which are of increasing importance in the complex of fruit quality. Societal priorities for health and environmental concerns, etc., also gain increasing importance in a growing technical context, and these factors are discussed in relevant, growing technical lectures.

\vspace{0.5em}
Determinants such as flower development, and particularly fruit growth and fruit development, affect the quality components. Since physiological and growing technical factors may have opposite effects on yield and fruit quality components in some cases, the fruit grower's task is to seek the best overall compromise.

\vspace{0.5em}
The components determining Yield per hectare and fruit quality, along with their primary determinants, follow a hierarchical path, as outlined in Table 1 \cite*{rm_01_L02_fruit_yield_quality}: 

\begin{enumerate} 
    \item Number of plants / ha is determined by $\leftarrow$ Planting System, and $\leftarrow$ Growing System (Rubus, strawberry). 
    \item Bud Number / plant, bud type is influenced by $\leftarrow$ Plant size, structure, $\leftarrow$ Elongation growth, shot type development, and $\leftarrow$ Bud Development. 
    \item Number of inflorescences / bud is determined by $\leftarrow$ Flower initiation. 
    \item Number of flowers/cluster is influenced by $\leftarrow$ Flower initiation and $\leftarrow$ Flower Development. 
    \item Flower quality is determined by Number of seed primordia / Flower and Position in cluster. 
    \item Initial fruit set / flower is influenced by $\leftarrow$ Pollination, fertilization, initial setting. 
    \item Final / initial fruit set is affected by $\leftarrow$ Fruit drop (June drop). 
    \item Fruit Size is determined by Number of seeds / fruits ($\leftarrow$ Pollination) and Quantity fruit flesh/seed, and is influenced by $\leftarrow$ Fruit Growth, Fruit Development. 
\end{enumerate}


\section{RM for L03 - Buds and Bud Development}

Buds are growth points which can develop into shoots, determining the plant's future extension growth (dimension), or develop flowers, thereby determining the cropping \cite*{rm_02_L03_buds_bud_development}. Bud number and bud type are important components influenced by plant size and structure, which determine the number and distribution of various shoot types, consequently influencing bud number and type at the start of the growing season. Terminal buds, and sometimes the upper side buds on last year's long shoots, may develop new annual shoots. Lateral buds are formed in the axils of the leaves as the stem grows, culminating in terminal bud formation, with this elongation growth increasing the number of lateral buds. Other lateral buds on long shoots and end buds on short shoots may remain as short shoots or spurs, completing their short terminal growth with a terminal bud. Short shoots may mature well-developed lateral buds (often developing flowers), typical of fruiting spurs on cherry. Sleeping eyes (buds) only break after a strong stimulus, such as pruning. Flower buds of apple and pear are mixed buds, containing leaves and flowers, plus a new small bud (bourse-shoot bud), meaning a flower bud is usually followed by a new growth point. Stone fruit, conversely, has 'simple' or 'naked' flower buds, containing only leaves or flowers, resulting in the loss of a growth point at that position upon flowering, which can lead to bare areas on the shoot, typical of some sour cherry cultivars.

\vspace{0.5em}
Elongation growth, and its distribution, determine bud development during the growing season. Terminal buds on short shoots are formed early (June). Lateral buds on longer annual shoots are formed in line with leaf development, with terminal bud formation occurring from June until September, depending on growth intensity. Elongation growth and bud development are subject to correlative inhibition mechanisms, including apical dominance, which involves a polar transport of growth substances (auxins, gibberellins) from terminal buds that inhibits lateral bud growth. Heavy crop load can strongly reduce shoot elongation growth and the number of lateral buds due to competition for assimilates. Terminal bud formation occurs earlier if the ratio is high, see equation \ref{eq:root_strength}:

\begin{equation}
    \text{Root activity} = 
    \dfrac{\text{Root mass} \times \text{Root activity}}
          {\text{Number of Growth Points in the Top}}
    \label{eq:root_strength}
\end{equation}

\vspace{0.5em}
Mature trees tend to have earlier terminal bud formation (possibly June-July) than young trees (often not before September). Elongation growth is enhanced by increasing temperature and by increased nitrogen and water supply (see sections 19.2.1 and 20.2.3).

\vspace{0.5em}
The formation of a terminal bud is an abrupt process where bud scales develop instead of leaves, forming a compressed shoot inside with primordials for transitional leaves, true leaves, and possibly flowers. Bud break in the same growing season (re-growth) may occur exceptionally where correlative inhibitions are broken sharply, such as by removing shoot tips or performing early summer pruning. This is only possible until a certain date; with apple, for example, up to approximately 1$^{st}$ August, after which buds enter endo-dormancy.

\vspace{0.5em}
Buds enter the endo-dormant phase when they can no longer break due to strong influences on correlative inhibitions. Bud dormancy is lifted by a species- and variety-specific period of low temperature, termed 'chilling'. Efficient chilling temperature is usually calculated as temperatures below +7\textdegree. In the temperate climate zone, the buds go through a physiological period of dormancy where there is no visible growth, which, if not met, can cause poor development or bud fall. The duration of dormancy varies by species (Table 1-2 shows Black currant ending dormancy in December, Apple in March). The depth and length of dormancy can be influenced by growth conditions in summer and geographical location, with cool summers accelerating termination. Dormancy is controlled by growth regulating substances, with Abscisic acid (ABA) having the strongest relationship with dormancy, while Gibberellins (GA) and/or cytokinins are growth-promoting substances. In areas with too little chilling, cultivation can be enabled by initiating the process with drought, followed by artificial defoliation (e.g., spraying with 10\% urea) and then applying a spray to interrupt dormancy, such as 0.1\% active DNOC +4\% emulsified crude oil at the first watering.

\vspace{0.5em}
Woody plants propagated by seeds must reach a certain age before flowering, termed the juvenile period or phase. This period may be short (e.g., Rubus and Ribe species), or long (e.g., 4-5 years for apple, 6-8 years for pear). The period between the end of the juvenile phase and first natural flowering is the transitional period. The juvenile period is characterized by different leaf forms and, sometimes, the transformation of short shots into woody thorns (e.g., pear). To shorten the juvenile period in apples, experiments have reduced the period to approximately a third of normal using continuous growth in greenhouse conditions with increased day length and CO$_2$ addition.


\section{RM for L04 - Chapter 2: Flower bud formation, flower development and flowering}

A proportion of the buds formed during the growing season develop into flower buds, a process that is a very important yield component \cite*{rm_03_L04_flower_bud_formation}. Flower bud formation begins with flower initiation, which occurs when there are no inhibitory effects. The time of initiation is typically summer - late summer, occurring in the approximate order of: Ribes, stone fruit, pome fruit, strawberry, and Rubus. This initiation is followed by flower differentiation.

\vspace{0.5em}
Flower bud formation is most detailed studied in apple (pome fruit), primarily because it is an important crop and because it is frequently a limiting factor, especially concerning alternation (every two-year cropping). Studies show that a bud must undergo development before initiation can occur. For the apple variety 'Cox's Orange', a critical node number of about 20 must be formed inside the bud. If the total number of nodes reaches 20 or more, a flower bud is formed, with a flower primordial in the terminal position and others in the corners of bracts (usually six to seven flower buds in total). The earlier a bud is formed (like spur buds), the longer time it has to develop, making flower formation easier. If node development is too rapid, the bud may break again in the same year (re-growth), causing delays that inhibit flower formation. Microscopically, the first sign of initiation is an increase in width of the growing point from 12-15 to 18-25 cells. Different theories exist regarding the mechanism, often involving positive and negative-acting factors, such as the Carbon/Nitrogen (C/N) ratio, where starch accumulation promotes flower bud formation. Flower bud formation may be associated with a high auxin / gibberellin content. The Source / sink ratio and flower bud formation are positively correlated. A strong negative correlation between fruit number per tree (fruit / leaf ratio) and flower bud formation explains the phenomenon of alternation. Reduced fruit growth rate, resulting in smaller fruits, leads to very large flower bud formation. Increased photosynthetic activity due to CO$_2$ addition is accompanied by an increase in flower bud formation. Increased shoot growth generally correlates negatively with flower bud formation. However, on young trees, recent studies suggest that a positive correlation can be found between shoot growth and flower bud formation on annual shoots, particularly when nutrients and water are supplied by drip irrigation. This allows the buds on annual shoots to reach their lower critical node number (14-16).

\vspace{0.5em}
In stone fruit, alternation is typically less pronounced than in apple, possibly due to fruit size-related sink strength. In 'Stevnsbær', most buds on short shoots bloom, and the negative correlation between shoot growth and flower bud formation is less strong, primarily affecting longer annual shoots. Flower initiation may depend on the rate of development rather than a critical node number.

\vspace{0.5em}
In Ribes, flower bud formation requires relatively short days (<12-14 hours) and low temperature. There is no direct effect of fruit number on flower bud formation; however, good cropping reduces shoot growth, leading to fewer buds and thus fewer flowers per bush the following year.

\vspace{0.5em}
In strawberry (short-day types), day length is the crucial factor, with initiation occurring under Danish conditions when day length falls below 11 to 14 hours. Low temperature also promotes initiation. Vegetative growth (e.g., runner formation) is promoted by long days and high temperatures.

\vspace{0.5em}
Flower development determines flower quality, which relates to the ability of the flower to set fruit. The differentiation of flower parts follows a sequence: 

\begin{enumerate} 
    \item Sepals and petals differentiate rather quickly after initiation. 
    \item Stamens and carpels differentiate next. 
    \item Anthers (pollen sacks) and ovules often only differentiate next spring. 
\end{enumerate} 

\vspace{0.5em}
In cherry ('Stevnsbær'), bud death and abortion of flower primordiums can cause mortality rates between 30-80\% of initial primordiums. In strawberries, the inflorescence is a cyme. With increasing flower order (primary, secondary, etc.), development becomes less complete, leading to inferior quality, characterized mainly by a decreasing number of pistils (Achens) per flower.

\vspace{0.5em}
Once bud dormancy is over, flowering time is mainly determined by temperature. The duration of the flowering process is also controlled by temperature; hot conditions may reduce it to one week, while cooler years may extend it beyond two weeks.


\section{RM for L05 - A model to predict the beginning of the pollen season}

The aim of the present study was to test phenoclimatographic models, comprising the Utah phenoclimatography Chill Unit (CU) and ASYMCUR-Growing Degree Hour (GDH) sub-models, on the allergenic trees Alnus, Ulmus, and Betula in order to provide a method to predict the beginning of the pollen season \cite*{rm_04_L05_model_predict_pollen_season}. This type of model relates environmental temperatures to rest completion and bud development. Flowering is a phenological event resulting from a long period of development, beginning with the initiation and differentiation of buds into flower and vegetative buds during the summer. Falling temperatures cause a gradual change into a phase of winter rest with little or no growth activity. After a period, which length apparently depends on the climate and plant species, the plant gradually reverts to a phase of active growth in the spring.

\vspace{0.5em}
As the phenologic parameter, 14 years of pollen counts (1977 to 1990) from a Burkard Volumetric Spore Trap in Copenhagen were used. The observed dates for the beginning of the pollen seasons were defined from the pollen counts as a fixed percentage (2.5\%) of total counts. The observed dates of first bloom and pollen counts deviate profoundly from one year to another, with Alnus deviating from the 30$^{th}$ December to the 1$^{st}$ April, Ulmus from the 21$^{th}$ February to the 2$^{nd}$ May, and Betula from 2$^{nd}$ April to the 9$^{th}$ May. The chilling requirement was expressed using Chill Units (CU), where one CU is defined as one hour at 6\textdegree C, the optimum Chill Unit temperature for fruit trees. CU calculations began at the first day yielding positive chilling values. The experimentally determined temperatures for the GDH function included an optimum temperature of 25\textdegree C and a critical temperature of 36\textdegree C. The base temperature for GDH was changed from 4\textdegree C to 2\textdegree C, giving a slightly better correlation.

\vspace{0.5em}
The models used were: 

\begin{enumerate} 
    \item A fixed day model, using only the GDH model with 1$^{st}$ January as fixed initiation point.
    \item A CU/GDH model, with a fixed sum of Chill Unit requirement as the initiation point for subsequent GDH accumulation. 
    \item A dynamic CU/GDH model, based on a dynamic relationship between CU and GDH. 
\end{enumerate}

\vspace{0.5em}
In the ordinary CU/GDH model, the statistically estimated requirements for the trees were: Alnus: CU = 1550, GDH = 200; Ulmus: CU = 1850, GDH = 700; and Betula: CU = 1900, GDH = 2446. The minimum standard deviation for Betula was obtained at a CU estimate of 1900, giving an average of 2446 GDH.
The dynamic CU/GDH model, equation \ref{eq:dynamic_gdh}, adjusted the necessary GDH to budbreak according to the CU obtained, proposing a non-linear S-curve function: 

\begin{equation}
    \text{GDH} = C + \dfrac{(D - C)} \cdot {1 + \exp\left(\dfrac{CU - CU_0}{A}\right)}
    \label{eq:dynamic_gdh}
\end{equation}

\vspace{0.5em}
When compared with observed dates, the dynamic model provided the best predictions: within 2-4 days for Alnus, 8-10 days for Ulmus, and 3-5 days for Betula. This deviation size is acceptable given the large temporal variation from year to year.
It is concluded that the CU and GDH relationships defined for fruit trees are generally applicable and give a reasonable description of the growth processes of other trees. Alnus tends to be completely regulated by temperature, but other parameters like the photoperiod apparently must be involved for Ulmus and Betula if predictions are to be further improved. The involvement of a dynamic relationship, though complicating the model, tends to improve the simulation of gradual changes during dormancy and budbreak. Furthermore, the results indicate that frost damage, detected using an LT$_{50}$ submodel (e.g., in 1981 and 1988), might be an important factor that can strongly affect total pollen counts.


\section{RM for L06 - Physiology of Temperate Zone Fruit Trees}

Fruit set plays an important role in modern fruit production, as a large yield of fruit is only expected if conditions for pollination and fruit set are favorable. Fruit set is a complex series of physiological events where no single element can be limiting to the overall process \cite*{rmb_01_physiology_temperate_zone_fruit_trees}. This overall process begins with the transfer of viable pollen to a receptive stigma, followed by germination, pollen tube growth, nutrient supply, and the successful fertilization of the mature embryo sac, which must occur before subsequent growth of the embryo. The events involved in fruit set are grouped into three main categories: the flowering process itself; circumstances that influence pollination and fertilization; and conditions conducive to fruit set without fertilization. All processes connected with fertilization are influenced by regulators, nutrition, and rootstock type, all of which are controllable by the grower \cite*{rmb_01_physiology_temperate_zone_fruit_trees}.

\vspace{0.5em}
The biology of bloom describes the process in which sepals and petals slowly enlarge, and the stigma(s) and stamens are exposed \cite*{rmb_01_physiology_temperate_zone_fruit_trees}. Bloom is considered to begin when 12-15\% of the flowers are open and ends when 95-100\% of the flowers are open. Cultivars generally do not bloom at the same time, and ensuring effective pollinizers requires that the bloom period of the main variety and the pollinizer overlap sufficiently. The timing of flowering varies from year to year, regardless of the absolute date, due to the differential sensitivity of various stages of bloom development to temperature stimuli \cite*{rmb_01_physiology_temperate_zone_fruit_trees}.

\vspace{0.5em}
The majority of fruit trees are self-incompatible, defined by the inability of fertile hermaphrodite seed plants to produce zygotes after self-pollination. Self-pollen is usually genetically inhibited from germination or pollen tube growth, preventing fertilization. However, even in self-fruitful cultivars, cross-pollination usually sets heavier crops. Triploid cultivars contain more pollen, which is generally less viable and has less value as a pollinizer \cite*{rmb_01_physiology_temperate_zone_fruit_trees}. Temperature has a great influence on the amount of pollen produced, with long cold winters or cold temperatures in early spring leading to a reduction in the number and vitality of pollen grains \cite*{rmb_01_physiology_temperate_zone_fruit_trees}. Bees are considered the most important pollinator insect \cite*{rmb_01_physiology_temperate_zone_fruit_trees}.

\vspace{0.5em}
The concept of the Effective Pollination Period (EPP), introduced by Williams in 1966, describes the limited period immediately following flower opening during which fertilization is possible \cite*{rmb_01_physiology_temperate_zone_fruit_trees}. The EPP duration equals the longevity of the ovule minus the time required for pollen tubes to reach the embryo sac. Ovule longevity varies significantly by cultivar; for example, the ovule longevity of 'Delicious' apple was estimated at 5 days, compared with 7-8 days for 'Jonathan'. The growth rate of compatible or partially compatible pollen is dictated by temperature. Pollen tube growth responds to temperature in a linear fashion, and a temperature response index can estimate the time required for pollen tube growth, compounded based on daily mean temperatures \cite*{rmb_01_physiology_temperate_zone_fruit_trees}.

\vspace{0.5em}
For a good fruit set, three requirements are sequential: first, the development of a strong flower bud during the previous fall, requiring sufficient photosynthate and nitrogen supply; second, a certain temperature range during and soon after bloom to ensure good pollination, pollen tube growth, and fertilization; and third, a relatively high photosynthate supply for the young developing fruit after fertilization \cite*{rmb_01_physiology_temperate_zone_fruit_trees}. Failure to satisfy these factors results in a poor fruit set and early drop of the young fruit. Early shedding of fruit is a regular feature of fruit set, often appearing in four waves at 12-14 day intervals. The third and fourth drops, called the 'June drop', are more conspicuous due to the larger size of the dropping fruit and involve a complete abscission process that includes the formation of ethylene.

\vspace{0.5em}
Parthenocarpic fruit set, where fruit sets and grows to full size without fertilization, is most widespread in pears among temperate zone fruits. Pear varieties can be classified into four groups based on their ability to set parthenocarpic fruit \cite*{rmb_01_physiology_temperate_zone_fruit_trees}: 

\begin{enumerate} 
    \item Not parthenocarpic: 'Pap Pear'. 
    \item Parthenocarpic fruit is shed at the time of June drop: 'Hardy'. 
    \item Variably parthenocarpic: 'Bosc', 'Clapp Favorit', 'Diel', 'Madame du Pois', 'Oliver Serres', 'Bartlett'. 
    \item Consistently parthenocarpic: 'Arabitka', 'Hardenpont', 'Passe Crassane', 'Pringall'. 
\end{enumerate} 

\vspace{0.5em}
In most fruit, the seed is the source of Gibberellic acid (GA), which is sensitive to temperature, explaining the dependence of parthenocarpy on high temperature around bloom \cite*{rmb_01_physiology_temperate_zone_fruit_trees}.



\section{\texorpdfstring{1\textsuperscript{st} RM for L07 - Physiology of Temperate Zone Fruit Trees}{1st RM for L07 - Physiology of Temperate Zone Fruit Trees}}

The utilization of plant reactions to changes in the fruit/leaf ratio requires consideration of the growth phase of the fruits, which varies between cultivars. Vigor, growth form, and cluster structure are also factors to be taken into account.

In fruit science, an organ with the status of an importer, such as strong growing shoot tips or developing fruit, is termed a "\textbf{sink}," importing dry matter and energy. Conversely, exporting organs, such as fully developed leaves that produce more sugar through photosynthesis than they use, are termed a "\textbf{source}". The surplus sugar is exported to competing sinks throughout the plant. The main source capacity covering the grapes' import/sink needs is located in the bottom 2/3 of the foliage wall.

The development of berries is roughly divided into 3 phases \cite*{rm_05.1_L07_green_harvest_fruit_thinning}:

The effects of affecting the fruit/leaf ratio must take into account the growth phase of the fruits, which are roughly divided into 3 phases characterized by different developmental processes. Phase 1 is dominated by cellular divisions and strong berry growth, quickly increasing berry size. This phase determines both the number of berries per cluster (based on the percentage of fruit set) and the size potential of the berries; increased cell divisions result in larger potential berry size. Reducing the number of bunches during Phase 1 decreases the fruit/leaf ratio, thereby increasing available resources, including carbohydrates (from leaves) and minerals (root intake). The result is an increase in both the percentage of fruit set and the number of cells per fruit, with the effect being stronger the earlier the action is taken. Early fruit (cluster) thinning can increase fruit set if the prospect for fruit set is less than desired. Simultaneously, there is a "risk" that the size of the berries increases, potentially resulting in too dense clusters.

\vspace{0.5em}
The availability of resources for fruit development can be supported or enhanced by tipping shoots that have reached the desired height on the foliage wall. Strong growing shoot tips are defined as "sinks" because they import dry matter and energy, as their new young leaves are unable to cover their own needs. Conversely, exporting organs, such as fully developed leaves that produce surplus sugar in photosynthesis, are "sources". When the shoot tip is removed, a competing sink is eliminated, and the plant sends more sugar resources downward to cluster development, which also has sink status. Tipping also eliminates the apical dominance of the shoot, encouraging side shoot growth, although a period exists when the resource surplus primarily benefits the development of young/new fruit clusters. In wine grape production, the goal is often concentrated grapes and a relatively large skin/fruit ratio (i.e., small berry size), so early cluster thinning is usually avoided.

\vspace{0.5em}
For cultivars such as 'Solaris,' utilizing the ability of fruit growth activity to limit shoot growth is often preferred. Early fruit thinning in Phase 1 is generally done only if the shoot growth is less than desired (e.g., on young plants). If growth is strong, maximum fruit load is kept, and the amount of leaves may be reduced. Uneven growth rates can be evened by tipping vigorous shoots when they reach the desired height on the foliage wall, moving resources to strengthen weak shoots. Phase 1 typically occurs for approximately 5 weeks, from 'Sct. Hans' until the end of July.

\vspace{0.5em}
Phase 2 involves decreasing cell division, dominant seed development, and acid levels reaching their maximum. Xylem transport ability decreases, and the plant/fruit development is most robust to stress influences, such as drought stress. If the fruit load is clearly higher than optimal for Phase 3 (maturation), a first coarse thinning (e.g., down to 2 clusters per shot) may be appropriate. Bunch size reductions can be considered in high-setting, high-risk varieties like 'Johanniter' or Pinot's, or in large-clustered cv's such as 'Bolero' and possibly 'Cabernet Cortis,' if the clusters have not yet closed up. This period is typically around the start to mid-August. Continuous trimming of significant summer side shoot growth is important. Cutting off a dense cluster tip may allow the grapes ample space, and it may be better to keep 2 such adjusted bunches per shot instead of 1 large and dense cluster.

\vspace{0.5em}
Phase 3 is marked by color shift, progressing from green to more yellow shades. Berries grow in size through cellular expansion, resulting in decreased firmness and a doubling of grape size. If a first rough adjustment was not made in Phase 2, it can be done early in Phase 3. Managing this phase is an act of balance; weaker growth requires greater thinning, while stronger growth requires delaying thinning to use the fruit to slow growth. As foliage becomes denser, leaf thinning begins, focused on the grape zone, and side shoots are heavily controlled. Thinning later than 3 weeks before harvest will reduce the effect on sugar accumulation (higher brix value). The final fruit/leaf ratio should normally be established by the first week of September.

\vspace{0.5em}
For varieties susceptible to bunch stem-necrosis, such as 'Rondo' and 'Cabernet Cortis,' gradual adjustments are particularly important to avoid/reduce shock treatments. Leaf removal usually begins early, focusing on inner leaves in the grape zone to create an "inner open box" while preserving the outer leaves as important sugar producers and for protection. Leaves are first removed on the east side in August to stimulate early morning drying. Gradually, the west side is opened to achieve the desired exposure of around 40-50\% of the clusters. The main source capacity covering the grapes' import/sink needs is located in the main leaves in the bottom 2/3 of the foliage wall.


\section{\texorpdfstring{2\textsuperscript{nd} RM for L07 - Manipulation of growth and development by plant bioregulators}{2nd RM for L07 - Manipulation of growth and development by plant bioregulators}}

The growth and development of deciduous fruit trees are traditionally regulated by practices such as choice of rootstock, pruning, training, fertilization, water supply, and adjustment of crop load by hand thinning. When these cultural practices are insufficient, Plant Bioregulators (PBRs) are applied. PBRs are natural or synthetic compounds that mimic or counteract naturally occurring hormones. Examples include the cytokinin BA, which promotes cell division and fruit growth, and AVG, which suppresses ethylene formation. PBR results are often variable due to a poor understanding of their mode of action, "carry over" effects in fruit trees, variation in environmental conditions affecting uptake and translocation, and the specific anatomy and metabolism of different crops and cultivars. Societal concern and high re-registration costs pose significant obstacles to the continued use and development of synthetic PBRs \cite*{rm_05.2_L07_manipulation_growth_development_plant_bioregulators}.

\vspace{0.5em}
PBRs are generally applied as aqueous foliar sprays. A major constraint on effective delivery is penetration through the leaf surface, which is covered by the cuticular membrane (CM), a non-cellular lipoidal barrier. The effectiveness depends on the PBR's nature, concentration, formulation, droplet size, and carrier volume. Decreasing droplet size or increasing carrier volume per leaf generally increases uptake, translocation, and biological response. High-volume spray applications often provide greater consistency. Environmental factors greatly affect PBR uptake and response. For instance, uptake of NAA and NAAm in apple and pear leaves increased with temperature in the range 5\textdegree C to 35\textdegree C.

\vspace{0.5em}
\subsection*{Regulation of Fruiting (Thinning)}
Excessive fruit set must be reduced because fruits may not reach adequate size and quality due to insufficient leaf area per fruit. Overcropping also reduces or inhibits flower-bud formation, which leads to a biennial pattern of cropping. Hand thinning is the most accurate method, but chemical thinning greatly reduces cost and time.

\vspace{0.5em}
To reduce excessive fruit set, several PBR strategies can be followed: 

\begin{enumerate} 
    \item The initiation of floral primordia may be partially inhibited. \item Flowers can be killed or prevented from setting fruits at the time of blossoming (chemical flower thinning). 
    \item Thinning may be postponed until fruit set can be judged and, if abundant, chemicals aimed at promoting the drop of young fruitlets can be used (chemical fruit thinning). 
\end{enumerate} 

\vspace{0.5em}
The third method (chemical fruit thinning) is usually preferred as the least risky strategy. However, a too long delay in thinning diminishes the desired effects on fruit growth and flower-bud formation. For 'Empire' apples, fruit weight increased significantly when hand thinning occurred at bloom, 10, or 20 days after bloom compared to 40 days after bloom.

\vspace{0.5em}
Flower thinning is crucial for cultivars that set abundantly, as later fruitlet thinning may be ineffective. Two flower-thinning strategies exist: 

\begin{enumerate} 
    \item Using caustic compounds, such as ATS, that damage or severely desiccate stigmas and styles to prevent effective pollination. These compounds may also act by stimulating wound-ethylene evolution. 
    \item Using chemicals, such as ethephon, that enhance ethylene formation in the flowers. Ethephon application at anthesis increases ovule senescence and reduces auxin basipetal transport, leading to enhanced flower abscission. 
\end{enumerate} 

\vspace{0.5em}
\subsection*{Flower thinning} 
Flower thining success may advance fruit maturity. Thinning activity is promoted by cool, cloudy, wet periods preceding application, and is enhanced by high temperatures after application, which increases competition for assimilates between sinks (shoot tips and fruitlets). For long-term sustainability, the most promising thinning compounds are those that occur naturally (e.g., BA), degrade rapidly (e.g., ethephon), or disappear quickly (e.g., NAA, ATS).


\section{RM for L08+09 - Chapter 5 Growth, Growth correlations and Assimilate turnover}

Fruit growth is an important yield determinant and affects fruit quality. Fruits contain 80-90\% water. Of the dry matter, at least 90\% originates from assimilates formed in leaves by photosynthesis. In pome and stone fruit, the sugar-alcohol sorbitol is mainly formed and serves as the transport agent. Assimilate production and allocation form the basis for growth and fruit development. By incorporating radioactive CO$_2$ ($^{14}CO_2$), one can track assimilate use \cite*{rm_06_L08_growth}.

\vspace{0.5em}
Approximately 21\% of the dry matter produced in leaves in autumn builds up storage nutrients, which are stored in the tree and utilized the following spring. A significant share, 18\%, is stored in the roots. Storage nutrients include starch, sorbitol, and sugar. Fall and spring are often dominated by starch, while sorbitol and sugar proportions increase in winter. Most stored reserves (17 of 21\%) disappear completely, assumed used for respiration to generate energy for spring growth. Warm spring weather may lead to enhanced respiration and consumption of reserves, resulting in low carbohydrate content, reduced ovule development, lower flower quality, and reduced fruiting ability. Early autumn defoliation reduces carbohydrate storage, which subsequently reduces shoot growth the following year. High fruit load reduces root growth and stored carbohydrate reserves. This may reduce shoot growth and thickness growth early next summer and decrease the cell number per fruit. Storage materials play a role in other species; in strawberries, starch accumulates in the roots and is utilized during leaf and flower development in spring.

\vspace{0.5em}
In the very early growth stages, storage material is widely used; flower buds and young shoot tips receive 50-65\% of their building material from reserves, with the remainder from young leaves. New foliage is initially a net importer, but leaves become net exporters when they reach around half their final size. Vigorous-growing shoot tips are strong users of assimilates, dominating competition over young fruit. High young fruit drop is evident by particularly strong growth shoots. If shoot tips are removed, the assimilate demand of the shoot is temporarily reduced, and young fruitlet drop decreases. As shoot growth decreases and terminal growth is completed, only approx. 15\% of what the leaves produce are retained in the shoot. Large quantities then support fruit growth, such as in Figure 5-2e where a fruit received approx. 35\% of substance produced in leaves of annual shoots further out. Strong summer pruning (in August) that removes many annual shoot leaves can reduce fruit growth and dry matter content.

\vspace{0.5em}
Although fruits are not strong competitors in the earliest developmental stages, their ability to attract assimilates (as strong sinks) becomes very large later. A fruit in July-August consumes approx. 80\% of the assimilates formed in leaves on the same spur (Figure 5-2f). If there are many fruits, they can pull assimilates from other branches over distances of 1-1.5 m; in Figure 5-2g, 68\% of assimilates formed in leaves on a branch without fruit were allocated to fruits on a branch with many fruits. In strongly growing apples, at least 95\% of fruit carbohydrate consumption originates from leaf assimilates. Flowers and young fruits may contribute 15-33\%, mostly at the 'tight cluster' stage.

\vspace{0.5em}
The temporal increment of dry matter often results in an S-shaped curve (Figure 5-8). Spur leaves complete growth in June, followed by annual shoot leaves, shoots, small branches, main branches, stem, and roots, which often have their main growth in September-October. The strong import ability of fruits allows them to dominate competition for assimilates. At very high cropping levels, apple fruits can utilize nearly 70\% of total dry matter production in a growing season. Vegetative growth is strongly inhibited by high fruit load. In young trees, compared to similar non-bearing trees, high fruit load may reduce: 

\begin{enumerate} 
    \item Leaf growth to half. 
    \item Shoot growth to one-third to half. 
    \item Branch growth to a third. 
    \item Root growth increment even more. 
\end{enumerate} 

\vspace{0.5em}
The inhibitory effect is more powerful further from the fruit, correlating with the temporal coincidence of stem and root growth with the fruits' main growth period. To achieve strong vegetative growth in young trees, fruit should be removed as early as possible. The total photosynthetic production in young apple trees may be more than two times greater per leaf area unit in bearing than in non-bearing trees in July and August.

\vspace{0.5em}
In blackcurrant, even with a much lower share of total dry matter accumulation in fruits compared to apples (Table 5-9), reduced leaf and shoot growth are found at a good cropping level. However, differences are much smaller on branches and especially roots compared to apple. This is likely due to the earlier fruit harvest (approx. August 1), which limits the fruit's effect on late-season growth organs like roots. Unlike apple, where the root/top ratio decreases with cropping, the opposite tends to be true for blackcurrant. The strong pull in assimilates exerted by fruits causes a greater photosynthetic intensity and more rapid transport out of leaves. Assimilate retrieval from nearby leaves to growing fruit occurs with almost no loss, possibly due to CO$_2$ re-fixation within the fruit.


\section{RM for L10 - Table Grapes in Cold Houses}

This compendium note provides detailed information on growing vines in unheated glasshouses on a month-by-month basis, assuming the vines are well established. Maximum ventilation should be provided in January to keep buds dormant. In February, start maintaining a slightly higher temperature, opening ventilators near 7\textdegree C. Syringing with tepid water should only occur on sunny days, before midday, ensuring the house and vines are dry by evening. By March, the temperature can rise to about 10\textdegree C before admitting air. A moist atmosphere must be maintained by damping, sometimes twice a day on sunny days. Borders should be examined to a depth of 45 cm; if dry, water (30-40 mm), and repeat (10-20 mm) in a week, combined with manure water immediately after (never water manure on dry soil) \cite*{rmb_02_L10_fundamentals_temperate_zone_tree_fruit_production}.

\vspace{0.5em}
Mildew is a common disorder. To prevent infection, spray with a fungicide or use sulphite smoke when shoots are 5 to 7 cm long. In early April, reduce young growths to two shoots per spur, retaining the terminal one and a strong shoot near the base for short spur pruning. Before the end of April, pinch young shoots at two to four leaves beyond the bunch. In May, gradually tie laterals down to the wires using a loop-tie (Fig. 3), preventing them from touching the glass. During flowering, maintain a fairly high temperature, keeping the top ventilator open day and night unless there is an exceptionally cold spell. Drier atmospheric conditions are desirable before midday for drying pollen. To pollinate, draw the hand gently down the bunches about midday, or tap cordons sharply around 9 a.m. for shy setters.

\vspace{0.5em}
Immediately after berry set in June, the number of bunches must be reduced, retaining a small surplus. As berries swell, thinning should start in two stages using vine scissors, first removing all seedless berries and those pointing towards the centre of the bunch, retaining all tip berries. Overcropping must be avoided; a rough estimate for vines on the single cordon system is one bunch for every foot (30 cm) of cordon from the basal spur. A mulch of farmyard manure may be applied in June, ensuring ventilation is maintained to disperse fumes. In July, the berries undergo the first swelling (BBCH 77) followed by stoning (BBCH 77-79), a critical period where temperature should not fluctuate. After stoning, a second thinning is advised, suspending shoulders with raffia to maximize swelling (BBCH 80-89).

\vspace{0.5em}
As berries approach maturity in August, gradually reduce atmospheric moisture. For the second swelling (BBCH 83), temperature can be increased by early closure of ventilators, accompanied by copious damping's. Scalding (discolored sunken patches) is caused by hot sun on moist tissues; ventilators must be opened slightly before the sun strikes the house. Ripening is categorized using BBCH stages, with total days for BBCH 70 to 89 ranging from 75 (very early) to 100 (very late).

\vspace{0.5em}
Pest and Disease control includes managing Mealy bug with a 5\% tar-oil wash when buds are dormant, or approved sprays during the season. Shanking is a condition caused by unhealthy roots or overcropping, requiring root defect correction, crop reduction, and maintaining pH and K/Mg balance. Deficiencies, such as Magnesium (yellowish orange discoloration), can be corrected by spraying 250g magnesium sulphate in 10L of water plus spreader.

\vspace{0.5em}
By November, bunches should be removed, and maximum ventilation given to rest the vines; sub-lateral growth should be removed to ripen laterals. Pruning should occur immediately in December after leaf fall, cutting the lateral back to two buds to reduce bleeding. Cultivars tested at Pometet, such as those in the Interspecific resistant types (Group C), are ranked based on 25 culture technical parameters and 36 technological parameters. Very high ranked cultivars generally show high sugar content (score 7) $\approx$ 19\% brix and acid levels (score 5-7) between 6 to 6,8 g/L. Ventilation is crucial and should be managed using top ventilators first, avoiding drafts, and seldom requiring bottom ventilators except at flowering.


\section{\texorpdfstring{1\textsuperscript{st} RM for L11-13 - CHAPTER 6 Fruit growth and fruit quality}{1st RM for L11-13 - CHAPTER 6 Fruit growth and fruit quality}}

Fruit growth determines fruit size and shape, which is an important yield and quality component. External and internal quality factors vary greatly between species. Botanically, fruit types include a pome (pear, fusion of ovary and receptacle), a drupe (peach, one-seeded from ovary), a berry (grape, multiseeded from single ovary), an aggregate fruit (blackberry, many ovaries of a single flower), and an epigynous or false berry (blueberry). Most fruit is high in water, often over 80\%. Carbohydrates dominate the dry matter in most species, though nuts contain protein and fat. Fructose has the highest relative sweetness (173, with sucrose as 100). Malic acid or citric acid are most often the dominant organic acids \cite*{rm_07.1_L11_13_fruit_growth_quality}.

\vspace{0.5em}
At least 90\% of the fruit's dry matter originates from assimilates formed in leaves via photosynthesis. In pome and stone fruit, the sugar-alcohol sorbitol is mainly formed and serves as the transport agent, while sucrose is the transport agent in most other species, including fruit bushes and strawberries. Approximately 21\% of dry matter produced in leaves in autumn is stored as reserves (starch, sorbitol, sugar), with 18\% stored in the roots. Storage reserves are crucial, as flower buds and young shoot tips receive 50-65\% of their building material from them in early growth stages.
Fruit growth is conceived to be affected by the assimilate level, which depends on the ratio of assimilate producing organs (sources) and utilizing organs (sinks). Source strength is the capacity to synthesize compounds for export, and sink-strength is the potential capacity for accumulation of metabolites.

\vspace{0.5em}
Leaf/Fruit Ratio and Assimilate Turnover
With increasing leaf/fruit-ratio, increased fruit growth, and increased concentration of total solids, soluble solids (sugar), and acid are observed in apples. Conversely, a larger fruit/leaf ratio causes a stronger "pull" in assimilates, which animates leaves to greater production (greater photosynthetic intensity) and faster transport, but results in a lower assimilate level and lower concentration of certain substances, such as fruit dry matter. The relationship between yield (y) and number of fruits (x) per hectare can be expressed as y=kx/K+x, potentially using k to express source activity and 1/K to express sink activity. High fruit load strongly inhibits vegetative growth in young trees, reducing leaf growth to half and shoot growth to one-third to half.

\subsection*{Source and Sink Activity}
Increasing source activity, such as through more light or higher CO$_2$ concentration, leads to a greater "pressure" from the leaves, resulting in larger fruits and increased accumulation of soluble solids (sugar). Light is an important factor affecting source activity in the field. Increasing sink activity in the fruits, often affected by substances from the root (e.g., potassium), causes fruits to "pull" assimilates stronger, resulting in larger fruit size but a lower dry matter content. For example, a good potassium supply resulted in larger fruit (117 g/fruit) but lower soluble solids (15,1\%) compared to fruits lacking potassium (98 g/fruit, 15,9\% soluble solids). Cultivar differences in fruit growth rate potential, such as 'Graasten' being approximately 25\% higher than 'Golden Delicious', are apparently due to genetic differences in fruit sink activity.
Variation in Species

\vspace{0.5em}
The importance of the leaf/fruit-ratio on fruit growth is less pronounced in small-fruited species. In plums, reduced fruit number increases fruit size and solids/sugar content. In small-fruited sour cherry ('Stevnsbær'), reducing fruit number gives no or only a small increase in size. In black currant, sink activity is the most important factor for fruit growth, and large fruits are negatively correlated with solids concentration. For strawberries, flower development and quality are essential for berry size. Berry weight is a linear function of achene number per berry at a given achene density (Annex 6-23). Development of achenes (seeds) is important as they release growth substances (auxin) needed for berry growth. In strawberries, thinning effects on berry growth are small, and the impact of physiological and cultivation factors on sugar and acid content is very limited. The relative importance of key factors varies across species (Annex 6-6):

\begin{enumerate} 
    \item Leaf/fruit ratio shows a positive correlation (+) with Fruit size, Dry matter, Acid, and Yellow background color. 
    \item Source-activity (light) shows a positive correlation (+) with Fruit size, Dry matter, Acid, and Yellow background color. 
    \item Sink-activity shows a positive correlation (+) with Fruit size, but a negative correlation (-) with Dry matter (sugar) and Red over-color (apples). 
\end{enumerate}


\section{\texorpdfstring{2\textsuperscript{nd} RM for L11-13 - CHAPTER 7 Fruit Development and fruit ripening}{2nd RM for L11-13 - CHAPTER 7 Fruit Development and fruit ripening}}

Fruit development is characterized by phases which gradually transition, resulting in an overall effect on the developed fruit that is quantitatively influenced by the source-sink ratio. Fruit development transforms into ripening, which is characterized by a decrease in size increment that eventually stops, as well as the synthesis of specific substances, such as colors and flavors. The fruit may be tree- or picking ripe (mature) for a longer period before becoming eating ripe (ripe), particularly with many apple and pear varieties, though berry species often unite these stages.

\vspace{0.5em}
Size increment typically follows an S-shaped (sigmoidal) curve in apples, pears, and strawberries. Stone fruit growth follows a double S-shaped curve that can be divided into three stages: 

\begin{enumerate} 
    \item Seed coat and endosperm develop. 
    \item Seed coat hardens (lignifies, or pit-hardening), and the embryo develops. Growth seems less powerful. 
    \item Growth is again rapid, and the flesh develops via cell expansion, increasing the proportion of intercellular space. 
\end{enumerate} 

\vspace{0.5em}
Black currants (inner fruits in the raceme) can also show a double S-shaped growth curve. Cell divisions are dominant during the first 10-20\% of the fruit developmental period, equalling the first approximately 4 weeks after flowering in apple. Thereafter, the time is dominated by cell expansion, although in strawberries, cell divisions may occur over a major part of the growing period.

\vspace{0.5em}
With fruit development, the attraction between cells decreases as changes shorten the molecules within the pectin fraction, decreasing fruit firmness. The content of solids generally increases, with soluble solids (sugar) rising particularly sharply at the end. Assimilates (at least 90\% of dry matter) are transported mainly as the sugar-alcohol sorbitol in pome and stone fruit, and as sucrose in most other species, including fruit bushes and strawberries. In apples, total acidity decreases with time, partly because de novo synthesis mainly occurs early in the growing season (when sorbitol is largely transformed into acid, e.g., malic acid, by June 30), and partly because already formed acid is eventually metabolized during ripening. Sorbitol transported into the fruit later (after August 20) yields especially sugars, mostly fructose. Starch accumulates and subsequently decomposes in apples, contributing to the increase in total sugar content.

\vspace{0.5em}
The development and maturation rate is influenced by the rate of turnover in each phase, largely affected by source-sink relationships, and the duration of each phase. Increased temperature promotes the metabolic turnover rate and shortens the duration of individual phases, thus increasing development and maturation rate. In late apple and pear varieties, increased temperature leads to larger fruit (Annex 7-6) and a higher sugar content (Annex 7-7). For the sour cherry cultivar 'Stevnsbær', the number of days between full bloom and harvest can be calculated as 160-(0.13$\times \sum$T$_{0-40}$), where $\sum$T$_{0-40}$ is the temperature sum in the first 40 days after full bloom.

\vspace{0.5em}
Harvest criteria in apples and pears include fruit size, change in base color from green (chlorophyll decomposition) to yellow (carotenoids), formation of red over color (anthocyanins, which require light), pip color change (white to brown), formation of the abscission layer, and firmness. High sugar content, approximated by soluble solids (refractometer), and decreasing acidity are important quality factors. The degradation of starch, followed by iodine-potassium-iodine solution staining, is also used as a harvest criterion. Because a single criterion is too risky, an index is used, such as the equation \ref{eq:streif_index} that shows the Streif index: 

\begin{equation}
\text{Streif index} = 
\frac   {\text{Fruit firmness (kg/cm$^2$)}}
        {\text{Refractometer (\%\text)} \times \text{starch value (0-10)}}
\label{eq:streif_index}
\end{equation}

\vspace{0.5em}
The index decreases during maturation. Optimal index values vary by variety, e.g., 'Elstar' 0.30 to 0.38. Fruits for fresh consumption are still picked by hand, while fruits for industry (e.g., sour cherries, black currants) are largely harvested mechanically.


\section{RM for L13 - Growth and development in black currant (Ribes nigrum)}
\textbf{III. Seasonal changes in sugars, organic acids, chlorophyll and anthocyanins and their possible metabolic background}

Potted black currant plants of 'Tenah' (one year old, 1992) and 'Ben Nevis' (three year old, 1993) were sampled weekly from shortly after anthesis until harvest to measure fruit growth as dry-matter increments and seasonal changes in soluble sugars, organic acids, chlorophyll, and anthocyanins. Fruit growth in Ribes may be divided into two major rapid growth phases separated by a relatively short transition period. The typical double sigmoid growth pattern includes a plateau between 25-38 d from fertilization. The first rapid growth period coincides with cell division and seed growth.

\vspace{0.5em}
In the very early stage of fruit development, high concentrations of sugar, malic acid, and chlorophyll were observed. Sugar concentration was initially high, then dropped to a minimum about 3-4 weeks after full bloom. Sugars started to accumulate rapidly, followed by a declining rate during the last weeks of ripening. During ripening, sugars, especially glucose and fructose, accumulated. Fructose reached a higher level while sucrose attained much lower values.

\vspace{0.5em}
For organic acids, the initial concentration of malic acid in young fruits was high, declining to a steady state level of about two thirds of the initial level. During ripening, malic acid concentration fell rapidly until harvest. Citric acid concentration increased rapidly along with fruit growth until harvest, resulting in citric acid dominating the acid content per fruit. The accumulation of citric acid was similar in both years/cultivars on a per fruit basis. The total acid content decreases because initial malic acid concentration is very low and malic acid dominates, initially accounting for 70-75\% of total organic acids. The final malic/citric acid ratio was 1:6-10 at maturity.

\vspace{0.5em}
The sugar/acid ratio was initially high (1.4-1.5), dropping dramatically to a minimum between 35 d after full bloom (0.42-0.45), before returning gradually to the initial level during the last growth phase.
The total chlorophyll concentration increased during early development. On a per-fruit basis, chlorophyll content rose strongly until 35 d after full bloom and then remained constant. Anthocyanins, low in immature fruits, showed a very rapid increase during ripening.

\vspace{0.5em}
Metabolically, immature green fruits have a high capacity to support their own growth and carbon requirement through photosynthesis and dark respiration. Photosynthesis may be supported by CO$_2$ fixation via PEPcarboxylase, leading to the formation of oxaloacetate (OAA) and subsequently malate. Citric acid accumulates strongly in both immature and ripening fruits. A close correlation was found between citrate content and water content per fruit (R$_2$=0.97). The large amount of citric acid suggests a high metabolic priority. The period of maximum sugar minimum concentration, spanning approximately 30-45 d from full bloom, may reflect a period of critical internal competition for resources.


\section{\texorpdfstring{1\textsuperscript{st} RM for L15-16 - Advances in Fruit Aroma Volatile Research}{1st RM for L15-16 - Advances in Fruit Aroma Volatile Research}}

Fruits produce a range of volatile compounds that form their characteristic aromas and contribute to flavor, which comprises sweetness, acidity, or bitterness (perception in mouth) and odor. Fruit volatile compounds are mainly comprised of esters, alcohols, aldehydes, ketones, lactones, terpenoids, and apocarotenoids. Aroma is a complex mixture whose composition is specific to species and often variety. Volatile esters often represent the major contribution in apple (Malaus domestica Borkh.) and peach (Prunus persica L.). Many C$_{10}$ monoterpenes and C$_{15}$ sesquiterpenes compose the most abundant group, sometimes determining the characteristic aroma, such as the terpenoids S-linalool, limonene, valencene, and $\beta$-pinene in strawberry, koubo, and citrus. The volatile profiles of fruit are complex and vary depending on the cultivar, ripeness, pre- and postharvest environmental conditions. Volatiles are classified as primary (present in intact fruit tissue) or secondary (produced upon tissue disruption). The proportion of glycosidically bound volatiles is usually greater than that of free volatiles, making them an important potential source of flavor compounds \cite*{rm_09.1_L15_16_fruit_aroma_volatile_research}.

\vspace{0.5em}
Volatile profiles are highly species-specific. For example, more than 300 volatile molecules are reported in apples, with esters being the most abundant. Hexyl acetate, hexyl 2-methyl butanoate, hexyl hexanoate, hexyl butanoate, 2-methylbutyl acetate, and butyl acetate are prominent in 'Pink Lady' apples throughout maturation. In strawberry, the furanones 2,5-dimethyl-4-hydroxy-3(2H)-furanone (furaneol) and its methyl derivative (mesifurane) are dominating aroma compounds. Esters, which are the most important group, cover 90\% of the total volatiles in ripe strawberry fruit. Banana fruity top notes are primarily volatile esters, such as isoamyl acetate and isobutyl acetate. Terpenes, such as D,L-limonene and valencene, are the major class of compounds in citrus. Peach aroma is defined by C$_6$ aldehydes and alcohols (green-note) and lactones and esters (fruity aromas).

\vspace{0.5em}
Factors affecting volatile composition include genetic makeup, maturity, environmental conditions, postharvest handling, and storage. Maturity is critical; immature fruits produce low quantities of volatiles and lose the capability of volatile production more readily during storage than mature fruit. C$_6$ aldehydes and alcohols are major contributors to the flavor of immature fruits, but their levels decrease drastically during ripening as furanone and ester production increases. Postharvest handling significantly influences volatiles. Storage temperature is fundamental. Storage under Controlled Atmosphere (CA) conditions (low O$_2$ and high CO$_2$) alters production, reducing the capacity to synthesize ethylene and aroma volatiles.

\vspace{0.5em}
Volatile aroma compounds are biosynthesized via several pathways starting from lipids (fatty acids), amino acids, terpenoids, and carotenoids. 

\begin{enumerate} 
    \item Fatty Acids Pathway: Fatty acids are major precursors for C$_1$ to C$_{20}$ straight-chain compounds. The $\beta$-oxidation pathway forms volatiles in intact fruit, providing alcohols and acyl CoAs for ester formation by removing C$_2$ units. The Lipoxygenase (LOX) pathway produces saturated and unsaturated volatile C$_6$ and C$_9$ aldehydes and alcohols, typically when tissue is disrupted. 
    \item Amino Acid Pathway: Amino acids, such as alanine, valine, leucine, and isoleucine, are direct precursors. They undergo deamination or transamination to form $\alpha$-keto acid, followed by reductions, oxidations, and/or esterifications to form aldehydes, acids, alcohols, and esters. 
    \item Terpenoids Pathway: Terpenoids are derived from the C$_5$ precursor isopentenyl diphosphate (IPP) and its isomer DMAPP. C$_10$ monoterpenes originate from geranyl diphosphate (GPP), formed in plastids via the MEP pathway. Terpene Synthases/Cyclases (TPSs) convert these precursors into diverse cyclic and acyclic compounds. 
    \item Carotenoid Pathway: Apocarotenoid derivatives are produced by the oxidative cleavage of carotenoids, catalyzed by Carotenoid Cleavage Dioxygenases (CCDs). Apocarotenoid volatiles are synthesized only at the latest stage of ripening. 
\end{enumerate}


\section{\texorpdfstring{2\textsuperscript{nd} RM for L15-16 - Advances in Fruit Aroma Volatile Research}{2nd RM for L15-16 - Advances in Fruit Aroma Volatile Research}}

The formation of volatile aroma compounds is influenced by the ripening process of the fruit, including the effects of harvest date and storage conditions. A negative relationship exists between the fruit/leaf ratio and the availability of assimilates and sizes, as well as the concentration of total and soluble dry matter and acids. Fruit growth is conceived to be affected by the assimilate level, which depends on the ratio of assimilate producing organs (sources) and utilizing organs (sinks). The great import ability of the fruits means that they will dominate over other organs in the competition for assimilates when they are out of the early growth stages \cite*{rm_09.2_L15_16_fruit_aroma_volatile_research}.

\vspace{0.5em}
One-year-old apple trees of Malus domestica cv. Jonagored were divided into three fruit/leaf ratio groups: Low (130 fruit:leaf DM 10$^{-3}$ kg/tree), Cropping median (172), and High (381). The study established that at a low fruit/leaf ratio, larger apples were produced, accompanied by higher concentrations of total dry matter, soluble solids, titrateable acids, and a lower firmness value, compared to apples from trees with a high fruit/leaf ratio. Fruit firmness decreased during the ripening period, and apples from trees with a high fruit/leaf ratio were softer than apples from trees with the low ratio at harvest.

\vspace{0.5em}
Headspace Gas Chromatography (GC) analysis showed that aroma compounds consisted of approx. 20\% esters, 71\% alcohols, and 6\% C$_6$ aldehydes. The total concentration of aroma compounds increased during the ripening period and was most pronounced at the lowest fruit/leaf ratio. Esters and alcohols were dominant, while C$_6$ aldehydes showed no significant difference between the fruit/leaf ratios. The higher aroma concentration associated with low fruit/leaf ratios is supported by the theory that greater availability of assimilates favors the accumulation of precursors for the synthesis of aroma compounds.

\vspace{0.5em}
Ethylene concentration was measured during a 3-week ripening period after cold storage. The ethylene concentration was between 45 and 83 ppm on the first day and peaked five days later (15 November). The ethylene concentration was most pronounced at the lowest fruit/leaf ratio during the ripening period and was lowest in fruits from trees with a low fruit/leaf ratio during the last part of the ripening period. The greater availability of assimilates at a low fruit/leaf ratio favors accumulation of substrates for synthesis of aroma compounds, larger pools of sugars and organic acids, and greater levels of aroma compound precursors like acetyl CoA and fatty acids, which are transformed into the esters and alcohols found in the study. High fruit load strongly inhibits vegetative growth in young trees, reducing leaf growth to half and shoot growth to one-third to half.


\section{\texorpdfstring{3\textsuperscript{rd} RM for L15-16 - Advances in Fruit Aroma Volatile Research}{3rd RM for L15-16 - Advances in Fruit Aroma Volatile Research}}

Fruit growth is an important yield determinant, with at least 90\% of the fruit's dry matter originating from assimilates formed in leaves via photosynthesis. In pome and stone fruit, the sugar-alcohol sorbitol is mainly formed and serves as the transport agent, while sucrose is the transport agent in most other species, including fruit bushes and strawberries. Approximately 21\% of the dry matter produced in leaves in autumn is stored as reserves (starch, sorbitol, and sugar), with a significant share (18\%) stored in the roots. Storage material supplies 50-65\% of the building material for flower buds and young shoot tips in the very early growth stages. The great import ability of fruits means they dominate competition for assimilates when they are out of the early growth stages. For example, a fruit in July-August consumes approx. 80\% of the assimilates formed in the leaves on the same spur. High fruit load strongly inhibits vegetative growth, potentially reducing leaf growth to half and shoot growth to one-third to half in young apple trees. The effect is more powerful further from the fruit, coinciding temporally with the main growth period of organs like roots (September-October). This strong sink activity causes a greater photosynthetic intensity and more rapid transport out of leaves in fruiting plants.

\vspace{0.5em}
Biological variation is substantial, and even within the same fruit cultivar, flavor and aroma can vary significantly. This variation affects uniformity and categorization strategies. Apples from three cultivars ('Elshof', 'Holsteiner Cox', and 'Ingrid Marie') were analyzed individually for sugar, acid, and aroma compounds, having been picked at maturity from four zones on the trees (top, bottom, east, west). Using Principal Component Analysis on the relative peak areas of 59 tentatively identified aroma compounds, the three cultivars were completely separated. For 'Ingrid Marie' apples, effects of position could not be elucidated due to inadequate juice yield.

\vspace{0.5em}
A discriminant Partial Least Squares regression analysis revealed distinct differences based on position for 'Elshof' apples, separating apples from the top and bottom parts of the trees. The level of sugar (soluble solids) was highest in apples from the top (13.4\% sugar) and lowest in apples from the bottom (12.3\% sugar). This positional effect relates to differences in quality and ripening levels, likely associated with light exposure affecting photosynthetic activity. High soluble solids are generally associated with high production of esters. For 'Elshof', compounds high in the top included 2-Methylbutyl acetate (2400a) and 2-Phenylethyl acetate (2.7a), while Hexanal (1300a) and 2-Butanol (5.6a) were high in the bottom. For 'Holsteiner Cox', apples from the west side separated well, characterized by lower levels of specific compounds, including 4-Methyl-5-vinylthiazole (2.2b), Anethole (1.0b), and 3-Hexenal (16b). Differences between top and bottom were most evident in 'Elshof'.

\vspace{0.5em}
The magnitude of variations in concentrations of characteristic odorants (e.g., 2-Methylbutyl acetate, pentyl acetate) is substantial, indicating that position on the tree is likely to affect sensory quality. It is concluded that when apples are sampled for categorization and characterization, it is extremely important to sample from all parts of the trees or clearly define the position from which the fruits are sampled. Data also indicates the importance of good light exposure on fruits to ensure high quality.


\section{\texorpdfstring{1\textsuperscript{st} RM for L17+18 - Sustainable production systems for organic apple production}{1st RM for L17+18 - Sustainable production systems for organic apple production}}

Organic apples often sold for fresh consumption must meet high quality requirements, obtaining the right size and being undamaged without important infections of pest and diseases. The organism most often causing damage on the skin of apples is apple scab, which creates brown or black spots. This disease causes a big reduction in yield and quality in organic production. The yield reduction for growers not using copper for control was 86\% compared to conventional production, depending on the cultivar and the year. Copper is an effective fungicide used in organic apple production in some European countries, but it has not been permitted in Denmark for the last 10 years, and the European Union wants to reject it from the list of permitted pesticides \cite*{rm_10.1_L17-18_sustainable_production_organic_apple_production}.

\vspace{0.5em}
To improve quality and yield, research has been carried out on cultivars with low susceptibility, using combinations of fertilisation, rootstocks, and planting distance to prevent or reduce apple scab infections. Control has involved using the warning program RIMpro to predict severe ascospore discharge periods. The most important prevention method is to use genetic resistant apple cultivars or cultivars with less susceptibility. Small, open, rather slow-growing apple trees reduce infection possibilities. The resistance to apple scab was broken down in most cultivars, but 'Florina' was still fully resistant during the experiment, and 'Vanda', 'Retina', and 'Redfree' were less infected.

\vspace{0.5em}
A high level of available nitrogen in the soil causes increasing growth and an extended growth season, which gives better infection possibilities for apple scab. Increased nitrogen also causes a decrease in the content of phenols in plant tissue. The effect of cover crops was investigated, comparing a permanent grass mixture (1), a clover grass mixture (2), and an annual cover crop (3). Fruits produced on trees managed with a permanent grass alleyway (1) had a lower nitrogen supply and obtained the best skin coloration. A lower nitrogen supply, especially during fruit development, resulted in more red fruits. Apple scab infections were more numerous on apples grown in the annual cover crop (3), which gave the largest nitrogen supply. Fruits from the grass alleyways (1) had the highest percentage of marketable fruits. Although the gross yield was bigger from trees grown in the annual cover crop (3), the marketable fruit crop was at the same level due to the higher percentage of disease infections.

\vspace{0.5em}
Early end of vegetative growth and development of end buds in autumn are very important to reduce the possibility of late infection in autumn and thereby reduce the risk of spring infections from conidia wintering in woody parts. End bud development of 51 rootstocks showed that rootstocks with high winter hardiness also had the earliest end bud development. MM106, a rootstock with poor winter hardiness and the latest end bud development of the 51 rootstocks, should not be recommended for organic apple production. The weak rootstocks M9 and the Russian rootstock B9 are potentially suitable due to: 

\begin{enumerate} 
    \item Early end bud development. 
    \item Weak growth (M9). 
    \item High winter hardiness (B9). 
\end{enumerate} 

\vspace{0.5em}
The fruit quality recorded consists of both outer quality (fruit size, fruit color, and damages caused by pest and diseases on the fruit skin) and inner quality (firmness and content of sugar and starch).


\section{\texorpdfstring{2\textsuperscript{nd} RM for L17+18 - Fruit quality in organic and conventional farming: advantages and limitations}{2nd RM for L17+18 - Fruit quality in organic and conventional farming: advantages and limitations}}

Fruit quality is essential for nutrition and human health, requiring urgent attention in current agricultural practices. Modern agriculture must address the challenges of sustainability, crop quality, and yield, corresponding to United Nations Sustainable Development Goals (SDGs). Conventional agriculture, which accounts for a huge part of global plant production, utilizes an average of 1.81 kg of pesticides and 28.48 kg of nitrogen per hectare. Fruit quality is a complex, multifactorial concept assessed by consumers based on organoleptic traits (appearance, aroma, flavor, and texture), maturity stage, nutritional value, and presence of harmful chemical compounds. Fruits contain essential bioactive compounds, including vitamins (E, C, B group), essential mineral nutrients (K, Ca, Mg, P, S, Fe, Zn, Mn), flavonoids, isoprenoids, and anthocyanins. Immature fruit is characterized by excessive firmness, poor aroma, low sucrose, high starch, and green color \cite*{rm_10.2_L17-18_sustainable_production_organic_apple_production}.

\vspace{0.5em}
Organic farming is a more sustainable production model than conventional agriculture. It can provide higher quality in some fruit crops due to the absence of synthetic fertilizers and pesticides, enhanced pollination, and reduced protection treatments, which boosts antioxidant compound production. A meta-analysis of 71 studies published over the past 5 years (2019-2023) showed that organic farming improved overall fruit quality in 38 (53.5\%) cases, whereas only 6 (8.4\%) showed negative effects. Effects appear largely species-specific, with grape berries, peppers, and apples benefiting most.

\vspace{0.5em}
Organic farming inherently contributes to high-quality fruit through several properties: biologically active soil, management practices such as compost and green manure, enhanced pollination, low/moderate biotic stress, and the absence of harmful compounds derived from fertilizers and pesticides. Total pesticide content in organic farming soils is 70-90\% lower than in conventional soils, and nitrate and nitrite residues are usually lower in organically grown crops. Long-term improved soil health is a key principle of organic farming. The low/moderate biotic stress inherent to organic systems, coupled with the absence of synthetic pesticides, leads to a controlled exposure to stress factors. This pathogen exposure can trigger the synthesis of plant defense metabolites, such as phenolics, terpenoids, and flavonoids. Salicylic acid and jasmonates are phytohormones involved in activating phenylpropanoid metabolism in response to biotic stress, enhancing compounds valuable for human nutrition. Furthermore, organic farms enhance pollination success.

\vspace{0.5em}
However, organic farming is not without potential risks. The use of compost and organic fertilizers, such as manure, means a significant input of microorganisms, potentially increasing the occurrence of pathogenic bacteria like Escherichia coli, yeasts, molds, and Bacillus cereus in organically grown vegetables. Nonmicrobial contaminants, such as heavy metals (Cu and Zn), can accumulate from sources like municipal sewage sludge or animal manures. Copper-containing plant protection products used in organic farming can also contribute to heavy metal excess. Despite these risks, organic farms generally perform much better in terms of pesticide residues. Fruit quality is a multifactorial concept affected by factors such as stress intensity, pruning and crop load, genetic traits, and rootstocks in both conventional and organic models. The integration of all parameters is essential for achieving high quality and productivity in sustainable farming models.


\section{RM for L19 - 9 Light Relations}

Visible light in the 400-700 nm waveband is the driving factor of biomass production via photosynthesis. Although the production of dry matter in apple correlates to the amount of visible light intercepted by trees, dry-matter production does not automatically translate into increased yield of marketable fruit. Light plays a twofold role: influencing processes leading to the production of large quantities of high-quality fruit, and supplying energy stored in chemical form (carbohydrates). The quantity of light plants use is defined as Photosynthetic Photon Flux (PPF) or Photosynthetically Active Radiation (PAR). Overhead sunny conditions yield a PPF of around 2000$\mu$mol $m^{-2} \cdot s^{-1}$. Diffuse radiation can penetrate the canopy virtually in every direction, resulting in higher light levels in inner tree-canopy positions. Apple leaves absorb approximately 80\% of incoming visible radiation \cite*{rm_11_L19_light_relations}.

\vspace{0.5em}
Orchard management aims to maximize light interception, with efficient orchards intercepting 60-90\% of available radiation, yielding high marketable production (up to 120-140 t ha$^{-1}$ year$^{-1}$). Orchard productivity is often higher when light interception is greater than 50\% of available light. Total tree net CO$_2$ exchange (NCE) is found to be greater for fruiting trees than for non-fruiting trees when expressed per tree leaf area. Fruiting reduces light interception due to lower leaf area, but stimulates the photosynthetic process, making the foliage more efficient. Yield (t $\cdot$ ha$^{-1}$) is strongly correlated with whole-canopy Leaf Area Index (LAI) (R$^2$=0.774).

\vspace{0.5em}
Apple leaves develop distinct morphological and physiological traits reflecting their light environment. 'Sun' leaves (exterior, well-illuminated) exhibit greater maximum photosynthetic capacity than 'shade' leaves (interior, low PPF). Shade leaves are typically denser, possessing more layers of cells, and higher nitrogen content. Shading spur leaves in the previous season was found to reduce the leaf photosynthetic rates.

\vspace{0.5em}
The partitioning of carbon between different sinks (vegetative vs. reproductive) is complex. Primary spur leaves are vital early in the season, supplying carbohydrates for fruit set and cell division. They are the only carbon source until developing shoot leaves become net exporters (around 10-15 unfolded leaves). Early shading delays the transition of shoot carbohydrates to export. Heavy shading (70\%) applied 5 weeks after full bloom (AFB) drastically reduced fruit growth and increased fruit drop. Low PPFs reduce flower-bud differentiation, fruit set, fruit size, soluble-solids concentration, and firmness.

\vspace{0.5em}
Red fruit color formation (anthocyanin synthesis) is light-dependent, requiring light intensity and quality. This process is stimulated by UV-B radiation (<320 nm) and enhanced synergistically by red light. Fruits grown under canopy shade show increased transpiration and shrivelling during storage.

\vspace{0.5em}
Controlling light levels in the orchard improves quality. Pruning is used to maintain adequate light penetration. The optimum training system for maximizing light distribution is the slender spindle. Reflective materials (Reflectants), such as aluminum or tar-paper, increase light availability in the lower and inner parts of the canopy. Reflective ground cover applied 2-3 months before harvest improves skin color (via UV region effects), increases fruit size, and boosts soluble-solids concentration. Bagging fruits 4-6 weeks AFB, followed by bag removal, also enhances red color formation.


\section{RM for L20+22 - Planting and training systems, pruning and fruiting control}

Woody plants possess high morphological plasticity and strong adaptability. Agronomic and pruning techniques are utilized to steer the "compensatory-adaptive" growth of the tree to obtain an equilibrium between vegetative and productive organs, and between these organs and the root system. Pruning is a powerful technique to control tree shape and size, and to improve fruit yield and quality. Modern intensive orchards utilize small trees and management methods aiming to begin production as early as possible, to facilitate mechanization, and to limit costs and labor. The essential goal of modern pruning is to obtain as early a canopy development as possible and achieve a balanced and independent (natural) growth. Maximizing orchard profitability is prioritized, even at the cost of shortening the life cycle, which usually lasts about ten to fifteen years. Pruning accelerates development, shapes the scaffold, and aims to quickly overcome the initial unproductive phase to obtain high and consistent yields. Young tree pruning is done concurrently with training to obtain a desired shape. Mature tree pruning adjusts crop load and fruit quality, maintains tree shape and size, and delays natural aging \cite*{rmb_03_L20+22_planting_training_systems_pruning_fruiting_control}.

\vspace{0.5em}
Pruning cuts are classified as either a thinning cut, which is the removal, at the base, of an entire shoot or branch, or a heading back cut, which is a partial removal of the distal part of a branch or limb to shorten it. Heading back cuts on two or more years old branches typically aim: 

\begin{enumerate} 
    \item to bring the branch back to its natural position (return back cut). 
    \item to restrain the development of the branch (diversion cut). 
\end{enumerate} 

\vspace{0.5em}
Long pruning leaves the apical part of the branch intact and is almost always rich, promoting the reproductive phase. Short pruning systematically shortens shoots or branches, creating a spur or shortened shoot, resulting in a strong reduction of buds ("poor pruning"), which produces fewer, but more vigorous shoots. Pear and apple trees tolerate winter cuts better than Prunus species.

\vspace{0.5em}
The architecture of the tree consists of the framework of branches formed according to a training system. Tree size reduction is achieved using dwarfing rootstocks, competition between trees, modulating root growth, or stimulating fruiting to create sinks for photosynthates at the expense of vegetative and root growth. Training systems are classified based on architecture: 

\begin{enumerate} 
    \item Volume shapes: Require well-defined, balanced, three-dimensional framework structures, requiring 3 to 5 years to complete (e.g., open vase). 
    \item Hedgerow systems: Create a thin continuous solid hedge (row of trees), easier to manage and adaptable to mechanical operations, requiring low-vigor rootstocks (e.g., palmette). 
\end{enumerate}

\vspace{0.5em}
Green pruning is a spring or summer practice. If topping (cutting the terminal part of the shoot) is done prematurely, it produces new shoots; if done late (when growth slows), it temporarily favors shoot growth arrest and can promote flower bud formation. Shoot twisting, performed on vigorous shoots (>40-50 cm long), partially breaks the shoot to temporarily block growth, promoting the induction of reproductive buds. Pruning in adult trees pursues general objectives including adjusting fruit load, preserving canopy shape, removing diseased parts, maintaining balance between canopy and root system, and improving fruit quality (size, color, sensory quality).

\vspace{0.5em}
Modern medium-high density orchards (2,000-3,500 trees ha$^{-1}$ for apples/pears) frequently use the Slender Spindle (spindlebush), a semi-free volume shape maintaining a central axis and 4 to 8 primary branches in a spiral, requiring dwarfing rootstocks. The Superspindel is an extremely simplified, almost columnar framework, devoid of branches, planted very thickly (30 to 60 cm on the row). The Regular palmette with oblique branches is a fixed hedgerow shape suitable for medium and medium-high density. In apple cultivation, the most common volumetric shape is similar to the spindle (Spindel busch). In general, the rows in apple orchards are laid out in a North-South alignment to promote a balanced development on both sides of trees.


\section{RM for L21 - Production of Fruit Juices and Effects of Processing on Quality}

Fruit juice is defined as the fermentable but unfermented juice obtained from fruit by mechanical processes, characterized by the color, aroma and taste characteristic of the fruit of origin. The juice must come from the endocarp for citrus fruits. Fruit juice also includes product made from concentrated fruit juice by restoring extracted water and flavor using aromatic substances obtained from concentration of the juice or juice of the same species. Fruit juice from certain fruits destined for concentration may also be obtained through diffusion processes. Apple juice may not be added sugar and must have a refractive dry matter content not less than that of an aqueous solution containing 100g sucrose per 1000g of solution. Fruit nectars are fermentable but unfermented products obtained by adding water and sugar or honey to fruit juice, concentrated fruit juice, or puree. Black currant nectar, for instance, must contain 8 g/L acid, calculated as tartaric acid, and must contain 25\% juice \cite*{rmb_04_L21_postharvest_biology_technology_fruits_vegetables_flowers}.

\vspace{0.5em}
Most apples used for juice production are industrial apples, which are sorted from table apples. Sugar and acid content are probably the most important sensory parameters in apple juice, with a sugar/acid ratio of approx. 15 perceived as optimal by a taste panel. Ripeness significantly affects quality; aroma content rises sharply with increasing ripeness while acidity and total content of phenols decrease. Total phenols give a bitter-astringent "taste". Black currants and sour cherries are major berry crops used for juice, valued for their high content of sugar, acidity, ascorbic acid, anthocyanin, and aroma compounds. Mechanical harvest and high temperatures expose berries to rapid enzymatic and microbiological changes; studies show that unless sour cherries are cooled, deterioration happens quickly (2-3 days), causing degradation of color, acid, and sugar, and the formation of ethanol and acetic acid.

\vspace{0.5em}
In juice extraction, apples are washed to remove debris and pesticide residue. Berries are either not washed or sprinkled gently. Fruit must be cut into pieces (0.2 to 1 cm$^3$) before pressing. For berries (especially black and red currant) and ripe apples, pasteurization (80-90\textdegree C) or enzymatic treatment using pectin degrading enzymes is often implemented on the fruit mass (maische) to improve juice yield and inactivate enzymes/microbes. Pectin degrades, increasing juice yield and increasing methanolin content (e.g., from 30-100 mg/L to 300-400 mg/L in enzyme maische), but may reduce esters and aldehydes. Press yield estimates are generally 65-85\% on a weight basis. Alternatives to pressing include extraction with hot water, allowed if the juice (extract) is subsequently concentrated.

\vspace{0.5em}
Juice clarification is necessary as most apple and berry juices are sold as clear juice. Turbidity is caused by colloidal particles, including protein-phenol aggregates and pectin-coated particles. Centrifugation and filtration alone are insufficient. Clearing agents are used to aggregate particles for sedimentation: 

\begin{enumerate} 
    \item Pectin splitting enzymes are used to modify pectin-protein particles. 
    \item Gelatine is active against phenolic compounds, forming hydrogen-bonded aggregates. 
    \item Bentonite can form aggregates with protein, removing excess gelatine. 
\end{enumerate}

\vspace{0.5em}
Heat treatment (pasteurization) is necessary to make juice storable due to enzymatic activities and microorganisms. Because fruit juices have a low pH, a relatively mild treatment is sufficient. The necessary time-temperature treatment is highly pH-dependent; apple juice requires only 10 seconds pasteurization at 90\textdegree C. Rapid warming and cooling via HTST treatment avoids heat damage.

\vspace{0.5em}
Processing causes changes in juice components. Anthocyanin content in black currant decreases significantly, with losses often occurring during the "Warming" step and pressing/clarification. Aromatic compounds like $\alpha$-pinene and ethyl butanoate also show a huge drop from berries to clear juice, with losses spread across processing steps, suggesting evaporation or loss to the press cake.
Most produced juice is stored and transported as concentrates, reducing storage capacity to 1/4-1/6. Concentration is typically achieved by evaporation combined with distillation of the aromatics fraction.


\section{RM L23 - Using Water Stress to Control Vegetative Growth and Productivity of Temperate Fruit Trees}

Although extensive evidence shows the negative impact of water stress on fruit yields, moderate water stress can reduce vegetative growth, which is often excessive, while simultaneously maximizing fruit size and quality. The long life cycle of fruit trees provides an opportunity for applying Regulated Deficit Irrigation (RDI). The RDI concept involves reducing water availability during specific growth phases when vegetative growth is highly active but fruit growth is less susceptible. The application of moderate water stress reduces vegetative growth primarily by inducing lower leaf water potential and reduced carbon exchange (photosynthesis). This reduction in overall CO$_2$ exchange is an early plant response to water stress \cite*{rm_12_L23_water_stress_control_vegetative_growth_productivity}.

\vspace{0.5em}
Fruit growth in fruit trees follows a double S-shaped pattern, notably in species like peach, apricot, and cherry, characterized by three developmental stages. Vegetative growth (shoots and root system) begins before and continues after the fruit growth period. RDI programs are ideally timed to exploit specific periods of fruit development. These periods include: 

\begin{enumerate} 
    \item The first critical period after fruit set, where stress could reduce total yield and fruit size potential. 
    \item The second period, which occurs during the lag phase (stage II of peach growth), separating the fruit growth periods that are driven by cell division and cell expansion, respectively. 
\end{enumerate} 

\vspace{0.5em}
Stress applied during the lag phase (Stage II) effectively suppresses shoot growth with only a minor impact on fruit growth, but a major impact on vegetative growth. This is because fruit growth and cell enlargement occur rapidly before and after the lag phase. Furthermore, postharvest water stress, applied after harvest (e.g., July to September in California), is used to suppress excessive vegetative growth. In nectarine, RDI applied from mid-June to mid-October achieved substantial savings of water without negatively affecting fruit size or quality, and postharvest RDI resulted in an increase in fruit soluble solids. In peach RDI experiments, a subsequent work demonstrated substantial water savings without productivity loss. However, RDI in pear resulted in reduced total yield, by 19\% to 29\% greater than control, and reduced fruit size.

\vspace{0.5em}
The physiological mechanism allowing RDI success is believed to involve the fruit's ability to adjust its turgor (cell size) by internal solute accumulation through osmotic adjustment. This is critical for cell expansion during the last phase of fruit growth. The degree, duration, and timing of stress are vital factors affecting RDI outcome. Trees in a stressed condition may be put under stress earlier in the season because they are already acclimated. Generally, moderate water stress reduces the vegetative growth while providing beneficial effects, often reducing the shoot/root ratio.


\section{\texorpdfstring{1\textsuperscript{st} RM for L24+25 - Leaf absorption of nutrients / Mineral nutrition}{1st RM for L24+25 - Leaf absorption of nutrients / Mineral nutrition}}

High hydrostatic pressures can be generated, leading to guttation (bleeding), typically 0.2-0.3 MPa in leafless birch, but limited to 0.1-0.2 MPa in leaves. Bleeding stops when leaves develop due to assimilation and water consumption, resulting in the transport of organic and inorganic compounds. Guttation occurs via exudation from hydathodes on leaf margins and tips, commonly observed in the cool temperate zone. Pruning of roots or shoots is suggested to reduce bleeding.

\vspace{0.5em}
The practice of foliar absorption involves applying mineral nutrients as aqueous sprays or suspensions to the leaves to supplement soil application. Leaves possess the capability to absorb mineral nutrients, although less readily than roots. The leaf epidermis is covered by the cuticle, a non-cellular lipoidal barrier that limits the uptake of water-soluble nutrients. The absorption process involves two main steps: diffusion across the cuticle and absorption by the underlying living cells. Nutrient uptake is a slow process, typically absorbing only 5-30\% of applied substances over several days. Uptake is dependent on formulation, pH, and cuticle status. No major difference exists in absorption rates between the upper and lower leaf surfaces. For red-skinned apples, absorption over 32 hours included 91.4\% for urea, 6.4\% for Fe-EDDHA, and 35.1\% for MnSO$_4$. Surfactants can increase absorption, especially for less surface-active compounds. Applying solutions when leaves are highly turgid or under cool, cloudy, wet conditions enhances uptake. Rapid evaporation under dry conditions decreases uptake.

\vspace{0.5em}
Foliar application is practical for rapid nutrient supply, especially to correct deficiencies of Mn or Zn, and to supply N in spring to support early growth. Pome and stone fruit often receive foliar nutrients 3-4 times per season. Foliar N application in spring can reduce pre-harvest fruit drop in apples. Table 5-2 illustrates that 'Cox's Orange Pippin' apple trees subject to full standard soil fertilization (FF) combined with full fertilization (FN) achieved the highest fruit weight (175 g), while P fertilization at 5\% (P-5\%) resulted in the lowest fruit weight (27 g).

\vspace{0.5em}
Mineral elements are essential for plants, playing roles in osmoregulation, membrane permeability, and enzyme systems. Elements are classified based on their mobility, which determines where deficiency symptoms first appear.

\vspace{0.5em}
Mobile elements (N,P,K,Mg,S) are easily translocated from older leaves to growing points, causing deficiency symptoms to appear in older leaves. Less mobile elements (Ca,B,Mo) move slowly, resulting in deficiency symptoms appearing in younger tissue, such as shoot tips and fruitlets.

\vspace{0.5em}
Nitrogen (N) deficiency reduces concentration in plant parts, leading to lighter green or purple coloration. High N supply enhances shoot growth and extends the growth season, delaying autumn senescence. Spring application of N promotes rapid shoot growth but may reduce flower differentiation. Calcium (Ca) is mainly transported through the xylem. Reduced Ca content can lead to reduced ovule development and lower flower quality, decreasing fruiting ability.


\section{\texorpdfstring{2\textsuperscript{nd} RM for L24+25 - Yield Components and Fruit Development in 'Golden Delicious' Aapple as Aaffected by the Timing of Nitrogen Supply}{2nd RM for L24+25 - Yield Components and Fruit Development in 'Golden Delicious' Aapple as Aaffected by the Timing of Nitrogen Supply}}

The Nitrogen (N) supply of 'Golden Delicious' apples was manipulated during the growing season by altering the N concentration in nutrient solutions or by urea sprays. Terminal shoot growth increased with the duration of “high” N supply, especially with early summer application, exhibiting a degree of correlation with the N concentration of the leaves. Blossom density decreased only at continuously low N supply when evaluated on the basis of the previous year's crop level. Fruit set was reduced when N in spur leaves, immediately after bloom, dropped below 2.8-3.0\%, which occurred with continuously low N, or with high N supply only in the early summer. Fruit growth at a fixed fruit/leaf ratio increased with N supply and growth variations were seen between years. The N supply during the early part of the fruit growth period was the more important factor. The yellow colouring of the fruit at certain fruit/leaf ratios was greater at low N levels or when the N supply was low during the latter part of the fruit growth period. Effects on fruit quality and yield components were small or experimentally inconsistent where the N-deficient treatments were excluded \cite*{RM_13_L24_25_nitrogen_supply_apple_yield}.

\vspace{0.5em}
The experiment utilized two-year-old 'Golden Delicious'/MM 104 trees in plastic pots, later transferred to 304 pots in the spring of 1970. A similar two-year-old 'Golden Delicious'/M 7a (a virus-tested selection of M 7, not identical to the EMLA clone) was planted in 15-l pots in the spring of 1971 and transferred to 304 pots in the spring of 1973. The experiment involved 12 replicated treatments, comprising combinations of N supply through the roots and urea spraying on the tops (Fig. 1). Cropping was not undertaken until 1972. N concentration in the leaves was affected distinctly by the N supply. N concentration remained low with the “low” N application alone (Treatment 9). “High” N supply in the roots in autumn and by urea spraying in spring (Treatment 7) gave high spring values and also prevented definite deficiency for the remainder of the season. The potassium (K) content of the leaves was highest with continuous “high” N supply (Table I). The concentrations of phosphorus (P) and N were high at “low” N application (Treatment 9) and decreased with the duration of high N supply (Table I).

\vspace{0.5em}
Regarding Vegetative growth, total dry matter production during the experiment period generally decreased, as the duration of “high” N supply was reduced, falling to about half of the maximum value at “low” N (Treatment 9) (Table II). Terminal shoot growth was particularly affected by the N supply and correlated well with the duration of “high” N supply (Fig. 3). Vigorous growth after additional N in June-July alone (Treatment 8) was partially due to reduced fruit set.

\vspace{0.5em}
Blossom density was negatively correlated between flower density and fruit/leaf ratio of the preceding growing season. Low N application (Treatment 9) (perhaps only half) was required to obtain a lower fruit/leaf ratio, compared to other treatments, to maintain a certain blossom density.
Fruit set was dependent on the N status of the trees, indicated by leaf values immediately after bloom. N concentration of spur leaves should be 2.8-3.0\% or more to ensure proper fruit set. Fruit set was low in Treatment 9 in all 3 years, and with “high” N during June-July (Treatment 8) in 2 of the years. N applied late autumn and in spring only (Treatment 7) conversely achieved adequate fruit set even when N level was comparatively low later in the season.

\vspace{0.5em}
Fruit growth increased inter alia on fruit/leaf ratio and thus on fruit size. A tendency for more rapid fruit growth existed with increasing duration of “high” N supply. Minimum fruit growth was found at “low” N throughout the season (Treatment 9).

\vspace{0.5em}
Fruit development (colour) shows that the degree of yellow colouring decreases with increasing fruit/leaf ratio of the tree, which also reduces average fruit size. “Low” N throughout the season (Treatment 9) yielded fruits of a stronger yellow colour. Continuous “high” N (Treatments 1, 2) resulted in greener fruits. Additional N supply applied late autumn and spring alone (Treatment 7) gave more yellow fruits than other treatments. N availability in early to mid-summer may favour vegetative growth at the expense of fruit, whereas in cropping trees, N status must be established in late autumn and spring to ensure proper fruit set. Leaf values below 1.8-2.1\% N have been proposed to ensure good fruit quality.


\section{RM for L25+32 - Pre-harvest factors influencing the quality of berries}

Quality of berries is a difficult concept to describe objectively. The concept of quality in berries can be defined as a set of agronomic/commercial, organoleptic, and nutritional qualities. Agronomic/commercial quality comprises characters belonging to the adaptation of the plant to specific cultivation, such as fruit size, plant yield, harvesting speed, and resistance to pests and diseases. Organoleptic quality includes characteristics generally related to attributes recognizable through the five senses of the consumer, such as fruit color, shape, acidity and sweetness, combined with flavor and aroma determined by volatile compounds. Nutritional quality is the ``hidden'' quality present in berry fruits, comprising all macro- and micro-nutrients, vitamins, and bioactive compounds. Recently, the nutritional value, intended as the content of bioactive compounds with healthy effects, has gained increasing priority in breeding programs \cite*{RM_14_L25_32_pre_harvest_factors_influencing_quality_berries}.

\vspace{0.5em}
Pre-harvest factors influencing berry quality include genetic factors, environmental factors, and agronomic factors. Genetic factors account for variability among different species and cultivars. The genetic background is the most important factor affecting fruit quality, particularly sensorial and nutritional quality. Environmental factors influence fruit quality through adaptability to different climatic conditions, such as latitude, soil conditions, production cycle, and light exposition. Agronomic factors are related to cultivation systems (open field or protected or soilless), fertilization, water stress and salinity, and fruit harvest.

\vspace{0.5em}
Plant yield is the amount of fruits with commercial value harvested for each production cycle. Fruit size is certainly one of the most important traits in new genotype selection. The fruit size is evaluated through the implementation of some different parameters: \begin{enumerate} \item Fruit Weight: indicates the average weight of a representative sample of fruits and refers to the dimension of the fruit; \item Fruit Length: indicates the average size on the longitudinal region of a representative sample of fruits; \item Fruit Diameter: indicates the average size around the equatorial region of a representative sample of fruits. \end{enumerate} Fruit color is a primary organoleptic trait, determined by pigments (anthocyanin, phenols, etc.) accumulated in the fruit skin cells. Fruit color is determined by the genotype but is also highly influenced by environmental conditions and maturation stage. Firmness indicates the resistance of the fruit to mechanical damages and is a priority for facilitating harvesting, transportation, and post-harvest management. Sweetness is indicated as Total Soluble Solids (TSS), expressed as \textdegree Brix, and acidity is expressed as Total Acidity (TA). The sweet-acid balance is measured through the Sugar/Acid (S/A) ratio. An acceptable strawberry flavor requires a minimum TSS of 7\% and a maximum TA of 0.8\%. Flavor is determined by VOCs (volatile organic compounds).

\vspace{0.5em}
Nutritional quality relates to the high content of phytochemicals, primarily phenolic molecules, dietary fibers, vitamins (A, C, E, and B complex), and essential macro- and micronutrients (phosphorus, potassium, calcium, magnesium, iron, manganese, copper, sodium, and aluminum). Anthocyanins, a subgroup of flavonoids, act as powerful antioxidants. Ellagic acid is a powerful healthy compound found in different berry species, representing 51\% of the total phenolic compounds in berries.

\vspace{0.5em}
The genetic background defines the capacity of a fruit to accumulate bioactive compounds. Wild species, such as F. vesca, often have higher levels of nutritional attributes (e.g., mean sugar concentrations twice or three times higher, higher total polyphenols) than cultivated species, despite potential loss of traits like fruit size and firmness. Plant age can influence fruit quality; higher sugar concentrations and yield were found in the second year of cultivation in F. x ananassa fruits.

\vspace{0.5em}
Environmental factors such as growing location influence nutritional quality; for example, strawberries in northern Europe showed higher dry matter, total acidity, and soluble solids content compared to southern Europe. Light exposure stimulates flavonoid biosynthesis, leading to elevated anthocyanins and flavonols content. Temperature also affects quality; high temperatures negatively affected black currant fruit quality by suppressing anthocyanins biosynthesis. Lower temperatures (e.g., 12\textdegree C) increased levels of flavonols and hydroxycinnamic acids in bilberry compared to 18\textdegree C. Agronomic practices influence phytochemical content. Organically grown strawberries possess higher levels of total phenolics, ellagic acid, and flavonols than conventional strawberries. Stress factors like minimum salinity or reduced water can induce a stress response that promotes phytochemical production, such as phenols. Deficit irrigation can influence fruit phenol content. Fruit harvest maturity is critical; while fully ripened berries maximize anthocyanins and aroma, picking at earlier stages, like white tip or three-quarter color in strawberries, maintains fruit firmness and phenolic/flavonoid concentration for longer storage.


\section{RM for L26+27 - Unravelling genetic diversity and cultivar parentage in the Danish apple gene bank collection}

Characterization of apple germplasm is important for conservation management and breeding strategies. Genotyping of plant genetic resources, such as the genus Malus, is crucial for gene banks to ensure preservation of maximum genetic diversity, as collections risk losing identity or accumulating duplicates. The largest collection of Danish apple (Malus domestica Borkh.) cultivars is maintained at the Pometum (University of Copenhagen, Denmark). This study genotyped a set of 448 Malus domestica accessions, primarily of local Danish origin, using 15 microsatellite markers. Ploidy levels were determined by flow cytometry. The study also included a reference set of cultivars, a private nursery collection (Assens), and a small set of Malus sieversii, Malus sylvestris and small-fruited, ornamental Malus cultivars. Plant material was collected as young leaves from vigorously growing shoots at the Pometum's gene bank collection. The ploidy level was determined using flow cytometer BD FACSAria™ Illu counting 5000 events and staining with 4',6-diamidino-2-phenylindole (DAPI) solution. For SSR genotyping, a set of 15 SSR markers was used. Parentage analysis for diploid samples was performed with ML-Relate, while parentage analysis for triploid samples was performed with Colony version 2.0.5.9 after converting microsatellite data to binary data \cite*{rm_15_L26_27_genetic_diversity_cultivar_parentage_danish_apple_gene_bank}.

\vspace{0.5em}
The microsatellite markers amplified 17-30 alleles per loci with an average degree of heterozygosity at 0.78. For the 448 M. domestica cultivars, 12 to 26 alleles per loci were amplified, with a mean of 19. The observed heterozygosity (H$_o$) was 0.78 and expected heterozygosity (H$_e$) was 0.81. We found 42 unique (private) alleles. The proportion of genotypes with unique alleles was larger in the Assens collection (19\%) than in the Danish collection and reference collection (7\% and 8\%, respectively).

\vspace{0.5em}
Of the 393 accessions whose ploidy levels were determined, 17\% were triploids. In the Danish cultivar collection, 22\% triploids were identified. We identified 104 (23\%) duplicate genotypes among all 448 M. domestica accessions, including colour sports. The Danish apple gene bank collection contained 10\% duplicate accessions, excluding the Assens collection duplicates. Duplicate genotypes were divided into subclones, such as phenotypically distinctive colour mutants that markers were unable to separate, or duplicates with name synonyms. The highest number of identical genotypes was found for Gravensteiner (15 subclones), followed by Almindelig Pigeon, Filippa, and Cox's Orange (all with 6 subclones).

\vspace{0.5em}
The SSR data allowed for studying putative relatedness such as parent-offspring relationships. We could infer first-degree relationships for many cultivars with previously unknown parentages. The cultivars Cox's Orange and 'Pigeonnet blanc d'hiver' have the highest number of putative offspring relatives. The cultivar Ingrid Marie was confirmed to be an offspring of Cox's Orange, and Cox's Pomona was unambiguously identified as the other parent, resulting in the hybridization event Cox's Orange$\times$Cox's Pomona.

\vspace{0.5em}
STRUCTURE analysis provided no evidence for a genetic structure but allowed us to present a putative genetic assembly that was consistent with both PCA analysis and parental affiliation. A putative assemble of five genetic clusters was shown, though no more than 21\% of genotypes had strong assignment (>0.80) to a specific group. The lack of genetic structure is supported by the Bayesian model-based approach and delta K values in a very low and tight range (delta K < 1.5). The largest genetic differentiation was found between groups 2 and 5 (F$_{st}$=0.056). Group 5 (derivatives of Cox's Orange) is the genetically most distinctive group and the least diverse (allelic richness 9.60), while groups 1 and 4 are the most diverse groups (allelic richness 14.73 and 12.20, respectively). The PCA plot indicates a cluster of Cox's Orange and first-degree relatives and no distinction between Danish cultivars and the reference collection.

\vspace{0.5em}
The Danish cultivar collection contains many unique accessions and considerable genetic diversity, making it a valuable resource within the European apple germplasm. The fingerprints can be used for cultivar identification and future management of apple genetic resources.

\section{RM for L29 - The effect of source-sink status on flowering and the performance of everbearing strawberry cultivars}

Everbearing strawberry cultivars offer the potential to extend the growing season and boost productivity through recurring flowering. Cultivation of everbearing cultivars is high-cost and intensive, necessitating high fruit yields for economic viability. A major challenge is the great fluctuations in the cropping pattern, which typically involves an initial flush of berries, followed by a lag-period with a significant decrease in production, and a subsequent second flush of berries. This study aimed to elucidate the effect of the source-sink status of the plant on the flowering pattern and performance of the everbearing cultivars 'Favori' and 'Murano'. It was hypothesized that the level of continuous floral initiation and flower development is dependent on the source-sink status of the plant, and manipulating this balance could potentially be used as a tool to gain an optimal or desired harvest profile and a more stable production.

\vspace{0.5em}
The methodology included Flower Mapping to determine the starting generative potential for the first flush. For the first flush, 'Favori' showed a greater potential with an average of 13,2 flower trusses per plant, while 'Murano' had 8,2. Based on this, 80 plants of each cultivar were divided into sections of 20 plants to be exposed to four source-sink treatments based on the removal of flower trusses. The treatments were defined by Growing Degree Hours (GDH) thresholds and allowed maximum Leaf Area Index (LAI) of 3. Flowers emerging until 3.500 GDH hours were removed for all treatments except the control to establish good vegetative growth. The treatments applied until 16.500 GDH were:

\begin{enumerate} 
    \item Control: No trusses removed until 3500 GDH; All trusses allowed 3500-16500 GDH. 
    \item Treatment 1: All trusses removed until 3500 GDH; All trusses allowed 3500-16500 GDH. 
    \item Treatment 2: All trusses removed until 3500 GDH; 6 trusses allowed 3500-16500 GDH. 
    \item Treatment 3: All trusses removed until 3500 GDH; 3 trusses allowed 3500-16500 GDH. 
\end{enumerate}

\vspace{0.5em}
Runners were removed upon the emergence of the first node. Leaf management was applied by keeping the total leaf area below a LAI of 3 to maintain a uniform source capacity. Harvesting defined three phases: First Flush (FF), Lag-Phase (LP), and Second Flush (SF). Quality parameters included individual sugars and organic acids, pH, total acidity, and Brix measurements (\% soluble solids).

\vspace{0.5em}
Results show that removal of flower trusses belonging to the first flush (F2, F3, M2, M3) has marketable and significant effects on yield and yield parameters. The general trend is that increased removal of trusses belonging to the first flush results in a yield increase. Since the number of berries per plant is similar across the treatments (excluding the control), the yield increase is attributed to increased berry weight. This sink relief provided yield compensation in the form of increased berry weight during the first flush. There is a contrasting clear and general trend of increased yields as the response to increased removal of flowers during the lag-phase and second flush. New initiation of flowers is supported as the likely explanation for the similar number of berries despite initial removal. For Favori, removing only very early trusses emerging before 3.500 GDH (F1) seemingly had the greatest influence on both yields and average berry weight compared to the control (FC).

\vspace{0.5em}
Regarding quality, the increased flower truss removal during the first flush yielded berries with an increasingly higher sugar content for both cultivars. This suggests that during this first flush, the berry development and dry matter accumulation are strictly source-limited, and sink relief promotes development in the remaining fruit sinks. The redistribution of yield to the lag-phase and second flush had no negative impact on assimilate partitioning in those phases for the flower truss removal treatments. No trends were observed in the change of concentrations of organic acids, titratable acidity, or pH.

\vspace{0.5em}
Findings confirm that continuous flower initiation is resource limited. Altering source-sink status through early truss removal effectively manipulates flowering and fruiting patterns, leading to a more balanced and stable cropping with increased yields during the normal slowdown periods. Relieving internal competition enhances dry matter partitioning, increasing sugar concentrations in remaining fruits and improving crop quality.


\section{RM for L30 - The European Blueberry (Vaccinium myrtillus L.) and the Potential for Cultivation. A Review}

The European blueberry (EB), also known as bilberry (Vaccinium myrtillus), is an important wild berry in Northern Europe, highly demanded by the processing industry due to its delicious taste and high dietary value. EB is a deciduous woody dwarf shrub typically found in spruce- and pine-dominated heath forests of medium fertility. It grows on better-drained acid soils. No efforts towards the domestication of EB have been made; it is still harvested in forest fields without cultivation. The successful domestication of the sweet lowbush blueberry (V. angustifolium), which shares similar growth and morphological characteristics, suggests opportunities to increase the highly variable yearly EB yield (potential close to 2 tons per hectare) through practices such as fertilization, irrigation, cutting trees, and weed control \cite*{rm_16_L30_european_blueberry_potential_cultivation}.

\vspace{0.5em}
The genus Vaccinium belongs to the family Ericaceae. EB belongs to the section Myrtillus, while American blueberries belong to the section Cyanococcus. EB is a shrub varying in size from approximately 5 to 90 cm, with a chromosome number of 2n=24. Unlike American blueberries, which produce an inflorescence containing a cluster of berries, the EB grows one (rarely two) flower from a leaf axil bud. EB growth pattern is rhizomatous, forming open colonies. Rhizomes can be up to 5.5 m long, and one single clone may occupy several square meters. Aerial shoots, or ramets, typically live for less than 6 years, although ages up to 34 years have been recorded.

\vspace{0.5em}
EB flowers are born singly (rarely in pairs) in the axils of leaves on one-year-old twigs, which are located on aerial shoots at least three years old. Flower initials develop the year before flowering and overwinter in a dormant bud. A biannual flowering pattern, where EB flowered heavily every second year, has been observed. EB is obligately insect pollinated. The rate of selfing through geitonogamy is significant due to the extent of large EB clones. Cross-pollinated flowers matured approximately four times as many seeds and aborted fewer seeds than self-pollinated flowers, suggesting pollen limitation and inbreeding depression at the seed stage. Berry fresh weight is positively related to the total weight of seeds and the number of seeds.

\vspace{0.5em}
The most significant characteristic of blueberries is their high content of beneficial nutrients and bioactive phytochemicals. EB has blackish fruit flesh, distinguishing it from American blueberries, which have whitish flesh. EB berries are characterized by 15 anthocyanins and higher levels of total anthocyanins, phenols, and antioxidants compared to highbush varieties. V. myrtillus shows a relatively higher degree of vacuolization at full maturation, making them more suitable for local markets rather than long-term storage or transport.

\vspace{0.5em}
Regarding enhancement of fruit yield by management, mineral nutrition, especially when water is sufficient, increased fruit yield. Fertilization with 150 kg ha\-1 of ammonium nitrate generally increased fruit yield in the first years after implementation. EB is a calcifuge plant and its growth is reliant on the symbiosis with ericoid mycorrhiza for nutrient availability, especially in nutrient poor environments. EB growth is markedly dependent on the P-level in the soils. Optimal conditions for EB include high humidity and minimal shade. Climate has a decisive impact, with issues noted including lethal injury during mild winters in the absence of snow cover, and frost during bloom, which can be rate limiting. Extreme winter warming reduced flower production by more than 80\%. Clear-cutting severely reduces vegetative growth, shoot survival, ground cover, and fruit production. When attempting domestication on farmland, adjusting soil pH, often by adding sulphur or organic matter of low pH, may be necessary. Vegetative propagation for homogeneous plant material currently has a low success-rate.


\section{\texorpdfstring{1\textsuperscript{st} RM for L31 - Bioactive Compounds in Cranberries and their Biological Properties}{1st RM for L31 - Bioactive Compounds in Cranberries and their Biological Properties}}

Cranberries are healthy fruit that contribute color, flavor, nutritional value, and functionality. The North American cranberry (Vaccinium macrocarpon) is one of only three fruits native to America. Public interest has been rising due to potential health benefits linked to numerous phytochemicals present in the fruit: the anthocyanins, the flavonols, the flavan-3-ols, the proanthocyanidins, and the phenolic acid derivatives. The presence of these phytochemicals appears to be responsible for the cranberry property of preventing many diseases and infections, including cardiovascular diseases, various cancers, and infections involving the urinary tract, dental health, and Helicobacter pylori-induced stomach ulcers and cancers. The predominant bioactive compounds found in cranberries are the flavonols, the flavan-3-ols, the anthocyanins, the tannins (ellagitannins and proanthocyanidins), and the phenolic acid derivatives.

\vspace{0.5em}
Cranberries and cranberry products are known for their elevated concentration in total polyphenols. On a serving size basis for all cranberry products, the order of total polyphenols is: frozen > 100\% juice > dried > cocktail > sauce > jellied sauce. More specifically, cranberries are rich in anthocyanins, in tannins (ellagitannins and proanthocyanidins), and have significant concentrations of the flavonoids—flavonols and flavan-3-ols.

\vspace{0.5em}
The cranberry is one of the leading fruit sources of flavonols on a weight basis, with an average content of 20-30 mg/100 g fresh fruit weight. Approximately 75\% of the cranberry flavonols consist of quercetin glycosides (quercetin 3-O-galactoside). Flavan-3-ol content averages 7 mg/100 g for a ripe fruit at harvest. Cranberries have a high content of oligomeric and polymeric pigments, referred to as condensed non-hydrolyzable tannins or proanthocyanidins. The proanthocyanidin content of raw cranberries can average 410 mg/100 g fresh fruit weight. Cranberry proanthocyanidins tend to occur as tetramers to decamers, often with A-type linkages between epicatechin units. Phenolic acids, including derivatives of hydroxycinnamic acid (HCA) and hydroxybenzoic acid (HBA), contribute to the characteristic and unique flavor of berries.

\vspace{0.5em}
Regarding biological properties, the ability of American cranberry juice (Vaccinium macrocarpon) to protect the urinary tract from adherence of urinopathogenic bacteria such as Escherichia coli is well-known. The French Food Safety Authority issued a health claim stating that "cranberry proanthocyanidins, in a daily dose of 36 mg, help reducing the adhesion of certain E. coli bacteria to the urinary tract". Cranberry antimicrobial activity was also demonstrated toward other illness-causing pathogenic bacteria, including Helicobacter pylori, Salmonella, Staphylococcus aureus, and Campylobacter. The polymeric tannins, and in particular, proanthocyanidins consisting primarily of epicatechin tetramers and pentamers with at least one A-type linkage, seem to be the protecting element against pathogenic bacteria.

\vspace{0.5em}
Considerable in vitro evidence exists, showing that cranberry phenolics are powerful antioxidants. Cranberry phenolics exhibit free radical-scavenging properties against superoxide radical (O$_2^-$), hydrogen peroxide (H$_2$O$_2$), hydroxyl radicals ($\cdot$OH), and singlet oxygen (1O$_2$). Analysis found that the anthocyanins contributed 54.2\% of the antioxidant activity of cranberries, whereas the flavonols contributed 34.6\% of the observed antioxidant activity.

\vspace{0.5em}
The compounds also exhibit antimutagen and anticarcinogen properties. Cranberry phenolic compounds modulate cell signalling pathways, potentially interfering with tumor development. Total cranberry extract (200 $\mu$g/mL) inhibited the proliferation of human oral, colon, and prostate tumor cells in vitro. The tumor inhibition by cranberry is likely to involve synergistic activities between the cranberry major phytochemicals—the anthocyanins, the flavonols, and the proanthocyanidins.


\section{\texorpdfstring{2\textsuperscript{nd} RM for L31 - Antioxidant Polyphenols from Tart Cherries (Prunus cerasus)}{2nd RM for L31 - Antioxidant Polyphenols from Tart Cherries (Prunus cerasus)}}

The Montmorency (Prunus cerasus) variety of cherries constitutes >95\% of tart cherry cultivations in Michigan and the United States. However, Balaton tart cherry (P. cerasus) is a new cultivar being planted commercially, which is regarded as a better variety due to higher anthocyanin contents. Flavonoids, a group of widely distributed polyphenolic compounds, have been reported to act as antioxidants in biological systems, sometimes having activity similar to that of $\alpha$-tocopherol, vitamin E. This study described the isolation, identification, and efficacy of antioxidant polyphenolic compounds from Balaton and Montmorency tart cherries. Pitted and frozen Montmorency and Balaton tart cherries were obtained from commercial growers in Traverse City, MI \cite*{rm_17.2_L31_bioactive_compounds_cranberries_biological_properties}.

\vspace{0.5em}
Montmorency and Balaton tart cherries were lyophilized and sequentially extracted with hexane, ethyl acetate, and methanol. The extraction of 200 g of lyophilized Balaton tart cherries yielded crude extracts of 0.42, 1.49, and 116.3 g, respectively. Methanolic extracts of dried Balaton and Montmorency tart cherries inhibited lipid peroxidation induced by Fe$^{2+}$ at 25 ppm concentrations. The percent inhibitions of hexane, EtOAc, and MeOH extracts of Balaton tart cherry were 9.9, 28.3, and 13.6\%, respectively. Further partitioning of the methanol extract with EtOAc yielded Fraction I, which inhibited lipid peroxidation by 76\% at 25 ppm. Fraction I from Balaton cherries contained the most active antioxidant compounds, which were purified by medium-pressure liquid chromatography (MPLC), preparative silica gel TLC, and HPLC.

\vspace{0.5em}
Purification of Fraction I afforded eight polyphenolic compounds, characterized by $^1$H and $^{13}$C NMR experiments. The purification yielded: 

\begin{enumerate} 
    \item 5,7,4'-trihydroxyflavanone (naringenin) 
    \item 5,7,4'-trihydroxyisoflavone (genistein) 
    \item chlorogenic acid 
    \item 5,7,3',4'-tetrahydroxy-flavonol-3-rhamnoside (quercetin 3-rhamnoside) 
    \item 5,7,4'-trihydroxyflavonol 3-rutinoside (identified as kaempferol 6''-O-$\alpha$-L-rhamnopyranosyl-$\beta$-D-gluco-
    pyranoside) 
    \item 5,7,4'-trihydroxy-3'methoxyflavonol-3-rutinoside (assigned as rhamnazin 6''-O-$\alpha$-L-rhamnopyranosyl-$\beta$-D-glucopyranoside) 
    \item 5,7,4'-trihydroxyisoflavone-7-glucoside (genistein 7-glucoside) 
    \item 6,7-dimethoxy-5,8,4'-trihydroxyflavone 
\end{enumerate} 

\vspace{0.5em}
Compounds 6 and 8 are reported for the first time from tart cherries.
Antioxidant assays revealed that 6,7-dimethoxy-5,8,4'-trihydroxyflavone (8) is the most active at 10 $\mu$M concentrations, followed by quercetin 3-rhamnoside (4), genistein (2), chlorogenic acid (3), naringenin (1), and genistin (7). The superior antioxidant activity of compound 8, despite lacking a 3-hydroxyl group and having only one hydroxyl group on the B ring, was probably due to the hydroxyl and two methoxy groups in ring A. The inhibitory effect of flavonoids on Fe$^{2+}$ lipid peroxidation is due to their ability to chelate Fe$^{2+}$ and their free radical scavenging activities. When assayed independently at the concentrations present in a 25 ppm mixture, compound 8 showed 110\% inhibition and compound 3 showed 76.7\% inhibition of Fe$^{2+}$ -induced lipid peroxidation. The mixture of compounds 1-8 gave 89.6\% inhibition at 25 ppm, but the sum of the antioxidant activities of the individual compounds was higher than that of the mixture, suggesting some purified compounds are more effective inhibitors of lipid peroxidation when tested alone. Compounds 8 and 3 were identified as the most active components in the mixture and probably in Fraction I.


\section{\texorpdfstring{3\textsuperscript{rd} RM for L31 - Antioxidant Polyphenols from Tart Cherries (Prunus cerasus)}{3rd RM for L31 - Antioxidant Polyphenols from Tart Cherries (Prunus cerasus)}}

An overwhelming body of research has firmly established that the dietary intake of berry fruits has a positive and profound impact on human health, performance, and disease. Commonly consumed berry fruits in North America include blackberry (Rubus spp.), black raspberry (Rubus occidentalis), blueberry (Vaccinium corymbosum), cranberry (Vaccinium macrocarpon), red raspberry (Rubus idaeus), and strawberry (Fragaria $\times$ ananassa). Other popularly consumed berry fruits worldwide include arctic bramble (Rubus articus), bilberries (Vaccinuim myrtillus), black currant (Ribes nigrum), and lingonberries (Vaccinium vitis-idaea). The International Berry Health Benefits Symposium, a series of biennial conferences initiated in 2005, was organized to investigate the latest scientific research related to berry consumption and human health \cite*{rm_17.3_L31_bioactive_compounds_cranberries_biological_properties}.

\vspace{0.5em}
The various biological properties of berry fruits have been largely related to their high levels and wide diversity of phenolic-type phytochemicals. Both lipophilic (minor) and hydrophilic (major) phytochemicals are found in berries, but the latter class has been largely implicated in the bioactivities. Berry phenolics include flavonoids (anthocyanins, flavonols, and flavanols), tannins [condensed tannins (proanthocyanidins) and hydrolyzable tannins (ellagitannins and gallotannins)], stilbenoids, and phenolic acids.

\vspace{0.5em}
Among berry phenolics, anthocyanins (pigments that account for attractive colors) are best studied and possess a wide range of bioactivities, including antioxidant, anticancer, and anti-inflammatory properties. The specific class of tannins present varies considerably among berries: blueberries and cranberries contain predominantly proanthocyanidins, whereas blackberries, black raspberries, red raspberries, and strawberries contain predominantly ellagitannins. For example, the bacterial antiadhesive properties observed for the cranberry are accounted for by its oligomeric proanthocyanidins, which possess an A-type structural linkage. Anthocyanins were previously reported to cross the blood-brain barrier of aged rats and localize in various brain regions, important for learning and memory.

\vspace{0.5em}
Although berry phenolics are best known for their ability to act as antioxidants, the biological activities exerted in vivo extend beyond antioxidation. Berry phytochemicals regulate the activities of metabolizing enzymes; modulate nuclear receptors, gene expression, and subcellular signaling pathways; and repair DNA oxidative damage. It is noted that the complementary, additive, and/or synergistic effects resulting from multiple phytochemicals are believed to be responsible for their wide range of observed biological properties, rather than these effects being due to a single constituent alone.

\vspace{0.5em}
Symposium discussions covered recent research on identifying berry phytochemicals, elucidating cellular and molecular mechanisms of actions, and assessing the bioavailability, metabolism, and tissue distribution of berry phenolics. Oral presentations focused on selected chronic human diseases that show promise in being positively affected by berry consumption, including heart health and cardiovascular disease, neurodegenerative and other diseases of aging, obesity, and certain human cancers, such as esophageal and gastrointestinal cancers. Discussions also addressed the effects of berry consumption on symptoms of the metabolic syndrome and on human performance enhancement.

\vspace{0.5em}
Future research should emphasize the interdisciplinary cross-fertilization of basic and clinical sciences to culminate in translational research. Focus on nutrigenomics (effects of nutrients on the genome, proteome, and metabolome) and nutrigenetics (effects of genetic variation on the interaction between diet and disease) is essential. Future studies must focus on the metabolomics of berry phenolics, evaluating whether metabolites (such as glucuronidated, sulfated, and methylated derivatives) formed in vivo accumulate within target tissues and act as "pro-drugs". Evaluating whether biological effects are enhanced by the complex interactions of multiple components within the food matrix is also necessary.


\section{RM for L32 - Deficit Irrigation as a Strategy to Save Water: Physiology and Potential Application to Horticulture}

Water is an increasingly scarce resource globally, and irrigated agriculture accounts for about two-thirds of the total fresh water diverted to human uses. Improving crop water-use efficiency (WUE), expressed as the ratio of harvested yield to water use, is a major concern. Deficit irrigation (DI) strategies have emerged as complementary approaches to increase WUE, deliberately allowing crops to sustain some degree of water deficit with only marginal yield decrease and significant reduction of irrigation water. The two main deficit irrigation strategies, based on crop physiological knowledge, are regulated deficit irrigation (RDI) and partial rootzone drying (PRD) \cite*{rm_18_L32_deficit_irrigation_strategy_save_water}.

\vspace{0.5em}
The foremost principle of RDI is that plant sensitivity to water stress is not constant during the growth cycle. In RDI, water is supplied to maintain plant water status within certain limits of deficit during phases where fruit growth is least sensitive to water reductions, potentially benefiting WUE, increasing water savings, and improving harvest quality. PRD involves exposing roots to alternate drying and wetting cycles, enabling plants to grow with reduced stomatal conductance but without overt signs of water stress. PRD operates based on root-to-shoot chemical signaling, where dehydration promotes the synthesis of hormonal signals, such as abscisic acid (ABA), in the drying roots. ABA travels to the leaves via the transpiration stream, reducing stomatal conductance, decreasing water loss, and increasing WUE.

\vspace{0.5em}
Stomatal regulation is a central process in determining WUE. Stomatal closure is among the earliest plant responses to drought. Stomata are sensitive to environmental signals (light, humidity, temperature, CO$_2$) and physiological signals (phytohormones, calcium). Stomatal closure responds earlier to soil water content than to leaf turgor. Xylem sap pH becomes more alkaline in response to soil drying, leading to an accumulation of ABA in the leaf apoplast adjacent to guard cells, inducing stomatal closure.

\vspace{0.5em}
Deficit irrigation has proven successful for several horticultural crops, particularly fruit crops. Grapevine (Vitis vinifera L.) is well adapted to deficit irrigation. PRD and conventional DI (irrigation at 50\% ETc) promoted intrinsic WUE (A/gs ratio) and long-term WUE ($\delta ^{13}$C increase) compared to full irrigated grapevines (FI, 100\% ETc). PRD also reduced vegetative vigor, potentially improving fruit quality due to better light exposure. In deciduous fruit trees, DI strategies mainly aim to reduce vegetative growth with minor changes to fruit development. DI and PRD in apple saved 45\% to 75\% water,of ten without affecting yield or fruit size, and advance dripening, increased flesh firmness, and increased total soluble solids (TSS).RDI in peach generally increased WUE and reduced vegetative growth without negative yield effects. Incitrus, DI(60–66\% of control) increased TSS and titrable acidity (TA), though fruit diameter reduced by about 10\%. For soft fruits, PRD (up to 50\% reduction) in raspberries did not reduce yields and resulted in higher WUE. However, in strawberries, DI and PRD (at 60\% ETc) decreased fresh berry yield and individual berry weight, while saving about 40\% of irrigation water. Vegetable crops generally tend not to cope as well as fruit crops. For tomato, DI can improve quality (e.g., higher soluble sugars and color intensity) but may result in pronounced yield loss (e.g., 60\%), attributed to flower abortion. However, PRD for processing tomato from fruit set until harvest resulted in significantly higher TSS and saved up to 25\% water.Inpotato,PRD(50-70\% ETc) maintained tuber yields and increased irrigation water use efficiency by 60\%.

\vspace{0.5em}
Deficit irrigation strategies also show commercial potential for ornamentals by reducing excessive growth and water consumption by 50\% to 90\%. Future research must focus on the combination of deficit irrigation with practices like mulching or protected cultivation, and the use of rootstocks adapted to water stress, to further improve WUE and minimize yield or quality losses.


\section{RM for L33 - CHAPTER 8 Vitamins, aroma compounds and colors}

The synthesis of aroma compounds and colors is almost exclusively linked to the progression of maturation. Vitamin C comprises ascorbic acid and its oxygenated form, dehydro-ascorbic acid, with ascorbic acid especially found in fruits. The content of Vitamin C varies widely with the fruit species and may have significant differences between varieties. The developmental stage and level of light exposure are probably the factors with the strongest effect on Vitamin C content. The majority of Vitamin C appears to have been formed when maturation begins. During ripening, the content may remain almost unchanged (strawberries and blackcurrants), decrease (blackcurrant and sour cherry), or increase (apple). During storage and postharvest maturation of apples, there is a clear decrease in Vitamin C content. In both apple and sour cherries, a higher concentration has been found in small compared to larger fruits within a tree, which is explained by the small fruits being behind in development. The light has an influence on Vitamin C content. Illuminated fruits or fruit tissues have a slightly higher content than in shade in both apple and sour cherries, and the content is higher on the south than on the north side of the tree. The colored parts of an apple have a higher content than the uncolored parts. The content is highest in the peel. Impacts found from nitrogen supply, pruning and rootstock on Vitamin C content are explained as indirect effects through the effects on the illumination of the fruits or degree of development. Ascorbic acid content in apples, averaged across five varieties (mg / 100 g fresh weight), demonstrates this concentration gradient: 

\begin{enumerate} 
    \item Peel (epidermis) 70 
    \item Chlorophyllic cell layer under peel 21 
    \item Outer portion of fruit meat 11 
    \item Inner part of fruit meat 6 
    \item Core house (without cores) 5 
\end{enumerate}

\vspace{0.5em}
During maturation, major color changes occur: green changes to more yellow shades, and red colors are formed in many varieties. Young fruits contain chlorophyll (a and b) in the (green) chloroplasts. The yellow-reddish colorants in the fruit are due to carotenoids, found in chloroplasts and chromoplasts. Carotenoids are built up of isoprene units (CH=C(CH$_3$)-CH=CH$_2$) and are often divided into carotenes (hydrocarbons, C$_{40}$H$_{56}$) and xanthophylls (oxygenated substances derived from carotenes). $\beta$-carotene is the most common carotene, and cryptoxanthene, zeaxanthin, antheraxanthin and violaxanthin are common xanthophylls. When fruits become yellow during maturation, it is primarily due to chlorophyll decomposition. The green color of chlorophyll covers the yellow carotenoids, making the fruit appear more yellow, even if the carotenoid content does not rise or possibly falls.

\vspace{0.5em}
The red-blue dyes are anthocyanins, which are water soluble and found in the cell sap. They behave as indicators from reddish shades in acid through colorless to blue at pH>approx.6. Anthocyanins chemically belong to flavonoids, which typically occur in conjunction with a sugar moiety, i.e., as glycosides. The aglycone is called anthocyanidin; Cyanidinin is a common anthocyanidin. Along with galactose, cyanidinin forms idaein, which is common in apples.

\vspace{0.5em}
For apples, the development of both yellow and red color is enhanced by increasing leaf / fruit ratio (e.g., 10 leaves per fruit yield 23\% red color, while 75 leaves per fruit yield 58\% red color) and increasing source activity, similarly to sugar content. Nitrogen inhibits the development of both yellow and red color. Light has a decisive and specific effect on the formation of over color (red) on apples, likely activating enzymes, including phenylalananine ammonia lyase (PAL), which participates in the anthocyanin synthesis. Good light conditions in the outer parts of the tree increase color formation significantly. Temperature affects color development. A common rule suggests red color development is promoted by sunny days and cold nights in the fall. Studies indicate that early in the ripening process, color formation is promoted by light and low temperature. However, later, during maturation, color formation gradually happens best in light and at relatively high temperatures. Sugar accumulation seems to be a necessary prerequisite for anthocyanin formation. In other fruit species, color development is also primarily related to the maturation process. Light does not have the same decisive significance for anthocyanin formation as in apples, as it occurs in many other fruits even in shadow, though color formation may be slower or later.


\section{\texorpdfstring{1\textsuperscript{st} RM for L34 - Using the Streif Index as a Final Harvest Window for Controlled-atmosphere Storage of Apples}{1st RM for L34 - Using the Streif Index as a Final Harvest Window for Controlled-atmosphere Storage of Apples}}

A final harvest window (FHW), expressed as Streif Index coefficients, was developed for identifying maximum fruit quality for strains of 'McIntosh', 'Cortland', and 'Jonagold' apples (Malus $\times$ domestica Borkh.) following 8 months of controlled-atmosphere (CA) storage. Determining an optimal harvest period is critical for growers to maximize poststorage fruit quality and minimize losses. The Streif Index is comprised of three maturity measurements: firmness, percentage soluble solids concentration (SSC), and starch conversion. The Index has been successfully used to estimate the time of optimum harvest for various apple cultivars in Germany, the Netherlands, Hungary, and Poland. Previous research utilized the Streif Index to identify a single optimum harvest date; this study sought to use the Streif Index to delimit a harvest window—a period of a few days—that more realistically parallels fruit maturation and assesses when to conclude harvest to exclude over-mature fruit from long-term storage \cite*{rm_19_L34.1_streif_index_final_harvest_window_controlled_atmosphere_storage_apples}.

\vspace{0.5em}
Experimental data were collected in the Annapolis Valley of Nova Scotia over three consecutive years (1995-97) on five apple strains: Redmax, Marshall, and Summerland 'McIntosh'; Redcort 'Cortland'; and Wilmuta 'Jonagold'. Samples were collected during nine preharvest intervals (twice per week) and four weekly harvest periods. Starch-to-sugar conversion was evaluated on a 1 (100\% starch) to 9 (0\% starch) scale using the McIntosh Starch Test Guide. Firmness (N) was assessed using a Ballauf penetrometer with an 11-mm-diameter tip, and juice SSC was determined with a hand-held refractometer. Titratable acidity (TA) was expressed as malic acid equivalents (g $\cdot$ L$^{-1}$).

\vspace{0.5em}
The Streif Index coefficient was calculated as the quotient of [firmness/(SSC $\times$ starch index)]. The relationship between the Streif Index (dependent variable) and day of year (independent variable) was derived by negative first-order linear regression equations. Apples from the four harvest periods were stored in standard CA storage (SCA) for 8 months at 3.5\textdegree C (McIntosh and Cortland) or 0\textdegree C (Jonagold), with 2.5 kPa O$_2$ and 4.5 kPa CO$_2$, followed by a 7-day shelf-life test at 0\textdegree C and 5 days at 20\textdegree C. Poststorage quality data—including firmness, SSC, TA, percentage of healthy fruit, and sensory data—were categorized and additively combined to produce an overall fruit quality rating scale. The final harvest (i.e., day of year) was identified as that which directly preceded at least a 10\% drop in the poststorage fruit quality rating compared with the first harvest rating.

\vspace{0.5em}
The FHW was calculated as the 3-year final harvest mean, with boundaries determined by the standard deviation of the mean, and expressed as Streif Index coefficients via the regression equations. The lower to upper FHW boundaries were determined as:

\begin{enumerate} 
    \item 4.18 to 5.34 for Redmax 
    \item 4.12 to 5.46 for Marshall 'McIntosh' 
    \item 4.51 to 5.68 for Summerland 'McIntosh' 
    \item 5.23 to 5.99 for Redcort 'Cortland' 
    \item 1.38 to 2.34 for Wilmuta 'Jonagold' 
\end{enumerate}

\vspace{0.5em}
The coefficients of simple determination (r$^2$) for the negative first-order regression models ranged from 0.71 to 0.88. The significantly lower Index values for 'Jonagold' result from the pattern of starch-to-sugar conversion, as cultivars that show rapid declines in starch content tend to display lower Streif Index values at optimal maturity. The Streif Index appears not to be strongly affected by year-to-year or orchard-to-orchard differences within a geographical region of similar climate. The practical utility of the Streif Index method lies in the ease with which apple fruit maturity at harvest can be evaluated for its suitability for long-term CA storage. For long-term CA storage (e.g., 8 months), maximum poststorage fruit quality will be obtained when harvest is concluded within the specified FHW.


\section{\texorpdfstring{2\textsuperscript{nd} RM for L34 - Fruit Ripening and Postharvest}{2nd RM for L34 - Fruit Ripening and Postharvest}}

Fruit thinning is required to avoid excessive fruit load that inhibits flower differentiation, potentially causing alternate bearing. Mechanical thinning, such as using the Darwin machine, or chemical thinners amplify natural abscission. In apple, thinning must be completed within 30 days after full bloom, when the 'king' fruit reaches a maximum cross diameter of 15 mm. The optimal leaf/fruit ratio in apple may range between 20 and 40.

\vspace{0.5em}
Fruit development follows two main models: a single sigmoid curve typical of pome fruit, or a double sigmoid curve characteristic of drupe and berry. In peach, fruit growth is divided into four developmental stages (S1, S2, S3, S4). S1 involves cell division/expansion, while S3 involves only expansion. The duration of the cell division phase (S1) significantly impacts the final fruit density, which affects storability and nutritional value. S2 in drupes is characterized by pit hardening. Hormones such as auxins, cytokinins, and gibberellins control the cell cycle during the cell division stage. Auxins, cytokinins, and gibberellins are associated with high levels during the cell division stage, while auxin increases again at S3, linking to both exponential growth phases. Ethylene regulates cell expansion processes.

\vspace{0.5em}
Physiological fruit drop is a self-regulatory mechanism. Abscission induction in apple fruitlets is linked to correlative inhibitions between shoot and fruits, which trigger a nutritional stress, the appearance of ROS, isoprene emissions, and an increase of ABA levels in the fruit cortex. AZ activation (iii) is likely caused by a decrease of Polar Auxin Transport (PAT) from the fruit, enhancing ethylene sensitivity. AZ activation leads to cell separation via the hydrolytic digestion of middle lamella and cell wall, mediated by abscission-specific polygalacturonase and cellulase genes.

\vspace{0.5em}
Assimilate transport and storage regulate sink strength and competition among plant organs. Sucrose is the main translocated carbohydrate, but species like apple, peach, and cherry (Rosaceae) also transport sorbitol. Phloem unloading can be symplastic (following a concentration gradient) or apoplastic (requiring specific membrane transporters). In grape berry, the phloem unloading shifts from symplastic (early development) to apoplastic (progression of growth/ripening), characterized by the hydrolysis of sucrose into glucose and fructose.

\vspace{0.5em}
Ripening is a complex physiological syndrome. Fleshy fruits are categorized as climacteric (e.g., apple, peach) or non-climacteric (e.g., grape, strawberry). Climacteric fruits show an increase in respiration and ethylene biosynthesis. Ethylene concentrations as low as 0.1-1.0 $\mu$l l$^{-1}$ can activate ethylene-dependent processes in climacteric fruits. Ripening involves: (i) pulp structural changes (softening) due to cell wall degradation (pectins, hemicellulose, cellulose), (ii) taste changes (sugar increase, organic acid decrease), and (iii) color changes (chlorophyll degradation, carotenoid and/or anthocyanin accumulation). Anthocyanin biosynthesis is promoted by pronounced day and night temperature fluctuations associated with high light intensities during ripening. Starch represents the main form of sugar accumulation in pome fruits (2 to 4\% FW at commercial harvest), which is converted to soluble sugars post-harvest.

\vspace{0.5em}
Harvest criteria traditionally rely on destructive indexes like flesh firmness (penetrometer), soluble solids content (refractometer), and starch content (Lugol's solution). Non-destructive techniques, such as Near-Infrared Spectroscopy (NIR), offer rapid, field-applicable analysis. The simplified NIR device provides the IAD (absorbance difference) index, which correlates with ripening progression, including ethylene biosynthesis, allowing for the establishment of homogeneous classes of fruit at harvest.

\vspace{0.5em}
Postharvest quality management aims to limit deterioration, primarily by slowing metabolic activity, especially respiration. Storage protocols use low temperature and Controlled Atmosphere (CA), which controls O$_2$, CO$_2$, and ethylene concentrations. For long-term CA storage (e.g., 8 months for apples), the Streif Index, calculated as [firmness/(SSC $\times$ starch index)], can be used to delimit a final harvest window (FHW). The use of 1-methylcyclopropene (1-MCP), an ethylene antagonist, is effective in reducing ethylene production and respiration rates in many climacteric fruits.


\section{\texorpdfstring{1\textsuperscript{st} Supplementary Reading - MU Guide: Budding}{1st Supplementary Reading - MU Guide: Budding}}

Budding is a method of grafting in which the scion (upper portion of the graft) is a single bud rather than a piece of stem or twig. This technique is frequently used for producing fruit trees, roses, and ornamental trees and shrubs to multiply a variety that cannot be produced from seed. Budding may also be used for topworking stone fruits, such as cherry, plum, and peach, which cannot be easily grafted with cleft or whip grafts. Budding, particularly “T” budding, is faster than other grafting techniques and typically results in a high percentage of successful unions; experienced budders may achieve 90 to 100\% take. Budding also forms a stronger union than those made with other grafting techniques. Because only a single bud is inserted, the technique allows for the production of numerous new plants even when scion wood of a new variety is scarce. Budding is well adapted to plant shoots from 1/4 to 1 inch in diameter.

\vspace{0.5em}
“T” budding should be performed when the bark of the stock slips (easily separates from the wood) and buds are fully developed. Most budding occurs from late July to early September (fall budding), where buds normally remain dormant until the following spring. Fall budding is the most common technique for producing fruit trees, although spring budding (March and April) and June budding are also possible. Young plants selected for the understock must have new, vigorous growth, and shoots on the lower 6 inches of the trunk should be removed to provide a smooth working surface 2 to 3 inches above the ground.

\vspace{0.5em}
The budstick is usually a twig from the current season's growth of the desirable variety, characterized by average vigor, health, and plump, well-developed buds. Buds on the center of the twig are generally superior. Leaves must be clipped off immediately, allowing approximately 1/2 inch of the leaf-stalk to remain as a handle.

\vspace{0.5em}
To prepare the bud for insertion, a sharp knife is required. The bud is cut from the budstick starting 1/2 to 3/4 inch below the base of the bud, making a smooth, slicing cut upward extending 1/2 to 3/4 inch above the bud. It is critical to cut the wood (shield) attached to the bud straight, as a curve leads to poor contact. Beginners are advised to leave the wood on the bud piece.
The most common cut made in the understock for budding is the T cut. The steps for the T cut and insertion are: 

\begin{enumerate} 
    \item Select a smooth, branch-free location. 
    \item Make a vertical cut about 11/2 inches long, parallel with the grain of the wood, drawing the knife upward. 
    \item Make a cross cut at the top of the vertical cut, forming a T, cutting through the bark but not into the wood. 
    \item Gently lift the bark at the junction of the T cut if the bark is not slipping properly. 
    \item Slide the base of the bud shield into the vertical slit at the top of the T cut until the top of the shield is even with or below the cross cut. 
\end{enumerate} 

\vspace{0.5em}
Prompt insertion is vital to prevent the bud or understock from drying. After insertion, the bud should be wrapped tightly, typically with rubber budding strips, starting above the bud and wrapping downward to keep the bud from being pushed out. Wrapping should cover the horizontal cut but never cover the bud itself.

\vspace{0.5em}
The bud should be checked 7 to 10 days after setting; if the bud and surrounding bark appear fresh, a union has formed. The understock should not be cut back until the bud union is completed. For fall budding, the stock is cut off near the cross of the T in spring, after the bud starts swelling. Any shoots developing from the rootstock must be removed immediately.

\vspace{0.5em}
Alternative methods include Chip Budding and Patch Budding. Chip budding involves inserting the bud shield into a small notch on the understock, where it is wrapped tightly with plastic film to conserve moisture. Patch Budding is useful for thick-barked trees, such as walnuts and pecans. Patch budding requires the bark of both the understock and the budstick to be slipping easily, and the removed patch of bark containing the bud must be the exact size of the patch cut in the understock. Double-bladed knives are often used for making the parallel horizontal cuts, usually about an inch apart. The patch must fit snugly on all four sides. Immediately after insertion, the patch must be wrapped, ensuring all four cuts are covered but the bud itself is left exposed. Tension should be released about 10 days after budding by making a single vertical cut on the back side of the understock.


\section{\texorpdfstring{2\textsuperscript{nd} Supplementary Reading - Grafting}{2nd Supplementary Reading - Grafting}}

Grafting is defined as the act of joining two plants together. The upper part of the graft is the scion, which becomes the top of the plant, while the lower portion is the understock (or rootstock), which provides the root system or part of the trunk. A piece of plant grafted between the scion and understock is known as an interstem (or interstock). Grafting is employed because some cultivars do not come true from seeds or are difficult/impossible to reproduce from cuttings. It is also a method for topworking large trees, using a root system better adapted to soil or climate, or producing dwarf plants via special understocks or interstems. Grafting is not a means of developing new varieties, nor does it intermingle the genetic structure of two different plants; any resulting different shoot types are the result of a chimera, a type of mutation.

\vspace{0.5em}
For a good graft union, the cambium (a single cell layer between the wood and bark) of the scion must line up as closely as possible with the cambium of the stock. Only plants closely related botanically generally form a good graft union, and the stock and scion must be compatible. Plants of the same genus and species can usually be grafted. Plants of different genera are less successfully grafted, for example, quince, genus Cydonia, may be used as a dwarfing rootstock for pear, genus Pyrus. Plants of different families cannot be successfully grafted for woody plants.

\vspace{0.5em}
Most grafting is done in late winter or early spring before new growth begins, after the chance of severe cold has passed but well before hot weather. Scion wood must always be dormant. Necessary materials include a sharp knife, grafting wax (hand or brush) to prevent drying, and materials like grafting tape or rubber budding strips for binding. Half-inch nails are used for veneer, bridge, and inarching grafts.

\vspace{0.5em}
Grafting techniques are determined by the size relationship between the scion and understock.
Whip Graft: This graft is suitable for stock and scions of similar diameter, preferably between 1/4 and 1/2 inch. The strongest graft results from a whip-and-tongue system. Both stock and scion require a smooth sloping cut (1 to 21/2 inches long). The tongue is formed by a downward cut about 1/2 inch long made one-third down from the tip of the slanting cut, as close to parallel with the grain as possible. The union must be wrapped and waxed.

\vspace{0.5em}
Cleft Graft: This is commonly used to topwork trees on branches not more than 21/2 inches in diameter. The stock must be sawed off at a right angle and split through the center about 2 inches deep. The scion is 1-year-old wood, about 1/4 inch in diameter, containing two or three buds. Preparing the scion involves: 
\begin{enumerate} 
    \item Starting below the lowest bud and making a long, smooth cut toward the base (1 to 11/2 inches long). 
    \item Turning the scion and making a second smooth cut of the same length so that the side containing the lowest bud is slightly thicker. 
\end{enumerate} 

\vspace{0.5em}
The scion is inserted with the thicker side outward so the cambiums match, often with two scions inserted per slit. All cut surfaces must be waxed thoroughly.

\vspace{0.5em}
Bark Graft: This technique is performed when the bark slips, usually in early spring. The stock is cut at a right angle, and a slit or two slits are made in the bark. The dormant scion (4 to 5 inches long) is cut to form a shoulder and a long, smooth cut 11/2 to 2 inches long. The scion is pushed under the bark until the shoulder rests on the stub.

\vspace{0.5em}
A common reason for graft failure is that the cambiums were not meeting properly. Other causes include incompatibility, incorrect timing, unhealthy or non-dormant scions, improper waxing, or girdling because tape was not cut or released in time.