\chapter{Literature résumés}
\setlength{\headheight}{12.71342pt}
\addtolength{\topmargin}{-0.71342pt}

This section of the course notes is designed to streamline access to the key findings from each reading material (RM), providing a concise and accessible overview of essential information. Created through experimentation with various AI platforms, this chapter also serves to enhance prompt engineering skills, exploring diverse methods of note-taking for maximum efficiency and clarity. The procedures for creating these summaries have varied, but all methods share a common approach: each RM has been fully read, with summaries and notes prepared after completing each respective subsection. By using these AI-co-op'ed approaches, these notes aim to be both a reliable reference and a resource for continuous improvement in capturing complex microbiology concepts.

\section{RM for L02 - Fruit Yield and Quality. Components and Determinants}

Fruit cultivation will normally be intended to maximize fruit yield and fruit quality \cite*{rm_01_L02_fruit_yield_quality}. The available tools for achieving this are variety selection (genetic selection) and choice of cultivation techniques, the physiological basis of which is treated in subsequent chapters. Maximizing fruit yield, often expressed as tonnes (or hekto litre in case of wine or juice) per hectares (ha), requires no further definition. Fruit yield is ultimately determined by fruit number per hectares and fruit size, understood as a product of several components. While component weights vary across fruit species, the main components can be summarized as shown in Table 1 and illustrated for cherry in Figure 1 \cite*{rm_01_L02_fruit_yield_quality}. Determinants define the amount or degree of development of each component within the genetic potential. Plant number per ha is primarily determined by the growing technique, mainly the selected planting system. The cultivation system may also determine factors such as the number of shots in raspberries and the number of rooted runners in strawberries. Other determinants are primarily physiological factors associated with growth and development, although they may also be influenced to some extent by growing technical interventions. The first main section of the course includes a review of these physiological determinants, serving as a basis for explaining and understanding the effects of different cultivation technical interventions.

\vspace{0.5em}
Maximizing fruit quality is also a primary target in growing, although fruit quality cannot be defined nearly as unambiguous as fruit yield. Fruit quality includes a variety of components that vary with fruit type and fruit use, involving different weightings in different contexts. Fruit size is an important quality component, especially for table fruit. Colour and other factors related to fruit appearance are also important, particularly those factors that are part of EU standards for the grading of the various fruit species. These components are of specific interest to fruit growers because they contribute to price differences between different grades. Components attached to fruit taste and enjoyment value—e.g., sugar, acid, and aromatics content—are also important and are expected to become even more important in the future. For industry fruit, such as sour cherries, emphasis is placed on juice colour and acidity, and for blackcurrant, on vitamin C content, despite there being no real price differences (quality premium). The concept of fruit quality can also include aspects such as the content of foreign substances like residuals from spraying of chemicals, which are of increasing importance in the complex of fruit quality. Societal priorities for health and environmental concerns, etc., also gain increasing importance in a growing technical context, and these factors are discussed in relevant, growing technical lectures.

\vspace{0.5em}
Determinants such as flower development, and particularly fruit growth and fruit development, affect the quality components. Since physiological and growing technical factors may have opposite effects on yield and fruit quality components in some cases, the fruit grower's task is to seek the best overall compromise.

\vspace{0.5em}
The components determining Yield per hectare and fruit quality, along with their primary determinants, follow a hierarchical path, as outlined in Table 1 \cite*{rm_01_L02_fruit_yield_quality}: 

\begin{enumerate} 
    \item Number of plants / ha is determined by $\leftarrow$ Planting System, and $\leftarrow$ Growing System (Rubus, strawberry). 
    \item Bud Number / plant, bud type is influenced by $\leftarrow$ Plant size, structure, $\leftarrow$ Elongation growth, shot type development, and $\leftarrow$ Bud Development. 
    \item Number of inflorescences / bud is determined by $\leftarrow$ Flower initiation. 
    \item Number of flowers/cluster is influenced by $\leftarrow$ Flower initiation and $\leftarrow$ Flower Development. 
    \item Flower quality is determined by Number of seed primordia / Flower and Position in cluster. 
    \item Initial fruit set / flower is influenced by $\leftarrow$ Pollination, fertilization, initial setting. 
    \item Final / initial fruit set is affected by $\leftarrow$ Fruit drop (June drop). 
    \item Fruit Size is determined by Number of seeds / fruits ($\leftarrow$ Pollination) and Quantity fruit flesh/seed, and is influenced by $\leftarrow$ Fruit Growth, Fruit Development. 
\end{enumerate}


\section{RM for L03 - Buds and Bud Development}

Buds are growth points which can develop into shoots, determining the plant's future extension growth (dimension), or develop flowers, thereby determining the cropping \cite*{rm_02_L03_buds_bud_development}. Bud number and bud type are important components influenced by plant size and structure, which determine the number and distribution of various shoot types, consequently influencing bud number and type at the start of the growing season. Terminal buds, and sometimes the upper side buds on last year’s long shoots, may develop new annual shoots. Lateral buds are formed in the axils of the leaves as the stem grows, culminating in terminal bud formation, with this elongation growth increasing the number of lateral buds. Other lateral buds on long shoots and end buds on short shoots may remain as short shoots or spurs, completing their short terminal growth with a terminal bud. Short shoots may mature well-developed lateral buds (often developing flowers), typical of fruiting spurs on cherry. Sleeping eyes (buds) only break after a strong stimulus, such as pruning. Flower buds of apple and pear are mixed buds, containing leaves and flowers, plus a new small bud (bourse-shoot bud), meaning a flower bud is usually followed by a new growth point. Stone fruit, conversely, has ‘simple’ or ‘naked’ flower buds, containing only leaves or flowers, resulting in the loss of a growth point at that position upon flowering, which can lead to bare areas on the shoot, typical of some sour cherry cultivars.

\vspace{0.5em}
Elongation growth, and its distribution, determine bud development during the growing season. Terminal buds on short shoots are formed early (June). Lateral buds on longer annual shoots are formed in line with leaf development, with terminal bud formation occurring from June until September, depending on growth intensity. Elongation growth and bud development are subject to correlative inhibition mechanisms, including apical dominance, which involves a polar transport of growth substances (auxins, gibberellins) from terminal buds that inhibits lateral bud growth. Heavy crop load can strongly reduce shoot elongation growth and the number of lateral buds due to competition for assimilates. Terminal bud formation occurs earlier if the ratio is high, see equation \ref{eq:root_strength}:

\begin{equation}
    \text{Root activity} = 
    \dfrac{\text{Root mass} \times \text{Root activity}}
          {\text{Number of Growth Points in the Top}}
    \label{eq:root_strength}
\end{equation}

\vspace{0.5em}
Mature trees tend to have earlier terminal bud formation (possibly June-July) than young trees (often not before September). Elongation growth is enhanced by increasing temperature and by increased nitrogen and water supply (see sections 19.2.1 and 20.2.3).

\vspace{0.5em}
The formation of a terminal bud is an abrupt process where bud scales develop instead of leaves, forming a compressed shoot inside with primordials for transitional leaves, true leaves, and possibly flowers. Bud break in the same growing season (re-growth) may occur exceptionally where correlative inhibitions are broken sharply, such as by removing shoot tips or performing early summer pruning. This is only possible until a certain date; with apple, for example, up to approximately 1$^{st}$ August, after which buds enter endo-dormancy.

\vspace{0.5em}
Buds enter the endo-dormant phase when they can no longer break due to strong influences on correlative inhibitions. Bud dormancy is lifted by a species- and variety-specific period of low temperature, termed ‘chilling’. Efficient chilling temperature is usually calculated as temperatures below +7\textdegree. In the temperate climate zone, the buds go through a physiological period of dormancy where there is no visible growth, which, if not met, can cause poor development or bud fall. The duration of dormancy varies by species (Table 1-2 shows Black currant ending dormancy in December, Apple in March). The depth and length of dormancy can be influenced by growth conditions in summer and geographical location, with cool summers accelerating termination. Dormancy is controlled by growth regulating substances, with Abscisic acid (ABA) having the strongest relationship with dormancy, while Gibberellins (GA) and/or cytokinins are growth-promoting substances. In areas with too little chilling, cultivation can be enabled by initiating the process with drought, followed by artificial defoliation (e.g., spraying with 10\% urea) and then applying a spray to interrupt dormancy, such as 0.1\% active DNOC +4\% emulsified crude oil at the first watering.

\vspace{0.5em}
Woody plants propagated by seeds must reach a certain age before flowering, termed the juvenile period or phase. This period may be short (e.g., Rubus and Ribe species), or long (e.g., 4-5 years for apple, 6-8 years for pear). The period between the end of the juvenile phase and first natural flowering is the transitional period. The juvenile period is characterized by different leaf forms and, sometimes, the transformation of short shots into woody thorns (e.g., pear). To shorten the juvenile period in apples, experiments have reduced the period to approximately a third of normal using continuous growth in greenhouse conditions with increased day length and CO$_2$ addition.


\section{RM for L04 - Chapter 2: Flower bud formation, flower development and flowering}

A proportion of the buds formed during the growing season develop into flower buds, a process that is a very important yield component \cite*{rm_03_L04_flower_bud_formation}. Flower bud formation begins with flower initiation, which occurs when there are no inhibitory effects. The time of initiation is typically summer - late summer, occurring in the approximate order of: Ribes, stone fruit, pome fruit, strawberry, and Rubus. This initiation is followed by flower differentiation.

\vspace{0.5em}
Flower bud formation is most detailed studied in apple (pome fruit), primarily because it is an important crop and because it is frequently a limiting factor, especially concerning alternation (every two-year cropping). Studies show that a bud must undergo development before initiation can occur. For the apple variety 'Cox's Orange', a critical node number of about 20 must be formed inside the bud. If the total number of nodes reaches 20 or more, a flower bud is formed, with a flower primordial in the terminal position and others in the corners of bracts (usually six to seven flower buds in total). The earlier a bud is formed (like spur buds), the longer time it has to develop, making flower formation easier. If node development is too rapid, the bud may break again in the same year (re-growth), causing delays that inhibit flower formation. Microscopically, the first sign of initiation is an increase in width of the growing point from 12-15 to 18-25 cells. Different theories exist regarding the mechanism, often involving positive and negative-acting factors, such as the Carbon/Nitrogen (C/N) ratio, where starch accumulation promotes flower bud formation. Flower bud formation may be associated with a high auxin / gibberellin content. The Source / sink ratio and flower bud formation are positively correlated. A strong negative correlation between fruit number per tree (fruit / leaf ratio) and flower bud formation explains the phenomenon of alternation. Reduced fruit growth rate, resulting in smaller fruits, leads to very large flower bud formation. Increased photosynthetic activity due to CO$_2$ addition is accompanied by an increase in flower bud formation. Increased shoot growth generally correlates negatively with flower bud formation. However, on young trees, recent studies suggest that a positive correlation can be found between shoot growth and flower bud formation on annual shoots, particularly when nutrients and water are supplied by drip irrigation. This allows the buds on annual shoots to reach their lower critical node number (14-16).

\vspace{0.5em}
In stone fruit, alternation is typically less pronounced than in apple, possibly due to fruit size-related sink strength. In 'Stevnsbær', most buds on short shoots bloom, and the negative correlation between shoot growth and flower bud formation is less strong, primarily affecting longer annual shoots. Flower initiation may depend on the rate of development rather than a critical node number.

\vspace{0.5em}
In Ribes, flower bud formation requires relatively short days (<12-14 hours) and low temperature. There is no direct effect of fruit number on flower bud formation; however, good cropping reduces shoot growth, leading to fewer buds and thus fewer flowers per bush the following year.

\vspace{0.5em}
In strawberry (short-day types), day length is the crucial factor, with initiation occurring under Danish conditions when day length falls below 11 to 14 hours. Low temperature also promotes initiation. Vegetative growth (e.g., runner formation) is promoted by long days and high temperatures.

\vspace{0.5em}
Flower development determines flower quality, which relates to the ability of the flower to set fruit. The differentiation of flower parts follows a sequence: 

\begin{enumerate} 
    \item Sepals and petals differentiate rather quickly after initiation. 
    \item Stamens and carpels differentiate next. 
    \item Anthers (pollen sacks) and ovules often only differentiate next spring. 
\end{enumerate} 

\vspace{0.5em}
In cherry ('Stevnsbær'), bud death and abortion of flower primordiums can cause mortality rates between 30-80\% of initial primordiums. In strawberries, the inflorescence is a cyme. With increasing flower order (primary, secondary, etc.), development becomes less complete, leading to inferior quality, characterized mainly by a decreasing number of pistils (Achens) per flower.

\vspace{0.5em}
Once bud dormancy is over, flowering time is mainly determined by temperature. The duration of the flowering process is also controlled by temperature; hot conditions may reduce it to one week, while cooler years may extend it beyond two weeks.


\section{RM for L05 - A model to predict the beginning of the pollen season}

The aim of the present study was to test phenoclimatographic models, comprising the Utah phenoclimatography Chill Unit (CU) and ASYMCUR-Growing Degree Hour (GDH) sub-models, on the allergenic trees Alnus, Ulmus, and Betula in order to provide a method to predict the beginning of the pollen season \cite*{rm_04_L05_model_predict_pollen_season}. This type of model relates environmental temperatures to rest completion and bud development. Flowering is a phenological event resulting from a long period of development, beginning with the initiation and differentiation of buds into flower and vegetative buds during the summer. Falling temperatures cause a gradual change into a phase of winter rest with little or no growth activity. After a period, which length apparently depends on the climate and plant species, the plant gradually reverts to a phase of active growth in the spring.

\vspace{0.5em}
As the phenologic parameter, 14 years of pollen counts (1977 to 1990) from a Burkard Volumetric Spore Trap in Copenhagen were used. The observed dates for the beginning of the pollen seasons were defined from the pollen counts as a fixed percentage (2.5\%) of total counts. The observed dates of first bloom and pollen counts deviate profoundly from one year to another, with Alnus deviating from the 30$^{th}$ December to the 1$^{st}$ April, Ulmus from the 21$^{th}$ February to the 2$^{nd}$ May, and Betula from 2$^{nd}$ April to the 9$^{th}$ May. The chilling requirement was expressed using Chill Units (CU), where one CU is defined as one hour at 6\textdegree C, the optimum Chill Unit temperature for fruit trees. CU calculations began at the first day yielding positive chilling values. The experimentally determined temperatures for the GDH function included an optimum temperature of 25\textdegree C and a critical temperature of 36\textdegree C. The base temperature for GDH was changed from 4\textdegree C to 2\textdegree C, giving a slightly better correlation.

\vspace{0.5em}
The models used were: 

\begin{enumerate} 
    \item A fixed day model, using only the GDH model with 1$^{st}$ January as fixed initiation point.
    \item A CU/GDH model, with a fixed sum of Chill Unit requirement as the initiation point for subsequent GDH accumulation. 
    \item A dynamic CU/GDH model, based on a dynamic relationship between CU and GDH. 
\end{enumerate}

\vspace{0.5em}
In the ordinary CU/GDH model, the statistically estimated requirements for the trees were: Alnus: CU = 1550, GDH = 200; Ulmus: CU = 1850, GDH = 700; and Betula: CU = 1900, GDH = 2446. The minimum standard deviation for Betula was obtained at a CU estimate of 1900, giving an average of 2446 GDH.
The dynamic CU/GDH model, equation \ref{eq:dynamic_gdh}, adjusted the necessary GDH to budbreak according to the CU obtained, proposing a non-linear S-curve function: 

\begin{equation}
    \text{GDH} = C + \dfrac{(D - C)} \cdot {1 + \exp\left(\dfrac{CU - CU_0}{A}\right)}
    \label{eq:dynamic_gdh}
\end{equation}

\vspace{0.5em}
When compared with observed dates, the dynamic model provided the best predictions: within 2-4 days for Alnus, 8-10 days for Ulmus, and 3-5 days for Betula. This deviation size is acceptable given the large temporal variation from year to year.
It is concluded that the CU and GDH relationships defined for fruit trees are generally applicable and give a reasonable description of the growth processes of other trees. Alnus tends to be completely regulated by temperature, but other parameters like the photoperiod apparently must be involved for Ulmus and Betula if predictions are to be further improved. The involvement of a dynamic relationship, though complicating the model, tends to improve the simulation of gradual changes during dormancy and budbreak. Furthermore, the results indicate that frost damage, detected using an LT$_{50}$ submodel (e.g., in 1981 and 1988), might be an important factor that can strongly affect total pollen counts.


\section{RM for L06 - Physiology of Temperate Zone Fruit Trees}

Fruit set plays an important role in modern fruit production, as a large yield of fruit is only expected if conditions for pollination and fruit set are favorable. Fruit set is a complex series of physiological events where no single element can be limiting to the overall process \cite*{rmb_01_physiology_temperate_zone_fruit_trees}. This overall process begins with the transfer of viable pollen to a receptive stigma, followed by germination, pollen tube growth, nutrient supply, and the successful fertilization of the mature embryo sac, which must occur before subsequent growth of the embryo. The events involved in fruit set are grouped into three main categories: the flowering process itself; circumstances that influence pollination and fertilization; and conditions conducive to fruit set without fertilization. All processes connected with fertilization are influenced by regulators, nutrition, and rootstock type, all of which are controllable by the grower \cite*{rmb_01_physiology_temperate_zone_fruit_trees}.

\vspace{0.5em}
The biology of bloom describes the process in which sepals and petals slowly enlarge, and the stigma(s) and stamens are exposed \cite*{rmb_01_physiology_temperate_zone_fruit_trees}. Bloom is considered to begin when 12-15\% of the flowers are open and ends when 95-100\% of the flowers are open. Cultivars generally do not bloom at the same time, and ensuring effective pollinizers requires that the bloom period of the main variety and the pollinizer overlap sufficiently. The timing of flowering varies from year to year, regardless of the absolute date, due to the differential sensitivity of various stages of bloom development to temperature stimuli \cite*{rmb_01_physiology_temperate_zone_fruit_trees}.

\vspace{0.5em}
The majority of fruit trees are self-incompatible, defined by the inability of fertile hermaphrodite seed plants to produce zygotes after self-pollination. Self-pollen is usually genetically inhibited from germination or pollen tube growth, preventing fertilization. However, even in self-fruitful cultivars, cross-pollination usually sets heavier crops. Triploid cultivars contain more pollen, which is generally less viable and has less value as a pollinizer \cite*{rmb_01_physiology_temperate_zone_fruit_trees}. Temperature has a great influence on the amount of pollen produced, with long cold winters or cold temperatures in early spring leading to a reduction in the number and vitality of pollen grains \cite*{rmb_01_physiology_temperate_zone_fruit_trees}. Bees are considered the most important pollinator insect \cite*{rmb_01_physiology_temperate_zone_fruit_trees}.

\vspace{0.5em}
The concept of the Effective Pollination Period (EPP), introduced by Williams in 1966, describes the limited period immediately following flower opening during which fertilization is possible \cite*{rmb_01_physiology_temperate_zone_fruit_trees}. The EPP duration equals the longevity of the ovule minus the time required for pollen tubes to reach the embryo sac. Ovule longevity varies significantly by cultivar; for example, the ovule longevity of 'Delicious' apple was estimated at 5 days, compared with 7-8 days for 'Jonathan'. The growth rate of compatible or partially compatible pollen is dictated by temperature. Pollen tube growth responds to temperature in a linear fashion, and a temperature response index can estimate the time required for pollen tube growth, compounded based on daily mean temperatures \cite*{rmb_01_physiology_temperate_zone_fruit_trees}.

\vspace{0.5em}
For a good fruit set, three requirements are sequential: first, the development of a strong flower bud during the previous fall, requiring sufficient photosynthate and nitrogen supply; second, a certain temperature range during and soon after bloom to ensure good pollination, pollen tube growth, and fertilization; and third, a relatively high photosynthate supply for the young developing fruit after fertilization \cite*{rmb_01_physiology_temperate_zone_fruit_trees}. Failure to satisfy these factors results in a poor fruit set and early drop of the young fruit. Early shedding of fruit is a regular feature of fruit set, often appearing in four waves at 12-14 day intervals. The third and fourth drops, called the 'June drop', are more conspicuous due to the larger size of the dropping fruit and involve a complete abscission process that includes the formation of ethylene.

\vspace{0.5em}
Parthenocarpic fruit set, where fruit sets and grows to full size without fertilization, is most widespread in pears among temperate zone fruits. Pear varieties can be classified into four groups based on their ability to set parthenocarpic fruit \cite*{rmb_01_physiology_temperate_zone_fruit_trees}: \begin{enumerate} \item Not parthenocarpic: 'Pap Pear'. \item Parthenocarpic fruit is shed at the time of June drop: 'Hardy'. \item Variably parthenocarpic: 'Bosc', 'Clapp Favorit', 'Diel', 'Madame du Pois', 'Oliver Serres', 'Bartlett'. \item Consistently parthenocarpic: 'Arabitka', 'Hardenpont', 'Passe Crassane', 'Pringall'. \end{enumerate} In most fruit, the seed is the source of Gibberellic acid (GA), which is sensitive to temperature, explaining the dependence of parthenocarpy on high temperature around bloom \cite*{rmb_01_physiology_temperate_zone_fruit_trees}.



