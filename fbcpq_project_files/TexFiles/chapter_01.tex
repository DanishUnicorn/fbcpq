\chapter*{Preface}
\setlength{\headheight}{12.71342pt}
\addtolength{\topmargin}{-0.71342pt}
These course notes have been prepared as part of the NPLK14014U course Fruit and Berry Crop Physiology and Quality at the University of Copenhagen, covering the period from September to November 2025.

\vspace{1em}
The notes compile material and reflections relevant to the course and are intended as a resource to enhance the learning experience for students. The content is shared freely and may be used as study material or as a template for structuring individual notes.

\vspace{1em}
All information is provided without responsibility for its correctness, and users are encouraged to verify data, formulas, and interpretations with the original sources and course materials.

\vspace{1em}
Please enjoy reading these notes, and feel free to reach out if you have any questions.


\chapter*{Course Description}
\setlength{\headheight}{12.71342pt}
\addtolength{\topmargin}{-0.71342pt}

\section*{Education}
MSc Programme in Agriculture


\section*{Content}
The focus is on fruit growth and fruit quality in relation to the use as fresh fruits or for processing. How is fruit growth and quality affected by the plants' physiological and genetic basis and how can it be influenced by different growing techniques and environmental factors? Similarities and differences among the fruit crop types (pit fruits, stone fruits, berries and nuts), with regard to demands in growing conditions are discussed. Furthermore, we analyze which physiological parameters are important in the different fruit species for determining yield and important quality components. Emphasis is on temperate fruits, nuts, berries and fruit vegetables, grown mainly in open field or in tunnel systems. The reference growing systems are the common commercial systems, including organic growing. The course also addresses examples of the genetic and quality variation among cultivars and the importance of different quality attributes in relation to postharvest use (fresh consumption, cooking, juice processing or fruit wine making).
In general the crop specific aspects of the following main topics will be covered:
\begin{itemize}
    \item Yield and quality components (organ development and interactions) and determinant factors
    \item Allocation of dry matter and nutrient  among sources and sinks in fruiting plants
    \item Control of vigour and plant structure by pruning and management of nutrients and irrigation
    \item Effects of preharvest factors (climate, a-biotic or biotic stresses) on internal and external quality of fruits
    \item Content and development of secondary and bioactive compounds in fruits.
    \item Maturation, ripening and assessment of optimal harvest and quality aspects of fruits and berries.
    \item Post harvest usability and sensory aspects of different cultivars and fruit types.
\end{itemize}  

In addition to fresh use,  special attention is given to production and quality of fruit juices.
Biotechnological aspects are addressed at a limited level.


\section*{Learning Outcome}
The course is targeted to students interested in plant science (Horticulture and Agriculture) and food science students who are particularly interested in fruit and berry crops and the quality and use of the raw materials/food products these crops provide.
\subsection*{Knowledge}

\begin{itemize}
    \item The physiological basis for production of fruiting crops (including fruit vegetables such as tomato and cucumber).
    \item Overview of development of the major plant organs with focus on the fruit and its quality and understand how and why it varies with genotype and preharvest growing conditions
    \item Describe the variation among the major cultivars used of fruits and berries in terms of development and quality parameters.
    \item Reflect on the importance of fruit and berries for human health
\end{itemize}

\subsection*{Skills}

\begin{itemize}
    \item Apply basic knowledge of physiology and biochemistry from plant and food science at the whole plant and organ level.
    \item Analyse a fruiting crop based on the crop specific yield and quality components.
    \item Explain how and why different techniques are used in the fruit industry and how it affects plant growth and product/fruit quality.
\end{itemize}

\subsection*{Competences}

\begin{itemize}
    \item Analyse the methods used to obtain optimal productivity and product quality.
    \item Discuss trade offs in management, such as between optimal sensory quality and storability, between yield and quality or pesticide use vs organic growing
\end{itemize}


\section*{Litterature}
Literature lists will be available from the course responsible.


\section*{Recommended Academic Qualifications}
Academic qualifications equivalent to a BSc degree is recommended.


\section*{Teaching and Learning Methods}
Besides lectures the course will include practicals, where the students are working with cultivar evaluation, quality analysis or aspects of fruit growing physiology (plant and organ development etc). Part of the hands on teaching will be field based in the experimental fruit collections at the Pometum.

\vspace*{1em}
The practicals will be made in groups, while the individual student is given the opportunity, in a major report written throughout the course, to focus on an area of special interests. Thus individual competences with emphasis on either fruit growing physiology or fruit quality aspects of fruits as raw materials for industry processing or fresh consumption can be developed. The topic of the major report are to be presented to the class in a short lecture based on a selected journal paper.
\vspace*{1em}

2 or 3 excursions will be arranged in connection with the different course subjects.

\section*{Workload}

\begin{table}[h]
    \centering
    \caption{A table with an overview over the workload for the course.}
    \label{tab:workload}
    \rowcolors{2}{white}{gray!7}
    \begin{tabular}{ l | c}
        \textbf{Category} & \textbf{Hours} \\ 
        \hline
        Lectures & 35 \\ 

        Class Instruction & 5 \\

        Preparation & 40 \\ 

        Practical exercises & 30 \\

        Excursions & 21 \\

        Project work & 75 \\

        \hline
        Total & 206 \\ 
    \end{tabular}
\end{table}

\section*{Exam}

\begin{table}[h]
    \centering
    \caption{A table with an overview over the elaborated description of the course}
    \label{tab:elaborated_description}
    \rowcolors{2}{white}{gray!7}
    \begin{tabular}{ l | p{10cm} }
        Credit & 7.5 ECTS \\ 
        
        Type of assessment & \begin{itemize}
                                \item Oral examination, 20 minutter
                                \item Written assignment, ca. 3 uger
                            \end{itemize} \\ 

        Type of assessment details & The portfolio includes a major report and 2 out of 4 additional products (e.g. exercise reports or presentation)

        Weight of exam components: Evaluation of major report 50 \%, oral examination in portfolio contents and curriculum 50\%. \\

        Examination prerequisites & Submitted and approval of the reports for theoretical and practical exercises \\ 

        Aid & All aids allowed 

        \href{https://kunet.ku.dk/study/food-science-technology-ma/Pages/topic.aspx?topicid=20fc0507-633c-455d-85ae-eb53a44d4072}{Read about how to use Generative AI on KuNet}\\

        Marking scale & 7-point grading scale \\

        Censorship form &   \begin{itemize}
                                \item No external censorship
                                \item One internal examiner
                            \end{itemize} \\

        Re-exam &   The exam is an oral exam, as for the ordinary exam. Submission of an individual major report 1 week before the oral re-exam is required. The topic may be as for the ordinary exam but in a revised version. \\ 
    \end{tabular}
\end{table}


\newpage