\setcounter{chapter}{0}
\setcounter{section}{0}
\chapter{Lecture Notes}
\setlength{\headheight}{12.71342pt}
\addtolength{\topmargin}{-0.71342pt}

\section{Lecture 01 - 02/09-2025}
\subsection{The Tropical Environment} 

\subsubsection{Aim} 
\begin{itemize} 
    \item Overview the most important aspects of tropical climates. 
    \item Ability to figure out how the climate is likely to be in certain places in the tropics. 
    \item Idea of which crop you can grow. 
\end{itemize}

\subsection{What Determines the Climate?} 
The climate is determined by several factors, including temperature and precipitation. Key aspects are the yearly average temperature and the yearly range in temperature, as some areas experience a larger difference between the highest and lowest temperatures than others. Similarly, average precipitation is important, but the yearly variation in rainfall also plays a significant role.
\subsubsection*{Core takeaway:} 
Climate is primarily defined by temperature and precipitation, considering both yearly averages and seasonal variations. Likely exam-relevant.

\subsection{Classification: Latitudes} 

\begin{itemize} 
    \item Tropical zone from 0\textdegree–23.5\textdegree (between the tropics) latitude: Here, solar radiation reaches the ground nearly vertically, more water evaporates, and the air is often moist. A dense cloud cover reduces the effect of solar radiation on ground temperature. 
    \item Subtropics from 23.5\textdegree–40\textdegree latitude: These regions receive the highest radiation in summer, have relatively thin cloud cover, and receive less moisture. 
    \item Temperate zone from 40\textdegree–60\textdegree latitude: This zone is characterized by significantly differing seasons and day lengths, less frequent climate extremes, a more regular distribution of precipitation, and a longer vegetation period. 
    \item Cold zone from 60\textdegree–90\textdegree latitude: The poles in this zone receive less heat through solar radiation, and day length varies the most. Vegetation is only possible during a few months and is often sparse. 
\end{itemize}

\subsubsection*{Core takeaway:} 
Earth's climate zones are classified by latitude, each with distinct characteristics regarding solar radiation, temperature, precipitation, and vegetation periods. Likely exam-relevant.


\subsection{Circles of Latitude and Longitude} 
\subsubsection{Earth's Movement and Tropical Rain Belt} 
The Earth spins around its axis, akin to a top, a process known as Earth's rotation. Simultaneously, it orbits or revolves around the Sun. The tropical rain belt runs along the equator and extends to about the Tropic of Cancer (23.5\textdegree north latitude) and Tropic of Capricorn (23.5\textdegree south latitude). By approximately 30\textdegree north and south latitude, the air cools enough to sink back to the surface, creating high pressure (H) and drier conditions.
\subsubsection{Earth's Orbit and Solar Energy} The Earth's revolution around the sun takes 365.24 days. At the equator, the Earth rotates at roughly 1,700 km per hour. The Earth is closest to the sun (perihelion) on January 3rd at 147 million km, moving faster at 27 km/s. It is furthest from the sun (aphelion) on July 4th at 152 million km, moving slower. Solar energy is relatively constant, approximately 400 W/m$^2$/year. About 300 W/m$^2$/year is lost as terrestrial re-radiation, leaving a surplus of 100 W/m$^2$ at the surface. Most of the radiation is absorbed by the Earth and warms it. Some of the outgoing infrared radiation is trapped by the Earth’s atmosphere, which also contributes to warming.

\subsubsection*{Core takeaway: }
Earth's rotation and revolution influence climate patterns, including the tropical rain belt, and its interaction with solar energy dictates global temperatures. Likely exam-relevant. 


\subsection{The Tropics} 
The tropics are characterized by a high input of solar radiation and high maximum temperatures, with little variation in temperature. Water supply is the most significant variable, marked by high rainfall variability and high rainfall intensity. The tropics cover 42\% of the Earth's surface. 
\subsubsection{Characterize the tropics !} 
\subsubsection{Precipitation} 
Precipitation patterns in the tropics include: 
\begin{itemize} 
    \item Wet climate (between 5\textdegree and 10\textdegree of the equator). 
    \item Wet dry climate (between 10\textdegree and 20\textdegree). 
    \item Two wet seasons: typically 1000-2000 mm (e.g., Salvador, Abidjan). 
    \item Two shorter rainy seasons (e.g., Nairobi). 
    \item One long rainy season: monsoonal, 750-1500 mm (e.g., Manila). 
    \item One short rain season: 250-750 mm (e.g., Darwin, Hyderabad). 
    \item Dry climate (e.g., Alice Springs, Lima, Khartoum)
\end{itemize}


\subsubsection*{Core takeaway:} 
The tropics receive high solar radiation and experience consistent high temperatures, with water supply and significant rainfall variability being defining features across different precipitation zones. Likely exam-relevant.


\subsection{Three Major Biomes} 
A biome is defined as a community of similar plants and animals occupying a large area. The three major biomes are Forest, Savanna, and Desert. 

\subsubsection{Tropical biomes and annual precipitation (mm)} Tropical biomes exhibit extremely high biodiversity, encompassing 50\% of the world’s terrestrial plant and animal species, despite covering only about 6\% of the world’s land area.

\subsubsection*{Core takeaway:} 
The tropics host three major biomes—Forest, Savanna, and Desert—which are critical for global biodiversity, harboring half of the world's terrestrial species in a small land area. Likely exam-relevant.


\subsection{Deforestation} 
Before human intervention, rainforests covered 15\% of the Earth's land area, but today they cover only 6\%. In the last 200 years, the total area of rainforest has decreased from 1,500 million hectares to less than 800 million hectares. A third of tropical rainforests have been destroyed in just the last 50 years. Approximately 119,000 - 150,219 km$^2$ are lost each year, affecting the world's most spectacular ecosystems.

\subsubsection*{Core takeaway: }
Deforestation has drastically reduced tropical rainforest coverage, leading to a significant loss of these vital ecosystems globally. Likely exam-relevant.


\subsection{Daily Weather Cycle in the Tropical Rainforest} 
In the morning, the sun shines and heats up the ground, causing hot and wet air to rise. In the afternoon, dark clouds form, bringing rain and thunderstorms to the rainforest.


\subsection{Prevailing Winds} 
\subsubsection{Latitudinal Variation in Evapotranspiration and Precipitation} 
(figure, see slide 9)
\subsection{Remember!} 
\begin{itemize} 
    \item Hot air weighs less than cold air. 
    \item Hot air can contain more water than cold air. 
    \item Air will flow from areas of high pressure towards areas with low pressure. 
    \item Condensation of water releases energy. 
    \item The temperature of the air drops approximately 1 degree for every 100 m, or 0.5 degrees if the air contains water. 
    \item Objects moving in the northerly or southerly direction will be deflected clockwise in the northern hemisphere and counter-clockwise in the southern hemisphere (Coriolis force) (see also Slide 10). 
\end{itemize}

\subsubsection*{Core takeaway:} 
Atmospheric dynamics, driven by temperature, pressure, and the Coriolis force, dictate air movement, moisture content, and temperature changes critical for understanding weather patterns. Likely exam-relevant.


\subsection{Coriolis Force} 
When the Earth rotates, a point close to the equator moves much faster than a point at one of the poles. This movement creates specific patterns on Earth and affects winds and ocean currents.

\subsubsection*{Core takeaway: }
The Coriolis force, a result of Earth's rotation, deflects moving objects and significantly influences global wind and ocean current patterns. Likely exam-relevant.


\subsection{Tropical Storms} 
Tropical storms include Hurricanes (in the Caribbean and United States) and Typhoons (in the Pacific Ocean). These storms are characterized by wind speeds exceeding 115 km/hour, low pressure, and a circular pattern of isobars with a diameter of 150-650 km. They bring extreme rainfall (up to 200 mm/day) and steep gradients that produce high wind speeds. 

\subsubsection{Cyclones Around Australia}
\subsection{Monsoons} 
Monsoons are large-scale sea breezes that occur when the temperature on land is significantly warmer or cooler than the temperature of the ocean. These temperature imbalances happen because oceans and land absorb heat in different ways.

\subsubsection*{Core takeaway:} 
Tropical storms like hurricanes and typhoons are intense low-pressure systems with high winds and extreme rainfall, while monsoons are seasonal wind shifts caused by differential heating of land and sea. Likely exam-relevant.


\subsection{Southeast Asian Rainforests} 
Southeast Asian rainforests experience four different seasons: the winter northeast monsoon, the summer southwest monsoon, and two inter-monsoon seasons. 

\begin{itemize} 
    \item The northeast monsoon season (November to March) has steady winds from the north or northeast, originating from Siberia, which bring typhoons and other severe weather. The east coasts of the Southeast Asian islands receive heavy rains during this time. 
    \item The southwest monsoon season (May to September) has less wind and is slightly drier, though it still rains every day. 
    \item During the inter-monsoon seasons, the winds are light. All seasons are hot and humid, with very little seasonal variation in temperature. 
\end{itemize}

\subsubsection*{Core takeaway:} 
Southeast Asian rainforests experience distinct monsoon seasons driven by regional wind patterns, resulting in varied rainfall but consistently hot and humid conditions year-round. Likely exam-relevant.


\subsection{Tropical Rainforests} 
Tropical rainforests are characterized by a type of tropical climate with no dry season, meaning all months have an average precipitation value of at least 60 mm (2.4 in). There are no distinct summer or winter seasons; it is typically hot and wet throughout the year, with both heavy and frequent rainfall. Around the equator, there are two seasons with heavy rainfall, receiving up to 10 meters a year. As one moves away from the equator, it becomes a bit drier in some months, but there is still more than 2 meters of rain annually. Most of the rainfall does not reach the ground directly, as the trees act as a canopy and catch the rain. 

\subsubsection{Rainforest Burned Down in South America} (image, see slide 14)
\subsubsection*{Core takeaway:} 
Tropical rainforests are defined by continuous high rainfall, consistent high temperatures year-round, and the significant role of their dense canopy in intercepting precipitation. Likely exam-relevant.


\subsection{Tropical Desert} Major tropical desert areas include the Sahara and Kalahari deserts in Africa, Arabian, Iranian and Thar Deserts in Asia, Arizona and Mexican deserts in North America, and the Great Australian Desert. 

\subsubsection{Oasis with Date Palm} (image, see slide 15) 

\subsubsection{External Resources / Ecosystem Map} 
\textit{[Requires further research: This section primarily provides links to external resources (YouTube and a NOAA ecosystem map) and does not contain descriptive content within the slides themselves.]}

\subsection{A Simple Illustration of the Major Crop Types in Relation to Climate} 
\textit{[Requires further research: This slide title suggests an illustration but the content is not provided.]}

\subsubsection*{Core takeaway:} 
Tropical deserts are extensive arid regions found across multiple continents, characterized by very low precipitation and extreme temperatures. Likely exam-relevant.

\section{Lecture 02 - 04/09-2025}
\subsection{Fertility of Tropical Soils}
The plan for the day includes discussing factors of soil formation, aspects of soil fertility, an introduction to tropical soil types, and the role of soil organic matter and soil fertility. A group exercise on how to improve the fertility of degraded soils is also part of the plan.

\subsubsection{What is soil?} 
Soil is defined as the unconsolidated mineral or organic material on the immediate surface of the Earth that serves as a natural medium for the growth of land plants (see also Slide 2).

\paragraph*{Core takeaway:} This section introduces the course, the instructor, the agenda, and a fundamental definition of soil. Exam relevance marker: Likely exam-relevant (definition of soil).


\subsection{Soil Profile and Formation} 
\subsubsection{Soil Profile} Figure: An illustration of a soil profile, depicting layers down to bedrock (see slide 2).

\subsubsection{What is soil?} 
This slide reiterates the definition of soil (see also Slide 1).

\subsubsection{Soil formation} 
\subsubsubsection{Weathering} 
Weathering is the disintegration and decomposition of solid rock material, encompassing both chemical and physical processes. The most important form of chemical weathering involves H+ ions from water penetrating rock mineral structures and displacing ions like K+, Ca2+, Mg2+, and Al3+. This process causes minerals to break down into clay and leads to the leaching of ions.

\subsubsection{Primary particles} 
\subsubsubsection{Mineral fraction} 
The mineral fraction of soil is categorized by particle size: 

\begin{itemize} 
    \item Sand size fraction: 50 $\mu m$ - 2 $mm$ 
    \item Silt size fraction: 2 $\mu m$ - 50 $\mu m$ 
    \item Clay size fraction: < 2 $\mu m$
\end{itemize}

\paragraph*{Core takeaway:} 
Soil formation involves weathering of bedrock into primary particles, which are classified by size. 

Exam relevance marker: Likely exam-relevant (weathering definition, particle sizes).

\subsection{Soil Components and Factors of Soil Formation} 
\subsubsection{Clay size fraction} 
\begin{itemize} 
    \item Clay size fraction: < 2 $\mu$m 
\end{itemize}

\subsubsection{Soil organic matter} 
The pool of soil organic matter is defined as biologically derived soil material (see also Slides 14, 15, 16). It consists of: 
\begin{itemize} 
    \item A large fraction of humic substances 
    \item Fresh and partly decomposed plant residues 
    \item A small fraction of living soil microbial biomass
\end{itemize}
\subsubsection{Soil texture} 
This slide poses a question: "A soil with 35 \% sand, 35 \% clay and 30 \% silt called?" 

\textit{[Requires further research: The answer to the soil texture question is not provided directly on the slide.]}

\subsubsection{Soil Structure} 
\textit{[Requires further research: This headline is present, but no content is provided for 'Soil Structure' on this slide.]}


\subsubsection{Factors of soil formation} The factors influencing soil formation include (see also Slides 4, 5): 
\begin{itemize} 
    \item Parent material 
    \item Climate 
    \item Topographical position 
    \item Biological factors 
    \item Time 
\end{itemize}

\subsubsection{Parent Material} 
Parent material refers to in situ rocks (bedrock) (see also Slide 4).

\paragraph*{Core takeaway:} 
This section details soil particle sizes, defines soil organic matter, lists the five key factors of soil formation, and introduces parent material. Exam relevance marker: Likely exam-relevant (soil organic matter components, factors of soil formation).

\subsection{Parent Material and Climate in Soil Formation} \subsubsection{Parent Material} 
Bedrock consists of sedimentary or metamorphic rock brought to the surface by geological processes. Parent materials are derived from the weathering of bedrocks and interact with other soil formation factors to determine the secondary minerals formed (see also Slide 3).

\subsubsection{Climate} A hot and humid climate leads to intensive weathering and leaching (see also Slide 3). This removes Aluminum (\textit{Al}) and Silicon (\textit{Si}), resulting in the formation of the clay mineral kaolinite, which has a low Cation Exchange Capacity (CEC) and is less fertile (see also Slides 7, 8, 10, 12, 15). Kaolinite's chemical formula is Al2Si2O5(OH)4. The topographical position of a soil on a landscape will affect the impact of climatic processes (see also Slide 5).

\paragraph*{Core takeaway:} 
Parent material originates from bedrock, and climate, especially hot and humid conditions, drives intensive weathering, leaching, and the formation of low-fertility clay minerals like kaolinite. Exam relevance marker: Likely exam-relevant (impact of climate on weathering and clay formation).

\subsection{Other Factors of Soil Formation and Soil Fertility Introduction} 
\subsubsection{Topography} 
Erosion and leaching cause minerals to accumulate at the bottom of a slope (see also Slide 3).

\subsubsection{Biological factors} 
Biological factors contribute to soil formation through (see also Slide 3): 

\begin{itemize} 
    \item Faunal activity (mixing of soil) 
    \item Plant activity (rooting, formation of acids, prevents leaching of nutrients) 
\end{itemize}

\subsubsection{Time} 
The age of soils varies significantly; for example, most Danish soils are approximately 12,000 years old, while some African soils are 500 million years old (see also Slide 3).

\subsubsection{Soil fertility} 
Soil fertility is defined as the ability of soil to sustain and provide essential nutrients and create favorable conditions for plant growth and development (see also Slides 11, 13, 15, 16, 17). Key aspects of soil fertility include:
\begin{itemize} 
    \item Nitrogen 
    \item Processes affecting inputs and losses of N 
    \item Phosphorus 
    \item Phosphorus Fixation 
    \item Base cations 
    \item Cation Exchange Capacity (CEC) 
    \item Base Saturation 
\end{itemize}

\paragraph*{Core takeaway:} 
Topography, biological activity, and time are crucial soil-forming factors. Soil fertility, defined by its capacity to support plant growth, hinges on nitrogen, phosphorus, base cations, CEC, and base saturation. Exam relevance marker: Likely exam-relevant (definition of soil fertility, factors of soil formation).

\subsection{Nitrogen and Phosphorus in Agroecosystems} \subsubsection{Nitrogen in Agroecosystems} 
Figure: A diagram illustrates the nitrogen cycle within agroecosystems (see slide 6). Inputs to the system include N fixation, deposition, organic fertilizer, and inorganic fertilizer. Outputs consist of leaching, denitrification, and NH3 volatilization. Internal processes within the soil involve mineralization, ammonification, nitrification, immobilization, and plant uptake. The forms of nitrogen include organic N, plant N, $NH_4^+$, and $NO_3^-$.

\subsubsection{P availability in soil} 
Figure: A diagram shows phosphorus availability in soil (see slide 6). Phosphorus exists in stable, labile, and organic forms, as well as in the soil solution P. Inputs of phosphorus come from manure, waste, and mineral fertilizer. Outputs include plant uptake and loss of P, as well as leaching. A significant process affecting phosphorus is Phosphorus Fixation (see also Slides 5, 11, 12, 13, 17).

\paragraph*{Core takeaway:} 
Nitrogen and phosphorus cycles in agroecosystems involve complex inputs, outputs, and internal processes that determine nutrient availability. Exam relevance marker: Likely exam-relevant (understanding N and P cycles, P fixation).

\subsection{Cations and Cation Exchange Capacity (CEC)} \subsubsection{Base and acid cations in soil} 
\subsubsubsection{Base cations} These positively charged ions include Calcium ($Ca_2^+$), Magnesium ($Mg_2^+$), Potassium ($K^+$), Sodium ($Na^+$), and Ammonium ($NH_4^+$) (see also Slides 5, 9, 11, 12, 13).

\subsubsubsection{Acid Cations} 
These include Aluminium ($Al_3^+$), Iron ($Fe_3^+$), and Hydrogen ($H^+$).

\subsubsection{Clay Minerals} 
Common clay minerals are classified as 1:1 type (e.g., Kaolinite) and 2:1 type (e.g., Smectite) (see also Slides 4, 8, 10, 15).

\subsubsubsection{Isomorphous substitution} 
Isomorphous substitution is a process where a higher charged ion is replaced with a lower charged ion within the mineral structure, resulting in a net negative charge. Examples include $Si_4^+$ being replaced with $Al_3^+$ in the tetrahedral sheet, and $Al_3^+$ being replaced with $Mg_2^+$ in the octahedral sheet.

\subsubsection{Cation Exchange Capacity (CEC)} 
CEC is defined as the amount of exchangeable cations that a soil can adsorb (see also Slides 4, 5, 8, 10, 12, 15, 17). It is expressed in terms of centimoles of positive charge adsorbed per unit of mass, specifically in centimol positive charge per kg of soil (cmol(+)/kg).

\paragraph*{Core takeaway:} 
Soil cations are categorized as base or acid, and clay minerals exhibit a net negative charge due to isomorphous substitution, which contributes to the soil's Cation Exchange Capacity (CEC). Exam relevance marker: Likely exam-relevant (definitions of base/acid cations, isomorphous substitution, CEC).

\subsection{Cation Exchange and Clay Mineral CEC Values}
Figure: This figure illustrates cation exchange on a plant root, where $H^+$ ions are exchanged for other cations from the soil solution (see slide 8). It also shows cation exchange occurring on the surfaces of organic material and clay particles.

Different clay minerals possess varying CEC values and properties: 
\begin{table}[h]
    \centering
    \caption{An overview of clay minerals and their properties}
    \label{tab:clay_minerals}
    \rowcolors{2}{white}{gray!7}
    \begin{tabular}{ l | c | c | c | c }
        \textbf{Type of clay mineral} & \textbf{Type} & \textbf{CEC /cmol (+)/kg} & \textbf{Expansible} & \textbf{pH dependent charge} \\ 
        \hline
        Kaolinite   & 1:1       & 1--10    & No     & Most   \\ 
        Smectite    & 1:2       & 80--120  & Yes    & Little \\ 
        Vermiculite & 1:2       & 120--150 & Partly & Little \\ 
        Illite      & 1:2       & 20--50   & No     & Medium \\ 
        Allophane   & Amorphous & 50--150  & No     & Most   \\ 
    \end{tabular}
\end{table}

\paragraph*{Core takeaway:} 
Cations are exchanged between plant roots, soil solution, and charged surfaces of clay and organic matter, with different clay minerals having distinct CEC values and characteristics influencing their behaviour. Exam relevance marker: Likely exam-relevant (mechanism of cation exchange, comparative CEC values of different clay minerals).

\subsection{pH Dependent Charge and Base Saturation} \subsubsection{pH Dependent Charge} 
Figure: A graph visually represents the relationship between pH and charge, indicating how soil charge can be pH-dependent across a range (e.g., pH 4.0, 5.0, 6.0, 7.0) (see slide 9).

\subsubsection{\% Base Saturation} 
Base Saturation is defined as the percentage of the exchange complex that is saturated with base cations (see also Slides 5, 12). It is measured in centimoles of positive charge. Adsorbed cations are in equilibrium with solution cations. The formula for Base Saturation is: $$ \text{Base Saturation} = \text{100\%} \times \frac{\text{Base Cations}}{\text{CEC}} $$ An example calculation is provided: Given CEC = 40 cmol(+)/kg, $K^+$ = 16 cmol/kg (= 16 cmol(+)/kg), $Ca^{++}$ = 4 cmol/kg (= 8 cmol(+)/kg), $Mg^{++}$ = 2 cmol/kg (= 4 cmol(+)/kg). Base saturation = 100 x (16+8+4) / 40 = 70\%.

\paragraph*{Core takeaway:} 
Soil charge can be pH-dependent, and Base Saturation quantifies the proportion of exchange sites occupied by base cations, indicating soil fertility. Exam relevance marker: Likely exam-relevant (definition and calculation of base saturation).

\subsection{Estimating CEC and Base Saturation for Tropical Soils} 
\subsubsection{Estimate the Cation Exchange Capacity (CEC) of the two soils} 
\subsubsection{Exercise 1} 
This exercise provides characteristics for two soil types for estimation: 
\begin{itemize} 
    \item Ultisol: Kaolinite, pH 4.6, 60\% clay, 4\% organic matter (see also Slides 11, 12) 
    \item Vertisol: Smectite, pH 7.2, 20\% clay, 2\% organic matter (see also Slides 11, 12, 13) 
\end{itemize} 

The calculation of CEC would involve considering CEC contributions from both clay and organic matter. A table providing average CEC values for different clay minerals is given (Avg. 4 cmol(+)/kg for Kaolinite, Avg. 95 cmol(+)/kg for Smectite, etc.) (see also Slide 8).

\subsubsection{Exercise 2} 
This exercise requires calculating the base saturation of the two soils (Ultisol and Vertisol) based on the CEC values calculated in Exercise 1, using given base cation contents.

\paragraph*{Core takeaway:} 
Exercises are presented to estimate CEC based on clay mineral type and organic matter content, and subsequently calculate base saturation, for different tropical soil types. Exam relevance marker: Likely exam-relevant (practical application of CEC and base saturation calculations).

\subsection{Fertility Comparison and Tropical Soil Types} \subsubsection{Discuss which soil is more fertile and how?} This question prompts a comparison of the fertility of Ultisol and Vertisol, using the following base cation content data:
\begin{table}[h]
    \centering
    \caption{Cation content in different soil types}
    \label{tab:soil_cations}
    \rowcolors{2}{white}{gray!7}
    \begin{tabular}{ l | c | c | c | c }
        \textbf{Soil type} & \textbf{$K^+$} (cmol) & \textbf{$Mg^{2+}$} (cmol) & \textbf{$Ca^{2+}$} (cmol) & \textbf{$Na^+$} (cmol) \\ 
        \hline
        Ultisol  & 0.08 & 0.1 & 0.3 & 0   \\
        Vertisol & 2.1  & 2.4 & 3.2 & 0.2 \\
    \end{tabular}
\end{table}

(Table, see slide 11) 

%\subsubsection{Tropical soil types} Soils are classified according to the United States Department of Agriculture (USDA) Soil Taxonomy. The tropical soil types listed are: \begin{itemize} \item Oxisol (see also Slide 12) \item Ultisol (see also Slide 12) \item Alfisol (see also Slide 12) \item Vertisol (see also Slides 12, 13) \item Andisol (see also Slide 13) \item Aridisol (see also Slide 13) \end{itemize}
%\subsubsection{Oxisols} Oxisols are soils with an oxic horizon, meaning they are highly weathered and dominated by Iron- and Aluminum oxides, with some kaolinite present. They typically have less than 10\% weatherable minerals. Oxisols are formed under conditions of intensive weathering and leaching in hot and humid climates.
%Core takeaway: This section provides data for comparing soil fertility between Ultisols and Vertisols and introduces the major classifications of tropical soil types, with a detailed description of Oxisols. Exam relevance marker: Likely exam-relevant (characteristics of tropical soil types, comparison of fertility).
%\subsection{Characteristics of Tropical Soil Orders} \subsubsection{Oxisols} Continuing from the previous slide, Oxisols are characterized by: \begin{itemize} \item Low CEC (see also Slides 4, 5, 7, 8, 10, 15, 17) \item High P fixation (see also Slides 5, 6, 11, 13, 17) \item Low pH \end{itemize}
%\subsubsection{Ultisol} Ultisols possess an argillic horizon (clay accumulation) and are subject to intensive weathering and leaching in hot and humid climates (see also Slides 10, 11). Their characteristics include: \begin{itemize} \item More weatherable minerals than Oxisols \item Well drained \item Low CEC \item Low level of bases \item High P fixation \item Low pH \end{itemize}
%\subsubsection{Alfisol} Alfisols also feature an argillic horizon (clay accumulation) (see also Slide 5). Key attributes are: \begin{itemize} \item Higher base saturation than Ultisol (see also Slides 5, 9) \item Seasonal moisture deficit \item Transition zone to semi-arid climates \item Medium CEC \item > 35\% base saturation \item Medium fertility \end{itemize}
%\subsubsection{Vertisol} Vertisols are distinguished by a high content of expanding clay minerals (see also Slides 10, 11, 13).
%Core takeaway: This section details the distinct characteristics, particularly in terms of CEC, P fixation, pH, and base saturation, for Oxisols, Ultisols, and Alfisols, and introduces Vertisols. Exam relevance marker: Likely exam-relevant (comparative characteristics of different tropical soil orders).
%\subsection{Further Characteristics of Tropical Soil Orders} \subsubsection{Vertisols} Continuing the description, Vertisols are typically: \begin{itemize} \item Formed from highly basic rocks and in climates that are seasonally humid \item Sticky when wet \item Hard when dry \item Neutral – alkaline pH \item Medium – high content of basic cations (see also Slides 5, 7, 9, 11) \item High fertility (see also Slides 5, 11, 12, 15, 16, 17) \end{itemize}
%\subsubsection{Andisol} Andisols are: \begin{itemize} \item Young soils developed from volcanic material \item High contents of organic matter (see also Slides 3, 14, 15, 16, 17) \item High content of basic cations \item High fertility \item High P fixation \end{itemize}
%\subsubsection{Aridisols} Aridisols are: \begin{itemize} \item Found under arid soil moisture regimes (i.e., in dry areas) \item Typically sandy \item Too dry for crop production unless irrigated \item Often used for grazing \item Low content of organic matter \end{itemize}
%Core takeaway: This section completes the overview of tropical soil orders, highlighting the high fertility of Vertisols and Andisols due to their unique properties, and the challenges associated with Aridisols in dry regions. Exam relevance marker: Likely exam-relevant (characteristics of Vertisols, Andisols, and Aridisols).
%\subsection{Soil Organic Matter and Carbon Cycling} \subsubsection{Soil organic matter and fertility} Soil organic matter largely comprises fresh and partly decomposed plant residues, with a smaller fraction consisting of living soil microbial biomass (see also Slides 3, 15, 16). Figure: A diagram illustrates the flow of carbon in the soil-atmosphere system (see slide 14). Atmospheric carbon is fixed through photosynthesis. Carbon is lost to the atmosphere through respiration. Organic carbon enters the soil via above- and below-ground litter. Some carbon transforms into soil organic carbon, while some is lost to the atmosphere through soil respiration.
%Core takeaway: Soil organic matter is critical for fertility, composed mainly of plant residues and microbial biomass, and plays a central role in the global carbon cycle. Exam relevance marker: Likely exam-relevant (composition of SOM, basic carbon cycle).
%\subsection{Factors Affecting Soil Organic Matter and Importance in Tropics} \subsubsection{Soil organic matter and fertility} \subsubsubsection{Inputs:} Factors contributing to soil organic matter include: \begin{itemize} \item Crop/vegetation \item Farming practice/residue use \item Manure applications \end{itemize}
%\subsubsubsection{Outputs:} Factors influencing the loss or transformation of soil organic matter include: \begin{itemize} \item Climate (temperature, precipitation) \item Soil properties (texture, mineralogy, stabilization, pH, etc.) \item Biological factors (decomposer organisms, etc.) \item Chemical factors (quality of residue, etc.) \item Soil management (tillage, drainage, etc.) \end{itemize}
%\subsubsection{Soil organic matter in tropical soils – why bother?} Soil organic matter is particularly important in tropical soils because: \begin{itemize} \item These soils are often weathered and low in nutrients \item They frequently contain clay types with low CEC \item They are erodible \item They experience high intensity rainfall events \item There is serious water deficiency in semi-arid and arid tropics \end{itemize}
%Core takeaway: Soil organic matter levels are a balance of inputs and outputs influenced by climate, soil properties, biological and chemical factors, and management. Its importance is amplified in tropical soils due to inherent challenges like low nutrient content and erodibility. Exam relevance marker: Likely exam-relevant (factors influencing SOM, reasons for SOM importance in tropics).
%\subsection{Soil Organic Carbon (SOC) and Soil Health} \subsubsection{Soil organic matter in tropical soils – why bother?} Tropical soils have been most depleted, yet their productivity must be increased to meet the demands of a growing population (see also Slide 15).
%\subsubsection{SOC is an important indicator of soil health} \subsubsubsection{Soil Organic Carbon} Management options to increase soil organic matter (SOM) / soil organic carbon (SOC) include: \begin{itemize} \item Tillage \item Crop rotations \item Perennials \item Root system \item Cover crops \item Crop residues \item Animal manure \item Biochar \end{itemize} SOC influences soil health through its Physical, Chemical, and Biological impacts: \begin{itemize} \item Physical: Aggregate stability, improved soil structure, improved soil porosity, bulk density, water holding capacity \item Chemical: Cation Exchange Capacity (CEC), soil pH, binds heavy metal (see also Slides 4, 5, 7, 8, 9, 10, 12, 15, 17) \item Biological: Earthworms, soil microorganisms, soil ecosystem \end{itemize}
%Core takeaway: SOC is a crucial indicator of soil health, with various management practices available to increase it, leading to significant physical, chemical, and biological benefits in the soil. Exam relevance marker: Likely exam-relevant (importance of SOC, management options, benefits of SOC).
%\subsection{Strategies for Enhancing Soil Fertility and Carbon Pool} \subsubsection{Reduction of P fixation} Figure: Chemical structure showing CO–O– Al3+ (see slide 17). This illustrates how organic matter can chelate aluminum, thereby reducing P fixation (see also Slides 5, 6, 11, 12, 13).
%\subsubsection{Reduction of Al toxicity} Figure: Chemical structure showing CO–O– Al3+ (see slide 17). Organic matter also helps in the reduction of Al toxicity.
%\subsubsection{Improve soil structure} Figure: Diagram showing how organic material and clay contribute to soil structure (see slide 17).
%\subsubsection{Strategies for Enhancing the Soil Carbon Pool} The management options to increase Soil Organic Matter (SOM) listed are: \begin{itemize} \item Tillage \item Crop rotations \item Perennials \item Root system \item Cover crops \item Crop residues \item Animal manure \item Biochar \end{itemize}
%\subsubsection{Theoretical exercise: How to increase soil fertility of degraded soils?} This exercise involves discussing possible ways to improve the fertility of degraded soils in groups (see also Slide 18). Group inputs count as the deliverable.
%Core takeaway: Enhancing the soil carbon pool through various management strategies directly improves soil fertility by reducing P fixation and Al toxicity, and improving soil structure. Exam relevance marker: Likely exam-relevant (benefits of SOM, management strategies).
%\subsection{Group Exercise: Management Options for Degraded Soils} \subsubsection{Potential Management Options to increase SOM:} The potential management options to increase soil organic matter (SOM) in degraded soils include: \begin{enumerate} \item Integration of legumes as intercrops or in rotation 2. Inorganic fertilizer 3. Manure (livestock) 4. Green manure, mulching, residue retention 5. Agroforestry techniques (including fallowing) 6. No tillage
%\subsubsection{Questions:} For each management option, the following questions are to be discussed: \begin{enumerate} \item What are the benefits of the option? 2. Which problems could (potentially) limit the adoption? 3. What are possible solutions to the problems/limitations?
%Core takeaway: This section outlines a practical exercise for identifying and evaluating various management options, from legumes to no-tillage, aimed at increasing soil organic matter and improving degraded soil fertility, considering both benefits and potential limitations. Exam relevance marker: Likely exam-relevant (management practices for soil fertility, critical thinking about their adoption).

